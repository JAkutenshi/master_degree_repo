\subsubsection{Система портов в YARP}

Коммуникация в YARP реализована согласно шаблону "Наблюдатель": особоые объекты портов доставляют сообщение до любого количества наблюдателей (тоже портов), в любое количество процессов, распределенных по любому множеству машин, используя подходы, в основе которых лежит некоторое количество протоколов. Все порты принадлежат какому-либо процессу, не обязательно одному. Любое соединение может использовать различные протоколы или различные физические среды коммуникации. Для использования доступен ряд протоколов, доступные для достижения лучших результатов для посавленных перед разработчиками целей:

\begin{description}
    \item [TCP] устойчивый, стоит исползовать для гарантированной доставки сообщений;
    \item [UDP] быстрее TCP, но не имеет гарантий доставки сообщений;
    \item [multicast] подходит для доставки одинаковой информации до множества целей;
    \item [shared memory] используется для локального взаимодействия, внутри одного процесса. Данный протокол выбирается автоматически, когда это возможно без вмешательства со стороны программистов. 
\end{itemize}
