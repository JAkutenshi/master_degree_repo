% https://tex.stackexchange.com/questions/286094/insert-code-keywords-inline?utm_medium=organic&utm_source=google_rich_qa&utm_campaign=google_rich_qa

\subsection{О загрузке тестов}
    Как говорилось ранее, тесты могут работать в нескольких потоках и в ходе выполнения они могут исполняться на разных физических ядрах процессора. Из-за различных счетчиков тактов на разных ядрах вычисление может быть не точным. Одним из решений является использование утилиты \textit{taskset}, которая позволяет ограничить выполнение процесса на наборе физических ядер компьютера. В итоге, используя ключи -a для применения ограничения ко всем потокам запускаемого процесса теста и -c для указания номера ядра, ограничиваем запуск процесса и всех его потоков на одном физическом ядре: \textit{\$ taskset -ac 0 <test_exec>} 

    Кроме того стоит отметить влияние на качество тестов такой опции, как масштабирование тактовой частоты ядер процессора. Современные процессоры могут менять тактовую частоту для, например, энергосбережения на ноутбуках. Важно настроить этот параметр так, чтобы для всех тестов были равные условия: одинаковая неизменная частота. Для этого использовалась утилита "cpupower" на хост-машине относительно docker-конетйнеров, выставляя для всех ядер параметр \textit{"performance"}, устанавливающий для ядра максимальную тактовую частоту. Итоговая команда для применения этой настройки для всех процессоров (в коде google benchmark идет проверка для всех ядер): \textit{\# cpupower -c all frequency-set -g performance}.
