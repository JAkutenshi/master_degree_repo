\subsection{О способе вычисления времени работы}

Для вычисления времени работы участков кода используется ассемблерная инструкция \it{RDTSC} (Read Time Stamp Counter), которая возвращает количество тактов процессора с момента его последнего сброса.

У этого способа есть определенные проблемы:
\begin{itemize}
    \item Из-за влияния работы кэшей процессора будет теряться точность вычислений. Решение: делать несколько итераций вычислений требуемого фрагмента в цикле.
    \item Проблемы многопоточности, в частности влияние обработки соседних потоков, а так же смена обрабатывающего ядра процессора, на котором, скорее-всего, используется другое значение счетчика тактов. Решение: отслеживание на каком ядре работает, сериализация идентификатора ядра. Решение представленно в статье: https://www.intel.com/content/dam/www/public/us/en/documents/white-papers/ia-32-ia-64-benchmark-code-execution-paper.pdf В Google Benchmarks (пока) это не реализовано. (TODO?) Так же возможным решением является привязка процесса к конкретному ядру процессора. Для Linux это делается при помощи утилиты taskset.
\end{itemize}


