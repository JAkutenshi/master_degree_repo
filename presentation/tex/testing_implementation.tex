\section{Реализация}


\subsection{Конфигурация окружения}
\begin{frame}{Конфигурация окружения}
	Тут спеки машины
	
	Тут картинка из иерархии docker-образов
\end{frame}

\subsection{Архитектура}
\begin{frame}{Конвейер}
	Тут картинка из шагов:
	\begin{itemize}
		\item Получения результатов с
		\begin{itemize}
			\item ROS benchmark 
			\item YARP benchmark
			\item MIRA benchmark
			\item ORoCoS benchmark
		\end{itemize}
		\item Разбиения на таблицы при помощи утилиты из 
		\begin{itemize}
			\item набор из основного json и дополнительных, которые будут сливаться с основным
			\item input.xml, описывающий входные json-ы и выходные csv
		\end{itemize}
		\item Обработка таблиц при помощи R и утилиты преобразования из
			\item набора из множества csv таблиц
			\item analyse.xml, описывающий что читать из таблиц, каким R-скриптом это обрабатывать
		\item В конце результаты: rmd html pdf
	\end{itemize}
\end{frame}

\subsection{Бенчмарк}
\begin{frame}{Пример алгоритма реализации бенчмарка}
	Картинка с описанием связей узлов, сообщений и макросов тестов
\end{frame}

\subsection{Преобразователь}
\begin{frame}{Реализация преобразователя json to csv}
	Картинка с описанием шагов:
	\begin{itemize}
		\item XML configuration
		\item JSON merge
		\item Преобразование (и схранение отдельным csv) JSON в структуру список-списков-списков(1) 
		\item Преобразование (1) в список-списков средних значений измерений с СКО по множеству многофакторных бенчмарков;
		\item Распиливание этого списка по множеству требуемых csv файлов
	\end{itemize}
\end{frame}

\subsection{Анализатор}
\begin{frame}{Реализация анализатора таблиц}
	Картинка с описанием шагов:
	\begin{itemize}
		\item XML configuration
		\item Преобразование таблицы в R-chunk и запись его в файл
		\item Вызов knit и файлы результатов на выходе
	\end{itemize}
\end{frame}
