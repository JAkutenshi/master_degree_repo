\section{Планирование тестирования}

\subsection{Предмет тестирования}

\begin{frame}{Сравнение существующих фреймворков}
	\vspace{-0.4cm}
	\begin{table}[h!]
		\small
		\centering
		\def\arraystretch{1.3}
		%	\resizebox{\textwidth}{!}{
		\begin{tabular}{|p{1.2cm} p{0.5cm} p{0.5cm} p{0.5cm} p{2.5cm} p{2.3cm}|}
			\hline
			\textbf{Название}     & \textbf{Код} & \textbf{Wiki} & \textbf{Год} & \textbf{Тип}    & \textbf{ЯП} \\ \hline
			\rowcolor{gray!30}
			ROS          & Да           & Да                            & 2018 & Гибридная              & C++, Python  \\ \hline\hline
			\rowcolor{gray!30}
			MIRA & Да & Да & 2018 & P2P     & C++, Python, JS   \\ \hline\hline
			\rowcolor{gray!30}
			YARP         & Да           & Да                            & 2018 & P2P     & C++, Python, Java \\ \hline\hline
			MOOS         & Да           & Да                            & 2018 & Централизованная       & C++, Java         \\ \hline
			OROCOS  & Да & Да  & 2016 & Гибридная              & C++, Python, Simulink    \\ \hline
			ASEBA        & Да           & Нет                           & 2018 & Распределенная         & Собственный язык  \\ \hline		
			OpenRTM & Да           & Нет                           & 2016 & Гибридная              & C++, Java, Python        \\ \hline
			URBI         & Да           & Нет                           & 2016 & Централизованная       & C++, Java, urbiscript    \\ \hline
		\end{tabular}
		%	}
	\end{table}

\end{frame}

\begin{frame}{Сравнение реализаций отобранных для тестирования фреймворков}
	\begin{table}[!h]
		\centering
		\footnotesize
		\def\arraystretch{1.3}
		\begin{tabular}{|p{2.7cm}|p{2cm}|p{2cm}|p{2cm}|}
			\hline
			& \textbf{ROS} & \textbf{MIRA} & \textbf{YARP} \\ \hline
			\textbf{Централизованные сервисы} & Сервисы поиска, именования, сервис параметров & Нет & Сервис имен \\ \hline
			\textbf{Взаимодействие между узлами} & Топики, параметры, сервисы & Каналы, RPC  & Порты, топики \\ \hline
			\textbf{Протоколы коммуникации} & TCP, UDP, собственный протокол rosserial & IPC, TCP & ACE, TCP, UDP, IPC \\ \hline
			\textbf{Формат сообщений} & Бинарный & Бинарный, XML, JSON  & Бинарный \\ \hline
		\end{tabular}
	\end{table}
\end{frame}


\subsection{Факторы производительности}
\begin{frame}{Факторы производительности}
	Общий фактор: \vspace{-0.5em}
	\begin{table}[]
		\footnotesize
		\def\arraystretch{1.3}
		\begin{tabular}{|l|l|l|l|l|l|}
			\hline
			\multirow{2}{*}{Размер сообщений} & 1Кб & 4Кб & 16Кб & 64Кб        & 256Кб       \\ \cline{2-6} 
			& 1Мб & 4Мб & 16Мб & \multicolumn{2}{l|}{64Мб} \\ \hline
		\end{tabular}
	\end{table}
	ROS: \vspace{-1.5em}
	\begin{table}[]
		\footnotesize
		\def\arraystretch{1.3}
		\begin{tabular}{|l|l|l|l|l|l|}
			\hline
			Размер буфера          & 1                  & 10                 & 100   & 1000        & 10000        \\ \hline
			Количество подписчиков & 1                  & 2                  & 4     & \multicolumn{2}{l|}{8}     \\ \hline
			Способ взаимодействия  & \multicolumn{2}{l|}{Издатель-подписчик} & \multicolumn{3}{l|}{Сервис-клиент} \\ \hline
		\end{tabular}
	\end{table}
	MIRA: \vspace{-0.5em}
	\begin{table}[]
		\footnotesize
		\def\arraystretch{1.3}
		\begin{tabular}{|l|l|l|}
			\hline
			Локализация модуля    & Один процесс & Различные процессы \\ \hline
			Способ взаимодействия & Каналы       & RPC                \\ \hline
		\end{tabular}
	\end{table}
	YARP: \vspace{-0.5em}
	\begin{table}[]
		\footnotesize
		\def\arraystretch{1.3}
		\begin{tabular}{|l|l|l|l|l|}
			\hline
			Тип порта & С буфером & Без буфера                  & \multicolumn{2}{l|}{RPC} \\ \hline
			Протокол  & TCP       & UDP* & FastTCP      & ShMem     \\ \hline
		\end{tabular}
	\end{table}
\end{frame}
