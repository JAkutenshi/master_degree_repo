\documentclass[conference]{IEEEtran}


\usepackage[utf8]{inputenc}
\usepackage[T1,T2A]{fontenc}
\usepackage[russian,english]{babel}

\renewcommand{\sfdefault}{cmss}
\renewcommand{\rmdefault}{cmr}
\renewcommand{\ttdefault}{cmt}
\renewcommand\IEEEkeywordsname{Ключевые слова}

\usepackage{booktabs}
\usepackage{csquotes}
%\usepackage{fontspec}
%\setmainfont[Ligatures=TeX]{Times New Roman}

\usepackage[%
style=ieee,
url=true,
defernumbers=true,
sorting=none,
bibencoding=utf8,
backend=biber,
language=auto,    % get main language from babel
autolang=other,
]{biblatex}
\addbibresource{refs.bib}

\begin{document}
	\title{Performance problem of multi-agent frameworks for autonomous robots software development}
	\author{
		\IEEEauthorblockN{Mikhail Efremov}
		\IEEEauthorblockA{
			Saint-Petersburg \\
			Electrotechnical University\\
			SPbETU "LETI" \\
			Email: jakutenshi@gmail.com}
	}

	\maketitle
	\selectlanguage{russian}
	\begin{abstract}
		Данная работа выделяет ряд проблем и задач, выполняемых таким промежуточным ПО, как фреймворки для разработки робототехнических приложений для автономных роботов. Данная статья делает упор на обнаружение критических областей для производительности в многоагентной архитектуре, так как подобный подход является наиболее пригодным для разработки рассматриваемых робототехнических фреймворков. Так же были выделены для исследования ряд существующих и наиболее используемых  фреймворков для рассмотрения способов решения проблем разработки такой архитектуры, выделяя подходы и технологии, которые необходимо протестировать для получения различных характеристик производительности распределенной системы.
	\end{abstract}

	\begin{IEEEkeywords}
		robotics, frameworks, middleware, performance, multi-agent systems
	\end{IEEEkeywords}

	\section{Введение}
		\chapter*{Введение}
\addcontentsline{toc}{chapter}{Введение}

Для программирования роботов доступно множество различных версий фреймворков с различными принципами работы, написанные на разных языках программирования и под разные платформы. В связи с тем, что появляются новые разработки, возникают новые задачи для автономных роботов, а так же технологии разработки ПО для них - возникает желание рассмотреть доступные и развивающиеся в данный момент решения и проанализировать с целью установки характеристик производительности и сравнения по полученным параметрам, чтобы разработчики могли обосновывать свой выбор при разработки приложений для автономных роботов. При этом необходимо учитывать как соответствие фреймворков возможным общим критериям (лицензия, статус разработки), так и важным для конкретной области: разработки ПО (программного обеспечения) для роботов. Для выполнения тестирования, следует определиться с тем, какие задачи, выполняемые фреймворком, являются значимыми для производительности системы в целом, какие компоненты системы требуется протестировать.

Для выполнения тестирования требуется использовать корректные инструменты. Тестирование производительности можно выполнять множеством различных методов, технические детали, лежащие в основе драйверов тестирования производительности различны, например, для разных архитектур процессоров. В данной работе рассматривается сравнение ряда фреймворков для автоматизации проведения тестирования производительности.

Составив план тестирования, определившись с тестовыми случаями и реализацией тестовых модулей, требовалось проанализировать полученные результаты и составить сравнительные выводы по итогам тестирования производительности.

\textit{Объектом исследования} в данной работе является множество ППО (промежуточного ПО) для разработки прикладного ПО автономных роботов.

\textit{Предметом исследования} является производительность подмножества наиболее доступного и используемого МАРППО (многоагентного робототехнического ППО).

\textit{Целью исследования} является получение результатов тестирования производительности для наиболее доступного и используемого \marm{}.
	\section{Обзор предметной области }
		Для обсуждения конкретных решений требуется из всего множества существующих фреймворков выбрать наиболее подходящие для данного исследования.

\subsection{Критерии сравнения аналогов} 

\subsubsection{Наличие открытого исходного кода} для лучшего понимания работы фреймворка, а так же лучшего понимания результатов тестирования и возможного улучшения тестовых задач желательно иметь возможность прочитать исходный код реализации функционала, влияющего на результат тестирования. Это может быть, например, реализация коммуникации между приложениями, которые были написаны при помощи исследуемого фреймворка.

\subsubsection{Наличие документации} написание обоснованных и корректных тестовых задач под конкретный фреймворк очень затруднительно, если отсутствуют инструкций по его использованию. Без наличия документации разработка приложений становится слишком сложной и корректность их выполнения не гарантируется.

\subsubsection{Текущий статус разработки} проекты, которые больше не поддерживаются разработчиками вполне могут рассматриваться для исследования, но в них скорее-всего используются устаревшие подходы, что приведет к низким показателям производительности.

\subsubsection{Архитектура фреймворка} разработка робототехнических систем налагает определенные ограничения и требования. Отдельным и самым важным критерием является надежность и устойчивость к ошибкам \cite{blasco2012multiagent}.  Поскольку любой программный продукт, даже хорошо протестированный, содержит какие-либо ошибки, включая те, которые приводят к неожиданному завершению. В случае завершения работы всего программного комплекса из-за ошибки отдельного модуля робот теряет работоспособность. Это особенно важно в таком приложении робототехнических фреймворков, как программирования команд или роя автономных роботов: потеря работоспособности ключевых узлов команды приводит к остановке выполнения поставленной ими задачи. Примером может являться исследование множеством роботов местности или выполняющая задачи команда из разнородных, выполняющих различные роли автономных роботов.

Таким образом, требуется как минимум распределенная архитектура. Чем ближе архитектура будет к P2P (peer-to-peer, одноранговая сеть) и многоагентным системам, тем устойчивее будет все робототехническое ПО. Для данного исследования будут рассматриваться гибридные P2P фреймворки и близкие к чистым P2P.

\subsubsection{Наличие инструментов для мониторинга и конфигурации системы} для анализа промежуточного ПО и обслуживаемых им приложений требуются такие инструменты, как мониторы используемых ресурсов и системы журналирования. Кроме того, развертывание большого распределенного ПО для проведения тестов является трудоемкой задачей и крайне желательны инструменты конфигурации и развертывания системы на целевой ОС и аппаратном обеспечении.

\subsubsection{Поддержка различных языков программирования} несмотря на то, что в этой работе наибольший интерес представляет именно промежуточный уровень системы, наличие альтернатив между различными языками программирования прикладного слоя робототехнической системы является преимуществом, поскольку позволяет в зависимости от задачи выбрать между, например, низкоуровневым программированием с возможным преимуществом в производительности и высокоуровневыми языками с наличием удобных для разработки интерфейсов и прикладных библиотек.

В данной работе не столь важны критерии:
\begin{itemize}
	\item Поддержки ограничений системы реального времени, поскольку это относится не к скорости работы, а к предсказуемости системы. Наличие данного пункта является преимуществом в целом, но относительно производительности оказывается не существенным.
	\item Поддержка конкретных операционных систем, поскольку рассматривается вопрос производительности слоя абстракции между, собственно, ОС и прикладным ПО.
	\item Наличие инструментов симуляции, инструментов с графическим пользовательским интерфейсом и набора прикладных библиотек с реализациями наиболее распространенных алгоритмов, поскольку это относится к функциональному уровню системы, ближе к прикладному. Вопросы производительности отдельных инструментов, которые могут требоваться при разработке, отладке и администрированию робототехнических систем не относятся к проблематике потребления ресурсов на поддержание каркаса, основы системы - фреймворка.
\end{itemize}


\subsection{Представленные для рассмотрения фреймворки}

\subsubsection{ROS} один из наиболее распространенных фреймворков, имеется и обширная документация, и открытый исходный код, разработка продолжается, широко используется. Имеет гибридную архитектуру, средства мониторинга и автоматизации развертывания системы. Для разработки по-умолчанию есть возможность использовать C++ и Python 2.7. \cite{ros-man}

\subsubsection{MIRA} этот проект активно разрабатывается, имеется открытый исходный код и обширная документация. Имеет децентрализованную архитектуру \cite{einhorn2012mira}, реализован на C++. Имеются возможности для ведения журналирования приложений, мониторы состояния коммуникаций и ресурсов системы. Для разработки предлагается использовать C++, Python и JavaScript. \cite{mira-man}

\subsubsection{MOOS} развивающийся проект, имеется документация и исходный код. На данный момент разрабатывается бета-версия MOOS 10. Проблема обеих версий: клиент-серверная архитектура, которая является спорным решением для разработки робототехнических систем из-за проблем устойчивости ПО к ошибкам. По этой причине в данной работе этот фреймворк рассматриваться не будет. \cite{moos-man}

\subsubsection{OROCOS/Rock} очень распространенный фреймворк, один из немногих поддерживающих ограничения реального времени. Документация обширная, разработка в основном ведется над набором инструментов Rock. Используется гибридная децентрализованная архитектура. Разработка ведется на C++, для разработки прикладных программ предоставляются такие языки, как C++, Python, Simulink \cite{blasco2012multiagent}. Имеются инструменты для развертывания и мониторинга системы. \cite{orocos-man,rock-man}

\subsubsection{ASEBA} для программирования используется концепция языков программирования пятого поколения: GUI, блоки, коннекторы. Разработка является скорее обучающим продуктом с коммерческой составляющей в виде конкретной модели робота thymio. Рассматриваться в данной работе не будет. \cite{aseba-site}

\subsubsection{SmartSoft} разрабатывающийся проект с обширной документацией. Для реализации компонентов используется C++. Практически не используется децентрализация, основной шаблон взаимодействия клиент-сервер. Кроме того, используется многопоточный подход, а не многопроцессный \cite{smartsoft-man}, что ставит под вопрос общую устойчивость всей системы. В данной работе рассматриваться не будет.

\subsubsection{YARP} активно разрабатывающийся фреймворк с открытым исходным кодом. Имеется обширная и подробная документация. Является одним из немногих практически полностью децентрализованных фреймворков, кроме того, поддерживает ограничения систем реального времени. Разрабатывается в основном на C++ и поддерживает такие языки как C++, Python, Java, Octave. Инструменты мониторинга и развертывания приложений имеются. \cite{yarp-man}

\subsubsection{OpenRTM-aist} распространенный фреймворк с открытым исходным кодом, является реализацией стандарта RT-middleware. Основная проблема: часть документации, при скромном содержании, на японском языке. Архитектура гибридная децентрализованная, имеются инструменты мониторинга, журналирования и развертывания, языки разработки: C++, Java, Python. \cite{openrtmaist-man}

\subsubsection{URBI} данный фреймворк имеет много недостатков: приостановленная разработка, о чем свидетельствует последнее изменение от 2014 года \cite{urbi-repo} и не работающий сайт самого проекта \cite{urbi-site}, централизованная архитектура. Рассматриваться в данной работе не будет. 

В таблице \ref{tab:frameworks} отображено соответствие найденных фреймворков предложенным выше критериям.

\begin{table*}[h!]
	\scriptsize
	\centering
	\caption{Соответствие найденных робототехнических фреймворков выделенным критериям}
	\label{tab:frameworks}
	\def\arraystretch{1.5}
	\begin{tabular}{lp{2cm}p{2cm}p{2cm}lp{2.7cm}p{2cm}}
		\toprule
		\textbf{НАЗВАНИЕ}     & \textbf{ОТКРЫТЫЙ КОД} & \textbf{НАЛИЧИЕ ПОНЯТНОЙ ДОКУМЕНТАЦИИ} & \textbf{ПОСЛЕДНИЕ ИЗМЕНЕНИЯ} & \textbf{АРХИТЕКТУРА}        & \textbf{ИНСТРУМЕНТЫ МОНИТОРИНГА} & \textbf{ПОДДЕРЖКА ЯП} \\ 
		\midrule 
		ROS          & Да           & Да                            & Недавно             & Гибридная          & Да                      & C++, Python               \\ 
		MIRA         & Да           & Да                            & Недавно             & Децентрализованная & Да                      & C++, Python, JavaScript   \\ 
		MOOS         & Да           & Да                            & Недавно             & Централизованная   & Да                      & C++, Java                 \\ 
		OROCOS/Rock  & Да           & Да                            & 2016                & Гибридная          & Да                      & C++, Python, Simulink     \\ 
		ASEBA        & Да           & Нет                           & Недавно             & Распределенная     & Да                      & Собственный язык          \\ 
		SmartSoft    & Да           & Да                            & Недавно             & Распределенная     & Да                      & C++                       \\ 
		YARP         & Да           & Да                            & Недавно             & Децентрализованная & Да                      & C++, Python, Java, Octave \\ 
		OpenRTM-aist & Да           & Нет                           & 2016                & Гибридная          & Да                      & C++, Java, Python         \\ 
		URBI         & Да           & Нет                           & 2016                & Централизованная   & Да                      & C++, Java, urbiscript     \\ 
		\bottomrule
	\end{tabular}
\end{table*}

Таким образом, в исследовании будут учавствовать следующие фреймворки:
\begin{itemize}
	\item ROS
	\item MIRA
	\item OROCOS/Rock
	\item YARP
	\item OpenRTM-aist
\end{itemize}



	\section{Выбор метода решения}
		Отобрав фреймворки с децентрализованной структурой, требуется определить те части, которые оказывают наибольшее влияние на производительность. Для этого требуется:
\begin{itemize}
	\item Разобрать базовые проблемы децентрализованных систем.
	\item Выявить на какие аспекты производительности каждая из проблем влияет.
	\item Рассмотреть какие решения были приняты в выбранных для исследования фреймворках.
\end{itemize}

На основании полученных данных требуется выделить задачи и отслеживаемые характеристики для тестирования фреймворков. 
	\section{Описание метода решения}
		\subsection{Проблемы децентрализованных систем}
Для распределенной децентрализованной многоагентной системы важны следующие аспекты \cite{blasco2012multiagent}:

\begin{table*}[h!]
	\scriptsize
	\centering	
	\caption{Реализация критических для производительности задач для отобранных фреймворков}
	\label{tab:solutions}
	\def\arraystretch{1.5}
	\begin{tabular}{lp{3.5cm}p{3cm}p{3cm}p{1.9cm}p{2cm}}
		\toprule
		\textbf{НАЗВАНИЕ} & \textbf{ЦЕНТРАЛИЗОВАННЫЕ СЕРВИСЫ} & \textbf{ВЗАИМОДЕЙСТВИЕ МЕЖДУ УЗЛАМИ} & \textbf{МЕХАНИЗМЫ КОММУНИКАЦИИ} & \textbf{ТИП СООБЩЕНИЙ} & \textbf{ВЗАИМОДЕЙСТВИЕ С АППАРАТУРОЙ} \\		
		\midrule     
		ROS         & Сервисы поиска, именования, сервис параметров & Топики, параметры, сервисы         & TCP, UDP, собственный протокол rosserial                     & Бинарный                     & Инкапсуляция в узлах\\ 
		MIRA         & Нет                                           & Топики, RPC                        & Внутрипроцессное взаимодействие, TCP                         & Бинарный, XML, JSON  & Инкапсуляция в узлах и RPC-API     \\ 
		OROCOS/Rock  & Сервисы поиска, именования                    & Порты, сервисы, события, параметры & CORBA, TCP, UDP, SSL,UNIX Sockets, MQueue, EtherCAT, CanOPEN & Сериализация на основе CORBA & Инкапсуляция в узлах и RPC-API \\ 
		YARP         & Сервис имен                                   & Порты, топики                      & ACE, TCP, UDP, внутрипроцессное взаимодействие               & Бинарный                 & Инкапсуляция в узлах и динамически подключаемые библиотеки    \\ 
		OpenRTM-aist & Сервис имен                                   & Порты, сервисы, параметры          & TCP, UDP, SSL,UNIX Sockets, CORBA                            & Сериализация на основе CORBA & Инкапсуляция в узлах и RPC-API\\ 
		\bottomrule
	\end{tabular}
\end{table*}

\subsubsection{Тип архитектуры} большая часть фреймворков использует гибридную архитектуру, поскольку такой подход позволяет избежать ряд проблем, а именно:
\begin{itemize}
	\item Не требуется инкапсулировать в каждый узел системы общие сервисы, как, например, службу поиска других узлов: эту задачу выполняет специально выделенный узел. Так же поступают и с агенствами - контейнерами, создающие, регистрирующие и хранящие узлы-агенты.
	\item Уменьшение нагрузки на сеть коммуникаций, поскольку вся нагрузка по общим для всех узлов запросам смещается в сторону обработки на выделенных узлах. Это приводит к появлению критических узлов в системе, в случае отказа которых весь робот может оказаться неспособным продолжать выполнение поставленных задач. Таким образом, в таком подходе требуется отдельно рассматривать производительность таких критических узлов системы, как служба поиска узлов, служба именования и, возможно, другие выделенные сервисы.
\end{itemize}
В случае, если используется максимально децентрализованная архитектура, то возникают следующие проблемы:
\begin{itemize}
	\item Увеличивается сложность узлов, а так же нагрузка на обработку данных между ними. Тем не менее то, насколько эффективно используются возможности фреймворка по взаимодействию между узлами, зависит от разработчика на прикладном уровне.
	\item Увеличивается нагрузка на систему коммуникации фреймворка. Под системой коммуникации понимается набор технологий и протоколов, обеспечивающих обмен данными между узлами системы. Часто для обеспечения коммуникации используются отдельные узлы, создаваемые самим фреймворком, будь то брокеры объектных запросов (ORB) при использовании CORBA или реализации шаблона взаимодействия \enquote{издатель-подписчик} при помощи топиков (от англ. topic - тема), которые имеют свою внутреннюю реализацию.
\end{itemize}

\subsubsection{Тип взаимодействия между узлами} при использовании многоагентной архитектуры используется несколько разных способов взаимодействия между узлами-агентами:
\begin{itemize}
	\item Порты, реализующие независимое чтение из входных портов, запись во входные и взаимодействие \enquote{один ко многим}.
	\item Топики, реализующие шаблон \enquote{издатель-подписчик} и взаимодействие \enquote{многие ко многим}.
	\item События, реализующие шаблон \enquote{наблюдатель} и асинхронное взаимодействие \enquote{один ко многим}.
	\item Сервисы, предоставляющие реактивный способ поведения узлам: возможность отправлять запрос от узла-клиента на узел-сервис, реализуя взаимодействие выполнения удаленных процедур.
	\item Свойства, предоставляющие возможность менять состояние узлов при помощи селекторов параметров агента (get/set). Реализация может отличаться в зависимости от архитектуры: гибридная архитектура представляет выделенный узел-сервис - службу свойств, децентрализованная инкапсулирует свойства узлов непосредственно в агентах системы.
\end{itemize}

Большинство фреймворков дает выбор: какой тип взаимодействия между узлами использовать, но, в случае с реализацией сервисов может возникнуть простой системы до тех пор, пока не работающий или перегруженный обработкой запросов узел, предоставлявший требуемый функционал, не будет восстановлен. Реализация отдельных типов взаимодействий влияет практически на все аспекты производительности: задержку, вычислительные ресурсы, использование памяти, энергопотребление. Для каждого из фреймворков следует тестировать каждый из способов взаимодействия по всем показателям производительности.

\subsubsection{Механизмы коммуникации} для доставки сообщений обычно используются различные подходы и протоколы. Обычно разработчики пишут свои решения на основе стека TCP/IP с некоторыми оптимизациями, например реализация узлов как отдельных потоков внутри процесса-агенства. Это позволяет использовать общую память процесса и ускорить взаимодействие между узлами внутри процесса-агенства, которые могут располагаться в пределах одного вычислительного устройства. Это влечет за собой возможность потерять доступ ко всем узлам агенства в случае неисправностей из-за ошибок, допущенных при реализации агентов. Эта проблема решается репликацией агенств на вычислительной системе. Кроме того для доставки сообщений между узлами используются как более высокоуровневые технологии (CORBA, ICE, ACE), так и приближенные к аппаратной части (EtherCAT, I2C, CANBus). Выбор механизма доставки данных между узлами влияет на задержку получения информации, на пропускную способность между узлами, целостность данных, а так же потребление ресурсов вычислительной системы: чем более высокоуровневая технология, тем больше. Дополнительный функционал на уровне коммуникации, например QoS или шифрование так же влияет на производительность. При наличии такого функционала его следует тестировать отдельно.

\subsubsection{Тип сообщений} сообщения в зависимости от их структуры влияют на производительность. Бинарные типы сообщений имеют меньший объем, а следовательно на их передачу требуется меньше энергии и времени. Структурированные типы, вроде XML и JSON, хорошо поддаются анализу для разработчика, но могут занимать больше времени на обработку своих данных, а так же требуют больше времени на передачу данных между узлами.

\subsubsection{Способ взаимодействия с аппаратурой} существует два основных способа предоставить интерфейс от аппаратной части системы к прикладному ПО:
\begin{itemize}
	\item Инкапсулировать взаимодействие с аппаратурой в отдельных узлах, отображающих устройства.
	\item Использовать промежуточный слой - сервер, который отвечает на запросы узлов и устройств.
	\item Динамически связывающиеся библиотеки.
\end{itemize}

От способа обращения будет зависеть отзывчивость системы на внешнее взаимодействие. Кроме того, различные реализации может использовать различное количество ресурсов. Кроме того, в случае использования дополнительного слоя абстракции с клиент-серверным взаимодействием между аппаратным слоем системы и фреймворком является уязвимым подходом с точки зрения устойчивости системы.

\subsection{Решения проблем в различных реализациях фреймворков}

На таблице \ref{tab:solutions} показано сопоставление возникающих проблемы и их решений для отобранных для исследования фреймворков \cite{blasco2012multiagent,bruyninckx2001orocos,einhorn2012mira,metta2006yarp,mohamed2008survey,elkady2012comprehensive}.

\subsection{Рассматриваемые области тестирования}
\begin{itemize}
	\item Централизованные сервисы, если они имеются во фреймворке с целью установления их быстродействия, устойчивости, потребления ресурсов памяти.
	\item Способы взаимодействия между узлами с целью установления задержки передачи сообщений между узлами, возможных затрат ресурсов на обработку сообщений.
	\item Коммуникация между узлами многоагентной системы с целью установления пропускной способности, затрат вычислительных ресурсов на передачу данных. При использовании CORBA и ACE - объем потребляемой памяти на поддержку коммуникации этими методами. При использовании шифрования - задержку передачи сообщения и затраты вычислительных ресурсов на преобразование данных.
	\item Формат передачи данных между узлами с целью установления скорости сериализации и десериализации, объема передаваемых данных, скорости извлечения данных в случае, если информация сжимается.
	\item Интерфейсы между аппаратной частью и фреймворком с целью установления задержки реакции системы на окружение.	
\end{itemize}

	\section{Заключение}
		\chapter*{\todo{Заключение}}
\addcontentsline{toc}{chapter}{Заключение}

Кратко (на одну-две страницы) описать основные результаты работы, проанализировать их соответствие поставленной цели работы, показать рекомендации по конкретному использованию результатов исследования и перспективы дальнейшего развития работы.
		
	\printbibliography
	
	\newpage
	\begin{otherlanguage}{english}
	\begin{abstract}
		This paper highlights a number of problems and tasks for such middleware as robotics software development frameworks for autonomous robots. This article emphasizes the detection of critical areas for multi-agent architecture performance, because this approach is most suitable for the development of the considered robotic frameworks. Also, a number of existing and most used frameworks were selected for research to consider ways to solve the problems of developing such an architecture, highlighting approaches and technologies that need to be tested to obtain various characteristics of a distributed system's performance.
	\end{abstract}
	\end{otherlanguage}

\end{document}