В ходе исследования были отобраны 5 фреймворков по различным критериям, среди которых самые важные: открытый исходный код, понятная и доступная документация, децентрализованная или гибридная архитектура.

В ходе исследования многоагентной архитектуры робототехнических фреймворков были выявлены 6 основных источников влияния на производительность данного промежуточного слоя системы:

\begin{itemize}
	\item Наличие в системе централизованных узлов.
	\item Типы взаимодействия между узлами системы.
	\item Механизмы коммуницирования между узлами системы.
	\item Дополнительный функционал в системе коммуницирования: например шифрование или QoS.
	\item Типы сообщений, используемых для передачи информации между узлами системы.
	\item Реализация интерфейса доступа к аппаратной части автономного робота
\end{itemize}


Исходя из выявленных проблем была составлена таблица решений этих задач отобранными фреймворками, а так же сформулированы основные показатели производительности системы в зависимости от контекста рассматриваемой проблемы.

В данной работе не рассматривалось некоторое количество фреймворков из-за централизованной архитектуры. В теории, опущенные из изучения разработки могут иметь интерес для тестирования производительности, но из-за разницы архитектур имеется большая разница в возникающих проблемах и задачах. Кроме того было показано, что сильно централизованные архитектуры плохо применимы к разработке робототехнических приложений для автономных роботов.

В дальнейшем, на основе полученной информации, планируется провести тестирование отобранных фреймворков в контексте выявленных важных для производительности областей в многоагентных робототехнических фреймворках.