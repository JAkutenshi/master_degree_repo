Объектом исследования данной статьи являются многоагентные фреймворки для разработки ПО для автономных роботов (далее "фреймворки") с целью установления важных для производительности элементов системы.

Под фреймворками понимается набор промежуточного (middleware) ПО, находящегося по уровню абстракции между ОС и прикладными приложениями, предназначенного для управления неоднородностью аппаратного обеспечения с целью упрощения и снижения стоимости разработки ПО.
Кроме того, в состав фреймворков входит набор библиотек и, опционально, инструментов для разработки прикладных программ, обслуживающих систему, в данном случае - автономного робота.

Для программирования роботов доступно множество различных версий фреймворков с различными принципами работы, написанные на разных языках программирования и под разные платформы. В связи с тем, что появляются новые разработки, возникают новые задачи для автономных роботов, а так же технологии разработки ПО для них - возникает желание рассмотреть доступные и развивающиеся в данный момент решения и проанализировать с целью установки характеристик производительности и сравнения по полученным параметрам, чтобы разработчики могли обосновывать свой выбор при разработки приложений для автономных роботов. При этом необходимо учитывать как соответствие фреймворков возможным общим критериям (лицензия, статус разработки), так и важным для конкретной области: разработки ПО для роботов. Для выполнения тестирования, следует определиться с тем, какие задачи, выполняемые фреймворком, являются значимыми для производительности системы в целом.

Целью данной статьи является получения списка проблем и задач, решаемых при разработке фреймворков, которые являются значительными для производительности робототехнической системы в целом.