Для обсуждения конкретных решений требуется из всего множества существующих фреймворков выбрать наиболее подходящие для данного исследования.

\subsection{Критерии сравнения аналогов} 

\subsubsection{Наличие открытого исходного кода} для лучшего понимания работы фреймворка, а так же лучшего понимания результатов тестирования и возможного улучшения тестовых задач желательно иметь возможность прочитать исходный код реализации функционала, влияющего на результат тестирования. Это может быть, например, реализация коммуникации между приложениями, которые были написаны при помощи исследуемого фреймворка.

\subsubsection{Наличие документации} написание обоснованных и корректных тестовых задач под конкретный фреймворк очень затруднительно, если отсутствуют инструкций по его использованию. Без наличия документации разработка приложений становится слишком сложной и корректность их выполнения не гарантируется.

\subsubsection{Текущий статус разработки} проекты, которые больше не поддерживаются разработчиками вполне могут рассматриваться для исследования, но в них скорее-всего используются устаревшие подходы, что приведет к низким показателям производительности.

\subsubsection{Архитектура фреймворка} разработка робототехнических систем налагает определенные ограничения и требования. Отдельным и самым важным критерием является надежность и устойчивость к ошибкам \cite{blasco2012multiagent}.  Поскольку любой программный продукт, даже хорошо протестированный, содержит какие-либо ошибки, включая те, которые приводят к неожиданному завершению. В случае завершения работы всего программного комплекса из-за ошибки отдельного модуля робот теряет работоспособность. Это особенно важно в таком приложении робототехнических фреймворков, как программирования команд или роя автономных роботов: потеря работоспособности ключевых узлов команды приводит к остановке выполнения поставленной ими задачи. Примером может являться исследование множеством роботов местности или выполняющая задачи команда из разнородных, выполняющих различные роли автономных роботов.

Таким образом, требуется как минимум распределенная архитектура. Чем ближе архитектура будет к P2P (peer-to-peer, одноранговая сеть) и многоагентным системам, тем устойчивее будет все робототехническое ПО. Для данного исследования будут рассматриваться гибридные P2P фреймворки и близкие к чистым P2P.

\subsubsection{Наличие инструментов для мониторинга и конфигурации системы} для анализа промежуточного ПО и обслуживаемых им приложений требуются такие инструменты, как мониторы используемых ресурсов и системы журналирования. Кроме того, развертывание большого распределенного ПО для проведения тестов является трудоемкой задачей и крайне желательны инструменты конфигурации и развертывания системы на целевой ОС и аппаратном обеспечении.

\subsubsection{Поддержка различных языков программирования} несмотря на то, что в этой работе наибольший интерес представляет именно промежуточный уровень системы, наличие альтернатив между различными языками программирования прикладного слоя робототехнической системы является преимуществом, поскольку позволяет в зависимости от задачи выбрать между, например, низкоуровневым программированием с возможным преимуществом в производительности и высокоуровневыми языками с наличием удобных для разработки интерфейсов и прикладных библиотек.

В данной работе не столь важны критерии:
\begin{itemize}
	\item Поддержки ограничений системы реального времени, поскольку это относится не к скорости работы, а к предсказуемости системы. Наличие данного пункта является преимуществом в целом, но относительно производительности оказывается не существенным.
	\item Поддержка конкретных операционных систем, поскольку рассматривается вопрос производительности слоя абстракции между, собственно, ОС и прикладным ПО.
	\item Наличие инструментов симуляции, инструментов с графическим пользовательским интерфейсом и набора прикладных библиотек с реализациями наиболее распространенных алгоритмов, поскольку это относится к функциональному уровню системы, ближе к прикладному. Вопросы производительности отдельных инструментов, которые могут требоваться при разработке, отладке и администрированию робототехнических систем не относятся к проблематике потребления ресурсов на поддержание каркаса, основы системы - фреймворка.
\end{itemize}


\subsection{Представленные для рассмотрения фреймворки}

\subsubsection{ROS} один из наиболее распространенных фреймворков, имеется и обширная документация, и открытый исходный код, разработка продолжается, широко используется. Имеет гибридную архитектуру, средства мониторинга и автоматизации развертывания системы. Для разработки по-умолчанию есть возможность использовать C++ и Python 2.7. \cite{ros-man}

\subsubsection{MIRA} этот проект активно разрабатывается, имеется открытый исходный код и обширная документация. Имеет децентрализованную архитектуру \cite{einhorn2012mira}, реализован на C++. Имеются возможности для ведения журналирования приложений, мониторы состояния коммуникаций и ресурсов системы. Для разработки предлагается использовать C++, Python и JavaScript. \cite{mira-man}

\subsubsection{MOOS} развивающийся проект, имеется документация и исходный код. На данный момент разрабатывается бета-версия MOOS 10. Проблема обеих версий: клиент-серверная архитектура, которая является спорным решением для разработки робототехнических систем из-за проблем устойчивости ПО к ошибкам. По этой причине в данной работе этот фреймворк рассматриваться не будет. \cite{moos-man}

\subsubsection{OROCOS/Rock} очень распространенный фреймворк, один из немногих поддерживающих ограничения реального времени. Документация обширная, разработка в основном ведется над набором инструментов Rock. Используется гибридная децентрализованная архитектура. Разработка ведется на C++, для разработки прикладных программ предоставляются такие языки, как C++, Python, Simulink \cite{blasco2012multiagent}. Имеются инструменты для развертывания и мониторинга системы. \cite{orocos-man,rock-man}

\subsubsection{ASEBA} для программирования используется концепция языков программирования пятого поколения: GUI, блоки, коннекторы. Разработка является скорее обучающим продуктом с коммерческой составляющей в виде конкретной модели робота thymio. Рассматриваться в данной работе не будет. \cite{aseba-site}

\subsubsection{SmartSoft} разрабатывающийся проект с обширной документацией. Для реализации компонентов используется C++. Практически не используется децентрализация, основной шаблон взаимодействия клиент-сервер. Кроме того, используется многопоточный подход, а не многопроцессный \cite{smartsoft-man}, что ставит под вопрос общую устойчивость всей системы. В данной работе рассматриваться не будет.

\subsubsection{YARP} активно разрабатывающийся фреймворк с открытым исходным кодом. Имеется обширная и подробная документация. Является одним из немногих практически полностью децентрализованных фреймворков, кроме того, поддерживает ограничения систем реального времени. Разрабатывается в основном на C++ и поддерживает такие языки как C++, Python, Java, Octave. Инструменты мониторинга и развертывания приложений имеются. \cite{yarp-man}

\subsubsection{OpenRTM-aist} распространенный фреймворк с открытым исходным кодом, является реализацией стандарта RT-middleware. Основная проблема: часть документации, при скромном содержании, на японском языке. Архитектура гибридная децентрализованная, имеются инструменты мониторинга, журналирования и развертывания, языки разработки: C++, Java, Python. \cite{openrtmaist-man}

\subsubsection{URBI} данный фреймворк имеет много недостатков: приостановленная разработка, о чем свидетельствует последнее изменение от 2014 года \cite{urbi-repo} и не работающий сайт самого проекта \cite{urbi-site}, централизованная архитектура. Рассматриваться в данной работе не будет. 

В таблице \ref{tab:frameworks} отображено соответствие найденных фреймворков предложенным выше критериям.

\begin{table*}[h!]
	\scriptsize
	\centering
	\caption{Соответствие найденных робототехнических фреймворков выделенным критериям}
	\label{tab:frameworks}
	\def\arraystretch{1.5}
	\begin{tabular}{lp{2cm}p{2cm}p{2cm}lp{2.7cm}p{2cm}}
		\toprule
		\textbf{НАЗВАНИЕ}     & \textbf{ОТКРЫТЫЙ КОД} & \textbf{НАЛИЧИЕ ПОНЯТНОЙ ДОКУМЕНТАЦИИ} & \textbf{ПОСЛЕДНИЕ ИЗМЕНЕНИЯ} & \textbf{АРХИТЕКТУРА}        & \textbf{ИНСТРУМЕНТЫ МОНИТОРИНГА} & \textbf{ПОДДЕРЖКА ЯП} \\ 
		\midrule 
		ROS          & Да           & Да                            & Недавно             & Гибридная          & Да                      & C++, Python               \\ 
		MIRA         & Да           & Да                            & Недавно             & Децентрализованная & Да                      & C++, Python, JavaScript   \\ 
		MOOS         & Да           & Да                            & Недавно             & Централизованная   & Да                      & C++, Java                 \\ 
		OROCOS/Rock  & Да           & Да                            & 2016                & Гибридная          & Да                      & C++, Python, Simulink     \\ 
		ASEBA        & Да           & Нет                           & Недавно             & Распределенная     & Да                      & Собственный язык          \\ 
		SmartSoft    & Да           & Да                            & Недавно             & Распределенная     & Да                      & C++                       \\ 
		YARP         & Да           & Да                            & Недавно             & Децентрализованная & Да                      & C++, Python, Java, Octave \\ 
		OpenRTM-aist & Да           & Нет                           & 2016                & Гибридная          & Да                      & C++, Java, Python         \\ 
		URBI         & Да           & Нет                           & 2016                & Централизованная   & Да                      & C++, Java, urbiscript     \\ 
		\bottomrule
	\end{tabular}
\end{table*}

Таким образом, в исследовании будут учавствовать следующие фреймворки:
\begin{itemize}
	\item ROS
	\item MIRA
	\item OROCOS/Rock
	\item YARP
	\item OpenRTM-aist
\end{itemize}


