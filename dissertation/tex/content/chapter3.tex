\chapter{Результаты тестирования}
\section{Характеристики тестируемого окружения}
	Тесты проводились на следующей аппаратной конфигурации:
	\begin{itemize}[noitemsep]
		\item 8 процессоров Intel Xeon E5-2580 2.4 ГГц;
		\item 64 Гб оперативной памяти.
	\end{itemize}
	Операционная система: Ubuntu 16.04. Для тестирования был установлен Docker версии 18.04.0-ce.
\section{ROS}
\label{title:chapter3:ros}
\subsection{Зависимость задержки от определенных факторов}
Разбирая вопрос факторов, влияющих на производительность прикладного ПО для системы ROS, предполагалась важность следующих факторов:
\begin{itemize}[noitemsep]
	\item размер буфера;
	\item количество подписчиков.
\end{itemize}
В ходе тестирования были получены данные, которые позволяют подтвердить или опровергнуть гипотезы о значимости данных факторов для производительности ROS приложений.

\subsubsection{Зависимость от буфера}
Для проверки гипотезы о том, что задержка передачи данных как-либо зависит от размера буфера топика, рассмотрим графики на рисунке \ref{img:ros_buf_l_k}, в которых отражены результаты тестирования задержки передачи данных в системе \enquote{издатель-подписчик} при единственном подписчике, но при разных объемах данных в зависимости  от размера буфера топика.
\img{img/ros/ros_buf_l_k.png}{График зависимостей задержек передачи данных при разных объемах данных в пределах мегабайта от размера буфера топика.}{img:ros_buf_l_k}{\textwidth}
Зависимость примерно линейная и задержка практически никак не зависит от размера буфера топика.

\subsubsection{Зависимость от количества подписчиков}
Для проверки гипотезы о том, что задержка передачи данных как-либо зависит от количества узлов-подписчиков на топик, рассмотрим графики на рисунках \ref{img:ros_subs_l_m} и \ref{img:ros_subs_l_k}, в которых отражены результаты тестирования задержки передачи данных в системе \enquote{издатель-подписчик} при буфере равном 1000, но при разных объемах данных в зависимости  от количества подписчиков.

\img{img/ros/ros_subs_l_k.png}{График зависимостей задержек передачи данных при разных объемах данных в пределах мегабайта от количества подписчиков.}{img:ros_subs_l_k}{\textwidth}
\img{img/ros/ros_subs_l_m.png}{График зависимостей задержек передачи данных при разных объемах данных больше мегабайта от количества подписчиков.}{img:ros_subs_l_m}{\textwidth}

Из графиков видно, что большого влияния количество подписчиков не оказывает на быстродействие системы. Однако, следует заметить высокое, пропорциональное количеству подписчиков, использование памяти. Первоначально тестирование планировалось для 32 подписчиков, но 64 Гб доступной оперативной памяти не хватало даже на 16 подписчиков при объеме сообщений 64 Мб. 


\subsection{Показатели производительности при различных объемах данных}
\subsubsection{\enquote{Издатель-подписчик}}
Разберем графики с зависимостью задержки (секунды) и пропускной способности (мегабайты в секунду) от различных объемов данных, передаваемых в системе. Для рассмотрения берется система из одного издателя, одного подписчика и буфера размера 1000.

\img{img/ros/ros_pubsub_l.png}{График зависимостей задержек передачи данных при разных объемах данных больше мегабайта от количества подписчиков.}{img:ros_pubsub_l}{\textwidth}
\img{img/ros/ros_pubsub_bw.png}{График зависимостей пропускной способности относительно публикующего и относительно принимающего от разного объемах данных.}{img:ros_pubsub_bw}{\textwidth}

На рисунке \ref{img:ros_pubsub_l} видно, что при сообщениях от 16 Мб задержка составляет больше секунды. Это для робототехнической системы является неприемлимым с точки зрения предметной области. Следовательно, ROS не рекомендуется использовать для передачи больших объемов данных, либо учитывать большую задержку при обмене информацией.

На рисунке \ref{img:ros_pubsub_bw} видно, что при сообщениях больше 16 Кб реальная пропускная способность падает из-за роста задержки передачи сообщений. Кроме того, пропускная способность относительно узла-издателя так же падает при передаче сообщений больше мегабайта.

\subsubsection{\enquote{Клиент-сервис}}
\img{img/ros/ros_rpc_l_k.png}{График зависимости времени запроса и ответа при объеме данных до 64 Кб.}{img:ros_rpc_l_k}{\textwidth}
\img{img/ros/ros_rpc_l_m.png}{График зависимости времени запроса и ответа при объеме данных от 256 Кб.}{img:ros_rpc_l_m}{\textwidth}
\img{img/ros/ros_rpc_bps.png}{График зависимости пропускной способности от объема передаваемых даных}{img:ros_rpc_bps}{\textwidth}
\img{img/ros/ros_rpc_pubsub.png}{График зависимостей задержки передачи данных от объема передаваемых даных для разных типов коммуникации в ROS}{img:ros_rpc_pubsub}{\textwidth}
Рассмотрим графики с зависимостью задержки в миллисекундах и пропускной способности в мегабайтах в секунду от различных объемов данных, передаваемых в системе. При рассмотрении клиент-сервисного подхода уникальны следующие детали:
\begin{itemize}[noitemsep]
	\item данные передаются в обе стороны: на запрос и ответ - следовательно объем передаваемых учитывается два раза в полосе пропускания;
	\item имеется возможность измерить время передачи данных в обе стороны.
\end{itemize}

На графиках \ref{img:ros_rpc_l_k} и \ref{img:ros_rpc_l_m} видно, что до 256 Кб задержка отправки ответа незначительна, но при передаче данных от 1 Мб роль задержки ответа резко возрастает.

На графике \ref{img:ros_rpc_bps} видно, что при передаче данных посредством вызова удаленных процедур есть предел полосы пропускания около 650 Мб/с.

Кроме того, интересен график на рисунке \ref{img:ros_rpc_pubsub}, из которого следует, что передача данных при помощи вызова удаленных процедур, при том, что данных фактически при \enquote{сервис-клиент} подходе передавалось в 2 раза больше, задержка у подхода \enquote{сервис-клиент} примерно в 4 раза меньше и приемлима для систем автономных роботов.






\section{YARP}
\label{title:chapter3:yarp}
\subsection{Выявленные ограничения}
Несмотря на то, что изначально предполагалось использование объема данных вплоть до 64 Мб, в ходе тестирования было невозможно отправить данные от 1 Мб. Тестирование либо прерывалось с сигналом \textit{Segment Fault}, либо длилось слишком долго. Тестирование прерывалось примерно спустя 15 минут, если управление не переходило к следующему тесту. Если предположить, что тест из 10 итераций занимает 900 секунд и более, то передача данных в автономной робототехнической системе с полученной задержкой неприемлима.

Кроме того, для гарантии целостности передачи информации при использовании RPC-порта не поддерживается протокол UDP.

\subsection{Сравнение типов портов}
Для сравнения производительности различных типов портов будет использоваться протокол TCP. График на рисунке \ref{img:yarp_ports_l} показывает требуемые зависимости от объема передаваемых данных.
\img{img/yarp/yarp_ports_l.png}{График зависимостей задержек передачи данных для разных типов портов от объема передаваемых данных}{img:yarp_ports_l}{\textwidth}
\img{img/yarp/yarp_ports_bw.png}{График зависимостей пропускной способности для разных типов портов от объема передаваемых данных}{img:yarp_ports_bw}{\textwidth}

Как видно на графике, производительность порта с буфером и без практически не различается с минимальным отставанием порта без буфера сообщений.

Если рассматривать пропускную способность всех типов портов, то график на рисунке \ref{img:yarp_ports_bw} показывает, что все три типа портов имеют низкую пропускную способность, колеблющуюся между 115 и 120 килобайтами в секунду.

\subsection{Сравнение протоколов коммуникации}

Сравним производительность буферизованного порта и RPC-порта с различными доступными для них протоколами.

\img{img/yarp/yarp_protocol_buf_l.png}{График зависимостей задержек передачи данных для буферизованного порта с разными протоколами передачи данных от объема передаваемых данных}{img:yarp_protocol_buf_l}{\textwidth}
\img{img/yarp/yarp_protocol_buf_bw.png}{График зависимостей пропускной способности для буферизованного порта с разными протоколами передачи данных от объема передаваемых данных}{img:yarp_protocol_buf_bw}{\textwidth}
\img{img/yarp/yarp_protocol_rpc_l.png}{График зависимостей задержек передачи данных для RPC-порта с разными протоколами передачи данных от объема передаваемых данных}{img:yarp_protocol_rpc_l}{\textwidth}
\img{img/yarp/yarp_protocol_rpc_bw.png}{График зависимостей пропускной способности для RPC-порта с разными протоколами передачи данных от объема передаваемых данных}{img:yarp_protocol_rpc_bw}{\textwidth}

На графиках \ref{img:yarp_protocol_buf_l} и \ref{img:yarp_protocol_buf_bw} изображены показатели производительности различных протоколов для буферизованного порта. Учитывая, что протокол UDP не гарантирует достижение сообщения получателем и реальная пропускная способность у данного протокола ниже из-за потерь дэйтаграмм, то наиболее стабильным с точки зрения пропускной способности и эффективным по времени передачи данных является использование разделяемой памяти, за что в YARP отвечает фреймворк ACE.

На графиках \ref{img:yarp_protocol_rpc_l} и \ref{img:yarp_protocol_rpc_bw} изображены показатели производительности различных протоколов для RPC-порта. Протоколы TCP и FastTCP показывают одинаково низкие результаты производительности и, как и в случае с буферизованным портом, использование разделяемой памяти предоставляет наибольшую производительность коммуникации между узлами робототехнической системы.






\section{MIRA}
\label{title:chapter3:mira}
\subsection{Различие производительности различных типов модулей}
MIRA выделяется подходом к модулям - узлам распределенной системы. В MIRA модулями являются не исполняемые модули, а объекты классов, полученные при помощи реализации шаблона \enquote{Фабрика объектов} внутри самого фреймворка. В общем случае, каждый модуль - это объект класса с определенным интерфейсом. Все модули компилируются как разделяемые библиотеки. Это позволяет реализовать наиболее быстрое межпроцессное взаимодействие: разделяемая память внутри процесса. 

Кроме того, имеется возможность запускать модули в разных процессах, разработчики реализовывают взаимодействие в данном случае при помощи протокола TCP. Тем не менее, гарантия доставки сообщений отсутствует, т.к. сообщение может быть на момент прибытия в нужный узел \enquote{неактуальным} и будет пропущено. В случае, если модули находятся в разных процессах этот факт очень заметен: в 7 из 90 случаев сообщения в ходе тестирования не были зафиксированы системой тестирования. Этим можно объяснить колебания задержки, измеряемой на получателе. Кроме того, получатель не может формально отличить некорректные сообщения первых итераций тестирования, таким образом результаты, измеряемые на узле-получателе так же имеют дополнительную ошибку в виде первых нескольких измерений для каждого теста.

\subsubsection{Задержка передачи данных}
\img{img/mira/mira_res_i_o_ml.png}{График зависимостей задержек передачи данных внутри одного процесса и между двумя процессами в зависимости от объема передаваемых данных на стороне принимающего.}{img:mira_res_i_o_ml}{\textwidth}

На рисунке \ref{img:mira_res_i_o_ml} показаны графики задержки передачи сообщений. На них четко видно, что реальная задержка передачи сообщений между модулями внутри одного процесса на несколько порядков (порядок 1000 наносекунд внутри одного процесса и 10 миллисекунд в разных) ниже, чем в различных процессах. 

\img{img/mira/mira_res_i_o_sl.png}{График зависимостей задержек передачи данных внутри одного процесса и между двумя процессами в зависимости от объема передаваемых данных на стороне посылающего.}{img:mira_res_i_o_sl}{\textwidth}

При этом, на рисунке  \ref{img:mira_res_i_o_sl} отображен график, из которого следует, что время непосредственно публикации данных на модуле-передатчике внутри одного процесса примерно в $1.5$ раза дольше, чем в разных процессах.

\subsubsection{Пропускная способность}

Пропускную способность стоит рассматривать с учетом времени, которое занимает передача сообщения от передатчика к получателю. Если рассматривать время только на передатчике, то результат будет не соответствовать реальности. 

\img{img/mira/mira_res_i_o_bps.png}{График зависимостей пропускной способности передачи данных внутри одного процесса и между двумя процессами в зависимости от объема передаваемых данных.}{img:mira_res_i_o_bps}{\textwidth}

Как видно из графика на рисунке \ref{img:mira_res_i_o_bps}, пропускная способность передачи данных внутри процесса составляет гигабайты в секунду. Фактически, в случае внутрипроцессного взаимодействия, пропускная способность ограничена лишь доступом к оперативной памяти. В случае же межпроцессной передачи данных, пропускная способность достаточно низкая, около десятков мегабайт в секунду.

\subsection{Производительность различных подходов коммуникации}

\img{img/mira/mira_res_rpc_pubsub.png}{График зависимостей задержки передачи данных для различных подходов коммуникации в зависимости от объема передаваемых данных.}{img:mira_res_rpc_pubsub}{\textwidth}
\img{img/mira/mira_res_rpc_bps.png}{График зависимости пропускной способности передачи данных при удаленном вызове процедуры от объема передаваемых данных.}{img:mira_res_rpc_bps}{\textwidth}

В MIRA, кроме шаблона \enquote{издатель-подписчик}, доступно взаимодействие по шаблону \enquote{сервис-клиент}. Сравнение будет вестись для модулей в разных процессах, поскольку чаще всего стоит задача удаленного вызова процедур у другого процесса.

График на рисунке \ref{img:mira_res_rpc_pubsub} показывает, что при удаленном вызове процедур с увеличением объема данных резко начинает возрастать время на передачу данных. Стоит отметить, что в тестах объем данных передавался в обе стороны: от клиента к сервису и обратно. Тем не менее, это не объясняет стремительный рост времени передачи данных при больших объемах данных.

График на рисунке \ref{img:mira_res_rpc_bps} показывает, что при удаленном вызове процедур виден порог пропускной способности: примерно 22 мегабайта в секунду.
%\section{\todo{OROCOS}}
\section{Сравнение производительности фреймворков}
\label{title:chapter3:compharison}
Сравниваться между собой будут следующие конфигурации фреймворков:
\begin{itemize}
	\item ROS \enquote{издатель-подписчик} с одним подписчиком и буфером на 1000 сообщений;
	\item MIRA каналы в разных процессах;
	\item буферизованные порты YARP с протоколом TCP.
\end{itemize}

Данные конфигурации выбраны поскольку для различных процессов используется так или иначе протокол TCP. Кроме того, общей является возможность вызова удаленных процедур.

\img{img/comp/comp_tcp_l.png}{График зависимостей задержек передачи данных от одного узла другому для различных фреймворков от объема передаваемых данных}{img:comp_tcp_l}{\textwidth}
\img{img/comp/comp_rpc_l.png}{График зависимостей задержек передачи данных при удаленном вызове процедур для различных фреймворков от объема передаваемых данных}{img:comp_rpc_l}{\textwidth}

Согласно графикам \ref{img:comp_tcp_l} и \ref{img:comp_rpc_l} MIRA даже при использовании различных процессов имеет наименьшую задержку при передачи сообщений между узлами системы. YARP даже с учетом неполных данных из-за невозможности передать данные больше 256 Кб, на обозреваемых результатах показывает худшие результаты. ROS показывает средние результаты, задержка передачи сообщений между двумя узлами при больших объемах данных резко возрастает. Кроме того стоит заметить, что ROS затрачивает гораздо большие по-сравнению с MIRA ресурсы оперативной памяти.

\section{Выводы}
Таким образом, результаты тестирования выявили следующие факты:
\begin{itemize}
	\item на производительность ROS не влияет ни количество подписчиков, ни размер буфера сообщений (но это сильно сказывается на объеме потребляемой оперативной памяти);
	\item производительность ROS резко падает при передачи больших объемов данных;
	\item YARP имеет наихудшие показатели производительности;
	\item при разработке прикладного ПО для автономных роботов на платформе YARP рекомендуется использовать разделяемую память на основе фреймворка ACE;
	\item MIRA имеет наилучшие показатели производительности при наименьшем потреблении системных ресурсов.
\end{itemize}