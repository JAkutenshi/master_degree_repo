\chapter{План тестирования}
\label{title:chapter2}

\section{Предмет тестирования}
\label{title:chapter2:performance_subject}
	\subsection{Факторы влияния на производительность \marm{}}
	\label{title:chapter2:performance_factors}
	Ниже приведены важные факторы для производительности \marm{} \cite{blasco2012multiagent}.

\begin{description}[noitemsep]
	
	\item [Тип архитектуры.] Большая часть фреймворков использует гибридную архитектуру, поскольку такой подход позволяет избежать ряд проблем, а именно:
	\begin{itemize}[noitemsep]
		\item Не требуется инкапсулировать в каждый узел системы общие сервисы, как, например, службу поиска других узлов: эту задачу выполняет специально выделенный узел. Так же поступают и с агенствами - контейнерами, создающие, регистрирующие и хранящие узлы-агенты.
		\item Уменьшение нагрузки на сеть коммуникаций, поскольку вся нагрузка по общим для всех узлов запросам смещается в сторону обработки на выделенных узлах. Это приводит к появлению критических узлов в системе, в случае отказа которых весь робот может оказаться неспособным продолжать выполнение поставленных задач. Таким образом, в таком подходе требуется отдельно рассматривать производительность таких критических узлов системы, как служба поиска узлов, служба именования и, возможно, другие выделенные сервисы.
	\end{itemize}
	В случае, если используется максимально децентрализованная архитектура, то возникают следующие проблемы:
	\begin{itemize}[noitemsep]
		\item Увеличивается сложность узлов, а так же нагрузка на обработку данных между узлами. Тем не менее то, насколько эффективно используются возможности фреймворка по взаимодействию между узлами, зависит от разработчика на прикладном уровне.
		\item Увеличивается нагрузка на систему коммуникации фреймворка. Под системой коммуникации понимается набор технологий и протоколов, обеспечивающих обмен данными между узлами системы. Часто для обеспечения коммуникации используются отдельные узлы, создаваемые самим фреймворком, будь то \todo{брокеры объектных запросов (ORB)} при использовании CORBA или реализации шаблона взаимодействия \enquote{издатель-подписчик} при помощи топиков \todo{(от англ. topic - тема)}, которые имеют свою внутреннюю реализацию.
	\end{itemize}

	\item [Тип взаимодействия между узлами.] При использовании многоагентной архитектуры используется несколько разных способов взаимодействия между узлами-агентами:
	\begin{itemize}[noitemsep]
		\item порты, реализующие независимое чтение из входных портов, запись во входные и взаимодействие \enquote{один ко многим};
		\item топики, реализующие шаблон \enquote{издатель-подписчик} и взаимодействие \enquote{многие ко многим};
		\item события, реализующие шаблон \enquote{наблюдатель} и асинхронное взаимодействие \enquote{один ко многим};
		\item сервисы, предоставляющие реактивный способ поведения узлам: возможность отправлять запрос от узла-клиента на узел-сервис, реализуя взаимодействие выполнения удаленных процедур;
		\item свойства, предоставляющие возможность менять состояние узлов при помощи селекторов параметров агента (get/set), реализация может отличаться в зависимости от архитектуры: гибридная архитектура представляет выделенный узел-сервис - службу свойств, децентрализованная инкапсулирует свойства узлов непосредственно в агентах системы.
	\end{itemize}
	
	Большинство фреймворков дает выбор: какой тип взаимодействия между узлами использовать, но, в случае с реализацией сервисов может возникнуть простой системы до тех пор, пока не работающий или перегруженный обработкой запросов узел, предоставлявший требуемый функционал, не будет восстановлен. Реализация отдельных типов взаимодействий влияет практически на все аспекты производительности: задержку, вычислительные ресурсы, использование памяти, энергопотребление. Для каждого из фреймворков следует тестировать каждый из способов взаимодействия по всем показателям производительности.

	\item [Протоколы коммуникации.] Для доставки сообщений обычно используются различные подходы и протоколы. Обычно разработчики реализуют решения на основе стека TCP/IP с некоторыми оптимизациями, например реализация узлов как отдельных потоков внутри процесса-агенства. Данный подход позволяет использовать общую память процесса и ускорить взаимодействие между узлами внутри процесса-агенства, которые могут располагаться в пределах одного вычислительного устройства, что влечет за собой возможность потерять доступ ко всем узлам агенства в случае неисправностей из-за ошибок, допущенных при реализации агентов. Данная проблема решается репликацией агенств на вычислительной системе. Кроме того для доставки сообщений между узлами используются как более высокоуровневые технологии (CORBA, ICE, ACE), так и приближенные к аппаратной части (EtherCAT, I2C, CANBus). Выбор механизма доставки данных между узлами влияет на задержку получения информации, на пропускную способность между узлами, целостность данных, а так же потребление ресурсов вычислительной системы: чем более высокоуровневая технология, тем больше потребление ресурсов системы. Дополнительный функционал на уровне коммуникации, например QoS или шифрование так же влияет на производительность. При наличии такого функционала его следует тестировать отдельно.

	\item [Формат сообщений.] Сообщения в зависимости от их структуры влияют на производительность. Бинарные типы сообщений имеют меньший объем, а следовательно на их передачу требуется меньше энергии и времени. Структурированные типы, вроде XML и JSON, хорошо поддаются анализу для разработчика, но могут занимать больше времени на обработку своих данных, а так же требуют больше времени на передачу данных между узлами.

	\item [Способ взаимодействия с аппаратурой.] Существует два основных способа предоставить интерфейс от аппаратной части системы к прикладному ПО:
	\begin{itemize}[noitemsep]
		\item инкапсулировать взаимодействие с аппаратурой в отдельных узлах, отображающих устройства;
		\item использовать промежуточный слой - сервер, который отвечает на запросы узлов и устройств;
		\item динамически связывающиеся библиотеки.
	\end{itemize}
	
	От способа обращения будет зависеть отзывчивость системы на внешнее взаимодействие. Кроме того, различные реализации может использовать различное количество ресурсов. Кроме того, в случае использования дополнительного слоя абстракции с клиент-серверным взаимодействием между аппаратным слоем системы и фреймворком является уязвимым подходом с точки зрения устойчивости системы.

\end{description}

В таблице \ref{table:chapter2:performance_factors_solutions} показано сопоставление возникающих проблемы и их решений для отобранных для исследования фреймворков \cite{blasco2012multiagent,bruyninckx2001orocos,einhorn2012mira,metta2006yarp,mohamed2008survey,elkady2012comprehensive}.

\begin{table}[!h]
	\centering
	\small
	\def\arraystretch{1.3}
	\caption{Реализация критических для производительности задач для отобранных фреймворков}
	\label{table:chapter2:performance_factors_solutions}
	\begin{tabular}{|p{3.6cm}|p{2.7cm}|p{2.7cm}|p{2.7cm}|p{2.7cm}|}
		\hline
		& \textbf{ROS} & \textbf{MIRA} & \textbf{ORoCoS} & \textbf{YARP} \\ \hline
		\textbf{Централизованные сервисы} & Сервисы поиска, именования, сервис параметров & Нет & Сервисы поиска, именования & Сервис имен \\ \hline
		\textbf{Взаимодействие между узлами} & Топики, параметры, сервисы & Топики, RPC & Порты, сервисы, события, параметры & Порты, топики \\ \hline
		\textbf{Протоколы коммуникации} & TCP, UDP, собственный протокол rosserial & IPC, TCP & CORBA, TCP, UDP, SSL,UNIX Sockets, MQueue, EtherCAT, CanOPEN & ACE, TCP, UDP, IPC \\ \hline
		\textbf{Формат сообщений} & Бинарный & Бинарный, XML, JSON & Сериализация на основе CORBA & Бинарный \\ \hline
		\textbf{Взаимодействие с аппаратурой} & Инкапсуляция в узлах & Инкапсуляция в узлах и RPC-API & Инкапсуляция в узлах и RPC-API & Инкапсуляция в узлах и динамически подключаемые библиотеки \\ \hline
	\end{tabular}
\end{table}

Таким образом, рассматриваемые области тестирования:
\begin{itemize}[noitemsep]
	\item централизованные сервисы, если они имеются во фреймворке;
	\item способы взаимодействия между узлами с целью установления задержки передачи сообщений между узлами;
	\item коммуникация между узлами многоагентной системы с целью установления пропускной способности, при использовании шифрования - задержку передачи сообщения;
	\item формат передачи данных между узлами, объема передаваемых данных, скорости извлечения данных в случае, если информация сжимается;
	\item \todo{интерфейсы между аппаратной частью и фреймворком с целью установления задержки реакции системы на окружение.}
\end{itemize}

	
	\subsection{Разбор методов коммуникации \marm{}}
	\label{title:chapter2:performance_approaches}
		\subsubsection{ROS}
		\label{title:chapter2:performance_ros_approaches}
		Система коммуникации в ROS состоит из двух типов: взаимодействие между узлами при помощи топиков и сервисов \cite{o2014gentle}.

\begin{itemize}[noitemsep]
	\item \textit{Взаимодйствие при помощи топиков} -- это широковещательный подход взаимодействия по шаблону \enquote{наблюдатель (издатель-подписчик)}. Узлы-издатели публикуют \enquote{тему} общения -- топик -- доступную для всех узлов. Все подписавшиеся на опубликованный топик узлы-читатели получают возможность читать отправленные на данный топик сообщения. Факторы, влияющие на производительность:

	\item \textit{Клиент-сервисное взаимодействие} -- это подход, согласно которому каждый узел может реализовывать сущность RPC сервиса, исполняющего какой-либо процесс по запросу узла-клиента, возможно, с аргументами, и, возвращающий, ответ на запрос, возможно, пустой. Факторы, влияющие на производительность, в сравнении с топиками, идентичные, но, с точки зрения метрик, отдельно требуется рассматривать задержку передачи как для запроса, так и для ответа с разным объемом сообщений для запроса и ответа.
\end{itemize}

Узлы взаимодействуют напрямую, Мастер-сервис только предоставляет информацию об именах существующих узлов, как DNS \todo{(Domain Name Service)}. Узлы-подписчики посылают запрос узлу, посылающему сообщения по определенному топику, и устанавливает с ним соединение по протоколу TCPROS \cite{ros-tcpros}, использующий в себе стандартные TCP/IP сокеты \cite{ros-concepts}.

Таким образом, факторы, влияющие на производительность коммуникации в ROS:
\begin{itemize}[noitemsep]
	\item тип взаимодействия:
	\begin{itemize}[noitemsep]
		\item через топики;
		\item через сервисы.
	\end{itemize}
	\item количество взаимодействующих узлов
	\begin{itemize}[noitemsep]
		\item подписчиков для топиков;
		\item клиентов для сервисов.
	\end{itemize}
	\item объем сообщений;
	\item размер буферов каналов коммуникации.
\end{itemize}
		
		\subsubsection{YARP}
		\label{title:chapter2:performance_yarp_approaches}
		В YARP основной способ передачи информации - система портов. Как и в ROS, это механизм, реализующий шаблон \enquote{наблюдатель}. Отличие от ROS в том, что реализован этот шаблон через подход портов - устройства потокового ввода и вывода. Кроме того, как и в ROS, имеется возможность использовать механизм RPC между узлами системы. Ниже разобраны оба механизма.

\begin{itemize}
	\item \textit{Порты} -- это именованные объекты, которые обеспечивают доставку сообщений множеству других портов, подписанных на данные порты по данному имени. Один узел может иметь множество портов внутри. 
	
	Порты в YARP реализованы как объекты потоков, в которые можно писать данные и с которых данные можно считывать. Этот подход позволяет разделить отправителя данных и принимающего данные, таким образом, соответствующим портам не требуется знать много информации друг о друге. Это с одной стороны повышает абстрактность, позволяя работать со множеством портом сразу и допускать, например, перезапуск или временную неработоспособность какого-либо узла, с другой влияет на производительность, так как полностью от зависимости передающего и получателя сообщения избавиться нельзя. В частности, порт знает, когда передача сообщений конкретному порту получателю закончена: только после этого посылающий может приступать к передаче следующему адресату. Объекты, которые пересылаются между портами, не копируются. Эту проблему можно решить разными способами \cite{yarp-ports}, например, хранением сообщений в очереди на отправку.
	
	Порты бывают двух типов: обычные и с буфером сообщений. Второй тип отличается от первого наличием очереди сообщений, в слотах которой под сообщения резервируются сообщение перед отправкой. В отличии от обычных портов, порты с буфером сообщений теперь отвечают за время жизни объектов, которые требуется пересылать между портами, предоставляя порту самому определять, например, очередность передачи сообщений. Например, поскольку, находящееся в очереди сообщение можно модифицировать, порт с буфером может определить сообщение до и после изменения данных и не станет пересылать старое сообщение, если не выставлен определенный флаг, указывающий на обязательность передачи всех сообщений. В отличии от ROS размер буфера порта не фиксированный и зависит от количества сообщений в очереди: длина очереди увеличивается если количество сообщений превышает некоторый порог.
	
	Обычные порты \inline{yarp::os::Port} и с буфером \inline{yarp::os::BufferedPort} сильно между собой отличаются и разумно сравнить их между собой для различных данных и различной нагрузке портов-получателей. Примером различия в производительности может служить возможность, если не выставлен специальный флаг, перейти к отправке других сообщений другим портам-адресатам пока все порты-получатели конкретного сообщения заняты.
	
	\item \textit{RPC} -- механизм синхронизированной передачи данных между портами. Формально, поскольку все порты имеют возможность двусторонней связи, не требуется никаких дополнительных абстракций для реализации подобного подхода в YARP. Тем не менее, разработчики выделили в отдельные классы \inline{yarp::os::RpcClient} и \inline{yarp::os::RpcServer} -- наследники базового класса всех портов \inline{yarp::os::Contactable} -- для наличия готового инструмента с обеспечением целостности и синхронности передачи данных. В отличии от ROS, в YARP вся коммуникация идет через порты, через единообразный интерфейс устройства потокового ввода и вывода. 
\end{itemize}

В YARP не требуется описывать данные в особых форматах. В самом простом случае, разработчик реализует обычный C++ класс и передает его шаблонным параметром при создании порта. Возможны трудности в случае, если передаются данные между различными машинными архитектурами, если данным требуется не очевидная сериализация. В целом, большинство проблем решаемы при помощи предоставляемых YARP макросами, интерфейсами и классами-обертками.

Для передачи данных может использоваться множество протоколов, а так же \enquote{транспортов} (carriers) - классов-обработчиков объектов-сообщений для передачи между портами. Есть множество транспортов, реализованных в YARP и доступных для работы со распространенными транспортными протоколами:
\begin{itemize}[noitemsep]
	\item TCP;
	\item UDP;
	\item multicast -- широковещательный протокол, наиболее эффективная реализация для передачи сообщений множеству YARP-портов;
	\item shared memory -- передача данных в пределах одной машины используя разделяемую область оперативной памяти.
\end{itemize}
Преимущество YARP состоит в том, что способ транспортировки сообщения в соединении изменяем в любой момент времени как реализацией внутри узлов системы, так и внешним конфигурированием узлов, что позволяет гибко управлять коммуникацией внутри робототехнической системы.

Отдельно стоит указать возможность управлять приоритетом соединений, т.е. наличие \todo{QoS (Quality of Service)}. Для определенного соединения можно назначить один из четырех возможных приоритетов передачи данных.

Узлы взаимодействуют напрямую, взаимодействия с сервисом имен можно избежать, задавая имена портам при инициализации самостоятельно. В таком случае контроль уникальности идентификаторов ложится на разработчика.

Таким образом, основным способом связи в YARP являются порты, на производительность передачи данных будут влиять следующие факторы:
\begin{itemize}[noitemsep]
	\item тип порта:
	\begin{itemize}[noitemsep]
		\item обычный;
		\item с буфером
		\item реализующий синхронизированную передачу данных;
	\end{itemize}
	\item тип транспорта (carriers):
	\begin{itemize}[noitemsep]
		\item tcp;
		\item udp;
		\item mcast -- широковещательный транспорт;
		\item shmem -- транспорт при помощи локально разделяемой памяти (используется ACE - Adaptive Communication Environment);
		\item local -- использование разделения памяти внутри одного процесса;
		\item fast\_tcp -- реализация tcp транспорта без подтверждения пакетов о доставке.
	\end{itemize}
	\item размер данных;
	\item количество портов-получателей;
	\item приоритет соединений, задаваемых при помощи QoS.
\end{itemize}

		\subsubsection{MIRA}
		\label{title:chapter2:performance_mira_approaches}
		В MIRA предоставляются стандартные два вида коммуникации: отправка сообщений между узлами по шаблону \enquote{наблюдатель} и удаленный вызов методов. В терминологии MIRA, узлами являются модули (units), объединяющиеся в домены. Модули компилируются и используются как разделяемые библиотеки \cite{mira-units}, что позволяет загружать нужные модули во время исполнения робототехнической системы на прикладном уровне. Основанный на разделяемом коде, высоком уровне абстракции и сериализации объектов классов при помощи сохранении информации о классе -- рефлексии, как аналогичный механизм, использующийся в Java для получения метаинформации о классах объектов во время исполнения программ -- подход позволяет уменьшить затраты ресурсов памяти в прикладных приложениях, а так же предоставляет разделение ответственности для разработчиков, позволяя максимально абстрактно реализовывать модули, подобно плагинам. 

В отличии от ROS и YARP, MIRA не позволяет как-либо управлять очередью сообщений. Известно, что эта очередь есть и имеется возможность сохранять сообщения в памяти на определенный промежуток времени, который может указать разработчик. Это позволяет получать доступ к \enquote{прошлому} - к сообщениям, которые пришли какое-то время назад. 

Для передачи сообщений между процессами используется протокол TCP, внутри одного процесса -- разделение памяти.

Объекты сообщений передаются между модулями в бинарном виде путем сериализации объектов, созданных при помощи шаблона \enquote{фабрика объектов}. Вся метаинформация указывается в переопределяемом методе \inline{template <typename Reflector> void mira::reflect(Reflector& r)} для любого объекта MIRA, таким образом, позволяя передавать любые объекты в среде коммуникации MIRA.

Управление потоками данных, приоритетами сообщений между модулями отсутствует.

Таким образом, основными факторами, возможно, влияющими на производительность системы коммуникации MIRA являются:
\begin{itemize}[noitemsep]
	\item способ передачи данных:
	\begin{itemize}[noitemsep]
		\item по шаблону \enquote{наблюдатель};
		\item при помощи вызова удаленных процедур.
	\end{itemize}
	\item локализация модуля:
	\begin{itemize}[noitemsep]
		\item в одном процессе (используется разделяемая память);
		\item в разных (используется протокол TCP).
	\end{itemize}
	\item объем пересылаемых данных;
	\item время хранения сообщений в буфере соединения.
\end{itemize}
		
		\subsubsection{OROCOS}
		\label{title:chapter2:performance_orocos_approaches}
		\toolchain{} является наиболее сложным \marm{} из всех представленных ранее, т.к. в нем сочетаются все вышеописанные идеи. Первый релиз \toolchain{} в 2006 году, раньше всех исследуемых \marm{}. 

\toolchain{} состоит из нескольких проектов, среди которых нам наиболее важен \rtt{} -- библиотеки для разработки компонентов, загружающихся во время исполнения приложения. Эта же идея используется в MIRA. Компоненты \rtt{} являются наследниками класса \inline{RTT::TaskContext} и переопределяют ряд методов (hooks), определяющих основу поведения RTT-компонента. Данные компоненты являются аналогом узлов ROS или YARP, модулей MIRA. В отличии от MIRA, \orocos{} имеет возможность написания самостоятельных приложений с отдельной точкой входа.

Методов коммуникации в \rtt{} все так же два: система именованных портов для передачи сообщений и вызов удаленных процедур. Разница состоит во множестве способов применять данные инструменты. По доступным возможностям \rtt{} превосходит YARP. Ниже, основываясь на документации разработчиков \cite{rtt-components}, приведено описание методов коммуникации не вдаваясь в подробности API \rtt{}, например, сигналов-событий (Signal event handler) и деятельностей (Activities), которые, расширяют возможности реакции компонентов на внешнюю среду и, как следствие, разные способы вступать в коммуникацию. Тем не менее, эти способы реакции не являются методами передачи сообщений.

\begin{itemize}[noitemsep]
	\item \textit{Именованные порты} -- аналогично портам в YARP, \rtt{} порты реализуют шаблон \enquote{наблюдатель}, где на порты-издатели подписываются порты-подписчики. Разница с YARP в том, что в \rtt{} нет деления на \enquote{обычные} и \enquote{буферизованные} порты, но есть разделение на исходящие \inline{RTT::OutputPort<typename T>} и принимающие порты \inline{RTT::InputPort<typename T>} с соответствующе разными интерфейсами. В то же время, оба типа портов в \rtt{} могут настраивать буферизацию, потокобезопасность и инициализацию соединения пользователем при создании самого соединения.
	
	\item \textit{Вызов удаленных методов} -- аналогично другим \marm{} этот механизм в \orocos{} позволяет вызывать удаленную процедуру, передавая в нее аргументы и получая результат. Вызов удаленных процедур может производиться двумя способами: при помощи непосредственно вызова процедур и при помощи сервисов.
	\begin{description}[noitemsep]
		\item [Вызов операции (operation calling)] отличие главным образом от других подходов в том, что в \orocos это именно процедура, которая может выполняться как в потоке компонента, предоставляющего эту процедуру, так и в потоке клиента, эту процедуру вызывающего. В случае выполнения процедуры в потоке клиента разработчику требуется позаботиться о потокобезопасности разделяемых данных в компоненте клиента. Кроме того, в \rtt{} имеется два подхода к вызову процедур и получению результата: ожидать возврата результата, заблокировав текущий поток выполнения, либо передать ожидание и обработку результата соответствующему объекту-обработчику событий. Последний подход так же требует от разработчика аккуратности при работе с разделяемыми данными, но при этом дает возможность уменьшить реальное (wall clock) время обработки результата.
		\item [Сервисы] отличаются от вызова операций по-сути возможностью именовать сервис, который предоставляет набор операций, которые можно у сервиса вызвать. Это предоставляет возможность компонентам искать другой компонент-сервис по имени во время исполнения приложения.
	\end{description}
\end{itemize}

В \rtt{} могут передаваться любые пользовательские данные. Стандартные типы языка C++ доступны по-умолчанию, но более сложные требуется описать при помощи библиотеки \inline{boost::serialization}.

Особенностью \orocos{} является возможность использования CORBA для передачи данных, а так же возможность использовать в качестве транспорта для внутрипроцессного взаимодействия \inline{MQueue}. Причем CORBA может использовать протокол OOB (Out-Of-Band) для передачи данных при помощи механизма MQueue. Это позволяет:
\begin{itemize}[noitemsep]
	\item следить за прерванными соединениями;
	\item быть уверенным, что принимающий поток создается строго после исходящего, а так же корректно закрывает исходящий поток, если не получилось создать принимающий.
\end{itemize}
В целом, совместное использование MQueue и CORBA позволяет добиться большей надежности соединений, но путем затрат ресурсов на инфраструктуру CORBA.

\orocos{} предоставляет создавать объект \enquote{политики соединения} (connection policies) для конфигурирования соединения между портами. В частности, именно в объекте политики соединения указывается транспорт для сообщений. Тем не менее, \orocos{} позиционируется как наилучшее решение для приложений с внутрипроцессным взаимодействием \cite{orocos-interprocess-case}. Для межпроцессного взаимодействия и для передачи данных по сети требуется использовать дополнительные протоколы и подходы, например, CORBA или, например, Qt TCP Sockets.

В отличии от YARP, \toolchain{} не предоставляет возможностей для контроля QoS. На практике, параметры соединений можно изменять, но на уровне конфигурирования транспорта. Политики соединений позволяют лишь указать тип транспорта и размер буфера.

Таким образом, для производительности \rtt{} важны следующие факторы:
\begin{itemize}[noitemsep]
	\item локализация компонентов: для связи компонентов в разных процессах потребуется использовать другой транспорт;
	\item используемый транспорт:
	\begin{itemize}[noitemsep]
		\item использование общей памяти - транспорт по-умолчанию;
		\item MQueue;
		\item CORBA;
		\item MQueue при помощи CORBA OOB.
	\end{itemize}
	\item размер данных;
	\item размер буферов соединений;
	\item подход к взаимодействию компонентов:
	\begin{itemize}[noitemsep]
		\item используя порты;
		\item используя вызов операций;
		\item используя сервисы.
	\end{itemize}
\end{itemize}

\section{Описание тестовых случаев}
\label{title:chapter2:performance_test_cases}
Исходя из факторов, описанных в разделе \ref{title:chapter2:performance_approaches}, можно составить набор сценариев для проведения тестирования производительности.

\subsection{ROS}
В столбцах таблицы \ref{table:chapter2:ros_test_cases} приведены различные значения по выделенным ранее факторам для тестирования. Таким образом, множество всех сценариев тестов производительности -- это все возможные подмножества из возможных значений факторов таблицы. \ref{table:chapter2:ros_test_cases}.
\begin{table}[!h]
\centering
\small
\def\arraystretch{1.3}
\caption{Факторизация параметров для тестирования производительности ROS}
\label{table:chapter2:ros_test_cases}
\resizebox{\textwidth}{!}{%
\begin{tabular}{|p{4cm}|p{3cm}|p{3cm}|p{2cm}|}
\hline
\textbf{Тип взаимодействия} & \textbf{Размер сообщения} & \textbf{Количество подписчиков или клиентов} & \textbf{Размер буфера}     \\ \hline
Издатель/ подписчик & 1 КБ             & 1                                   & 10                \\ \hline
Сервисы            & 10 КБ            & 2                                   & 100               \\ \hline
\multirow{8}{*}{}  & 100 КБ           & 4                                   & 1000              \\ \cline{2-4} 
                   & 1 МБ             & 8                                   & 2000              \\ \cline{2-4} 
                   & 10 МБ            & 16                                  & 5000              \\ \cline{2-4} 
                   & 50 МБ            & 32                                  & 10000             \\ \cline{2-4} 
                   & 100 МБ           & \multirow{4}{*}{}                   & \multirow{4}{*}{} \\ \cline{2-2}
                   & 250 МБ           &                                     &                   \\ \cline{2-2}
                   & 500 МБ           &                                     &                   \\ \cline{2-2}
                   & \todo{1 ГБ}             &                                     &                   \\ \hline
\end{tabular}%
}
\end{table}

\subsection{YARP}
В столбцах таблицы \ref{table:chapter2:yarp_test_cases} приведены различные значения по выделенным ранее факторам для тестирования. Таким образом, множество всех тестов, это все возможные подмножества из возможных значений факторов таблицы \ref{table:chapter2:yarp_test_cases} из столбцов 1-5, понимая трактовку последнего фактора -- QoS -- следующим образом: если QoS применяется, то организовать набор из постоянных 9 соединений без QoS и одного соединения, для которого применяются факторы в столбцах 1-4, а так же приоритеты QoS 5го столбца.
% Please add the following required packages to your document preamble:
% \usepackage{multirow}
% \usepackage{graphicx}
\begin{table}[]
	\centering
	\caption{Факторизация параметров для тестирования производительности YARP}
	\label{table:chapter2:yarp_test_cases}
%	\footnotesize
	\def\arraystretch{1.5}
	\newcommand{\sr}{\rule[-0.45cm]{0pt}{0.9cm}}
%	\resizebox{\textwidth}{!}{%
		\begin{tabular}{|p{2cm}|p{2.5cm}|p{2.4cm}|p{2.1cm}|p{2.7cm}|p{2.4cm}|}
		\hline
		\multirow{2}{2cm}{\textbf{Тип порта}} & \multirow{2}{2.5cm}{\textbf{Транспорт соединения}} & \multirow{2}{2.4cm}{\textbf{Размер сообщения}} & \multirow{2}{2.1cm}[-0.2cm]{\textbf{Число входящих портов}} & \multicolumn{2}{l|}{ \sr \textbf{QoS}} \\ \cline{5-6} 
		 & & & & \sr \textbf{Статус } & \sr \textbf{Приоритет} \\ \hline
		Обычный           & tcp   & 1 КБ    & 1  & Применяется       & LOW               \\ \hline
		С буфером         & udp   & 4 КБ    & 2  & Не применяется    & NORMAL            \\ \hline
		RPC               & mcast & 16 КБ   & 4  & \multirow{8}{*}{} & HIGH              \\ \cline{1-4} \cline{6-6} 
		\multirow{7}{*}{} & shmem & 64 КБ   & 8  &                   & CRITICAL          \\ \cline{2-4} \cline{6-6} 
		 & \todo{local}           & 256 КБ  & 16                &    & \multirow{6}{*}{} \\ \cline{2-4}
	     & fast\_tcp              & 1 МБ    & 32                &    &                   \\ \cline{2-4}
		 & \multirow{4}{*}{}      & 4 МБ    & \multirow{4}{*}{} &    &                   \\ \cline{3-3}
		 &                        & 16 МБ   &                   &    &                   \\ \cline{3-3}
		 &                        & 64 МБ   &                   &    &                   \\ \cline{3-3}
		 &                        & 256 МБ  &                   &    &                   \\ \cline{3-3}
		 &                        & \todo{1 ГБ}  &              &    &                   \\ \hline
		\end{tabular}%
%	}

\end{table}

\subsection{MIRA}
В столбцах таблицы \ref{table:chapter2:mira_test_cases} приведены различные значения по выделенным ранее факторам для тестирования. Таким образом, множество всех сценариев тестов производительности -- это все возможные подмножества из возможных значений факторов таблицы \ref{table:chapter2:mira_test_cases}.
\begin{table}[]
	\centering
	\caption{Факторизация параметров для тестирования производительности MIRA}
	\label{table:chapter2:mira_test_cases}
%	\footnotesize
	\def\arraystretch{1.3}
%	\resizebox{\textwidth}{!}{%
		\begin{tabular}{|p{4cm}|p{3cm}|p{3cm}|p{3cm}|}
			\hline
			\textbf{Подход к коммуникации} & \textbf{Реализация модуля} & \textbf{Размер сообщения} & \textbf{Время хранения сообщений в очереди} \\ \hline
			Шаблон наблюдатель & В одном процессе (разделяемая память) & 1 КБ   & 1 мс              \\ \hline
			RPC                & В разных процессах (TCP)              & 4 КБ   & 10 мс             \\ \hline
			\multirow{8}{*}{}  & \multirow{8}{*}{}                     & 16 КБ  & 100 мс            \\ \cline{3-4} 
			                   &                                       & 64 КБ  & 1 с               \\ \cline{3-4} 
			                   &                                       & 256 КБ & 10 с              \\ \cline{3-4} 
			                   &                                       & 1 МБ   & 100 с             \\ \cline{3-4} 
			                   &                                       & 4 МБ   & \multirow{4}{*}{} \\ \cline{3-3}
			                   &                                       & 16 МБ  &                   \\ \cline{3-3}
  			                   &                                       & 64 МБ  &                   \\ \cline{3-3}
			                   &                                       & 256 МБ &                   \\ \cline{3-3}
                               &                                       & \todo{1 ГБ} &  \\ \hline
		\end{tabular}%
%	}
	
\end{table}

\subsection{\orocos{}}
В данном случае имеется четкое разделение на взаимодействие компонентов внутри одного процесса и между процессами. В таком случае, образуется 4 возможных базовых сценариев тестирования:
\begin{itemize}[noitemsep]
	\item внутрипроцессное взаимодействие при помощи разделяемой памяти;
	\item межпроцессное взаимодействие:
	\begin{itemize}[noitemsep]
		\item MQueue;
		\item CORBA;
		\item CORBA + MQueue;
	\end{itemize}	
\end{itemize}
Таким образом, в таблице \ref{table:chapter2:orocos_test_cases} приведены различные значения по выделенным ранее факторам для тестирования, а множество всех сценариев тестов производительности -- это все возможные подмножества из возможных значений факторов таблицы, не считая первую половину столбца 1, который представлен для разделения расположения компонентов: в одном или в разных процессах.
\begin{table}[]
	\centering
	\caption{Факторизация параметров для тестирования производительности \toolchain}
	\label{table:chapter2:orocos_test_cases}
%	\footnotesize
	\def\arraystretch{1.3}
%	\resizebox{\textwidth}{!}{%
		\begin{tabular}{|p{3.2cm}|p{2.6cm}|p{3.6cm}|p{2.5cm}|p{1.6cm}|}
			\hline
			\multicolumn{2}{|l|}{\textbf{Протокол коммуникации}} & \multirow{2}{3.6cm}{\textbf{Тип взаимодействия}} & \multirow{2}{2.5cm}{\textbf{Размер сообщения}} & \multirow{2}{1.6cm}{\textbf{Размер буфера}} \\ \cline{1-2}
			\textbf{Расположение} & \textbf{Протокол} &  &  &  \\ \hline
			Внутри процесса & Разделяемая память & Порты & 1 КБ & 10 \\ \hline
			Между процессами & MQueue & Удаленный вызов операции & 4 КБ & 100 \\ \hline
			\multirow{2}{*}{} & CORBA & Сервис & 16 КБ & 1000 \\ \cline{2-5} 
			& CORBA+ MQueue & \multirow{7}{*}{} & 64 КБ & 2000 \\ \cline{1-2} \cline{4-5} 
			\multicolumn{2}{|l|}{\multirow{6}{*}{}} &  & 256 КБ & 5000 \\ \cline{4-5} 
			\multicolumn{2}{|l|}{} &  & 1 МБ & 10000 \\ \cline{4-5} 
			\multicolumn{2}{|l|}{} &  & 16 МБ & \multirow{4}{*}{} \\ \cline{4-4}
			\multicolumn{2}{|l|}{} &  & 64 МБ &  \\ \cline{4-4}
			\multicolumn{2}{|l|}{} &  & 256 МБ &  \\ \cline{4-4}
			\multicolumn{2}{|l|}{} &  & \todo{1 ГБ} &  \\ \hline
		\end{tabular}%
%	}
	
\end{table}
	
\section{Реализация}
	\subsection{Создание тестового окружения}
	\label{title:chapter2:testing_environment_description}
	Для создания единого окружения для тестирования производительности всех \marm{} удобно использовать технологию контейнерной виртуализации на уровне ОС, например, Docker. По сравнению с программной виртуализацией при помощи гипервизора, виртуализация на уровне ОС требует гораздо меньше накладных расходов на абстрагирование за счет уменьшение и упрощение слоев между ОС предоставляющей сервис и гостевым приложением. В Docker для этого используется механизм ядра Linux - cgroups, изолирующий набор ресурсов компьютера для процессов. 

Таким образом, при помощи данного подхода можно организовать изолированные Docker-контейнеры с одним окружением для каждого из рассматриваемых \marm{}.

Для этого был реализован основной Docker-образ \inline{ubuntu-dev}, который основан на ОС Ubuntu 16.04, а так же содержит ряд пакетов:
\begin{itemize}[noitemsep]
	\item для удобства управления и конфигурирования системы (locales, lsb-release);
	\item для возможности работать с GUI приложениями (ssh, xorg, xauth);
	\item git для работы с удаленными репозиториями систем контроля версий;
	\item для удобного написания кода программ (tmux, ranger, vim);
	\item для сборки и компиляции программ на C++ (build-essential, cmake, pkg-config);
	\item htop для мониторинга состояния системы;
	\item пакеты с библиотеками libboost-all-dev и libxml2-dev, использующиеся для компиляции программ для исследуемых \marm{}.
\end{itemize}

Итоговый Docker-file образа \inline{ubuntu-dev} представлен в листинге \ref{application:dockerfiles:ubuntu-dev}.

Используя созданный образ как основу, были созданы 4 образа для каждого из рассматриваемых \marm{}, Docker-файлы которых представлены на листингах \ref{application:dockerfiles:ubuntu-ros}, \ref{application:dockerfiles:ubuntu-yarp}, \ref{application:dockerfiles:ubuntu-mira} и \ref{application:dockerfiles:ubuntu-orocos}. 

Сложности возникли только с \toolchain{}, поскольку c 2015 года был удален основной репозиторий проекта. После этого исходный код постепенно переносился на инфраструктуру GitHub, но, к сожалению, в сценариях сборки проекта оставалось много зависимостей на недоступные сетевые хранилища исходного кода или библиотек. Кроме того, из-за различия в версиях некоторых пакетов ОС Ubuntu 16.04 с более старыми версиями ОС, возникали ошибки сборки проекта. Данная проблема решалась сборкой в три итерации, после первых двух требовалось ручное исправление промежуточных файлов конфигурации. Результат работы сценариев сборки проекта двух итераций был заархивирован и передается при сборке Docker-образу. Из-за сложностей, возникших при разрешении задачи сборки \toolchain{} и составления Docker-файла, была составлена подробная инструкция на английском языке. На рисунке \ref{img:dockers} показана иерархия связей Docker-образов.
\img{img/dockers.png}{Иерархия использования Docker-образов.}{img:dockers}{\textwidth}


	\subsection{Используемое API Google benchmark}
	\label{title:chapter2:testing_google_benchmark_api}
	Для разработки бенчмарков, согласно изложенному в разделе \ref{title:chapter1:performance_testing_benchmarks_review}, используется Google Benchmark. Для составления тестов использовались нижеописанные концепции.

\begin{description}[noitemsep]
	\item [Фикстуры] -- это класс-наследник от \inline{benchmark::Fixture} с переопределенными методами:
	\begin{itemize}[noitemsep]
		\item \inline{void SetUp(const benchmark::State &state)} -- функция, подготавливающая данные перед началом каждого теста. Примеры применения: 
		\begin{itemize}[noitemsep]
			\item создание сообщений определенного размера; 
			\item установка соединений.
		\end{itemize}
		\item \inline{void TearDown(const benchmark::State &state)} -- функция, вызывающаяся после окончания всех итераций тестирования. Может использоваться для корректного закрытия соединений и завершения потоков.
	\end{itemize}
	Фикстуры удобны для создания единого окружения для отдельных бенчмарк-тестов, позволяя комбинировать классы фикстур между различными тестами, а так же переиспользовать код.

	\item [Бенчмарк-тест] -- зарегистрированная для выполнения фреймворком функция. Пример описания приведен в листинге \ref{lst:chapter2:testing_google_benchmark_api:benchmark}. В данном описании сразу используется фикстура \inline{FIXTURE_NAME}, для подготовки данных для теста.
	
	\begin{lstlisting}[label=lst:chapter2:testing_google_benchmark_api:benchmark, caption=Описание бенчмарка с фикстурой]
BENCHMARK_DEFINE_F(FIXTURE_NAME, BM_TEST)(benchmark::State& state) {
	for(auto _ : state) {
		// Цикл выполнения бенчмарк-теста
	}
}
BENCHMARK_REGISTER_F(FIXTURE_NAME, BM_TEST)
	->Apply(CustomArguments);
	\end{lstlisting}
	
	Кроме того, в бенчмарк-тесты можно передавать аргументы. Передача аргументов может выполняться разными способами, но поскольку значения факторов в рассматриваемых случаях очень различны, стандартные методы Google Benchmark не подходят. Для передачи произвольного набора аргументов используется метод \inline{Apply}, в который передается идентификатор функции, которая формирует список кортежей аргументов. Пример задания произвольных значений приведен в листинге \ref{lst:chapter2:testing_google_benchmark_api:custom_args}.
	\begin{lstlisting}[label=lst:chapter2:testing_google_benchmark_api:custom_args, caption={Описание функции, формирующей список кортежей аргументов для бенчмарк-теста}]
static void CustomArguments(benchmark::internal::Benchmark* b) {
     for (int topics = 1; topics <= 32; topics *= 2)
         for (int size = (1 << 10); size <= 1 << 28 ; size *= 2)
             for (int buffer = 1; buffer <= 100000; buffer *= 10)
                    b->Args({topics, size, buffer});
}
	\end{lstlisting}
	По итогу выполнения данной функции в тесты будут передаваться кортежи параметров, например \inline{(1, 1024, 1), (1, 1024, 10)} \etc{} Для изъятия аргументов для текущего теста используется объект типа \inline{benchmark::State}, обращение к которому через метод \inline{range(index)}, вернет значение аргумента текущего кортежа аргументов по индексу \inline{index}. 
	
	\item [Пользовательские счетчики] -- это возможность по окончанию теста записать значение пользовательского счетчика в результат выполнения теста. Для этого в поле хэш-таблицы \inline{counters} объекта типа \inline{benchmark::State} передается имя (ключ) и значение пользовательского счетчика: \inline{state.counters["bm_id"] = bm_id}. Данные в хэш-таблице хранятся типа \inline{double}, таким образом важно следить за тем, чтобы не терялось значение счетчика при сохранении результата. Примером может быть сохранение идентификатора бенчмарка в сообщении, для обработки задержки передачи данных на другом узле. Идентификатором является \todo{UNIX-time} в наносекундах. Изначально идентификатор был типа \inline{unsigned long long}, но после преобразования в \inline{unsigned long double} может упасть точность до последних 4 цифр. Это следует учитывать при обработке результатов теста.
	
	\item [Управление временем] -- это возможность приостанавливать замер времени выполнения теста. Выполняется соответствующими остановке замера и возобновлению методами \inline{PauseTiming()} и \inline{ResumeTiming()} объекта типа \inline{benchmark::State}.
	
	\item [Указание способа вычисления времени] -- это возможность указать фреймворку считать время выполнения теста при помощи часов реального времени. Это важно в тех методах, в которых есть многопоточное исполнение, или ожидание ответа. Указать данное требование можно вызвав у бенчмарка метод \inline{UseRealTime()}. 
\end{description}

Пример запуска всех зарегистрированных бенчмарк-тестов показан на листинге \ref{lst:chapter2:testing_google_benchmark_api:main}.
\begin{lstlisting}[label=lst:chapter2:testing_google_benchmark_api:main, caption={Пример запуска всех зарегистрированных тестов при помощи Google Benchmark}]
int main(int argc, char** argv) {
	benchmark::Initialize(&argc, argv);
	if (benchmark::ReportUnrecognizedArguments(argc, argv)) return 1;
	benchmark::RunSpecifiedBenchmarks();
	
	return 0;
}
\end{lstlisting}

Google Benchmark имеет макрос, который генерирует представленную функцию точки входа в листинге \ref{lst:chapter2:testing_google_benchmark_api:main}, но в рассматриваемых фреймворках требуется инициализировать узлы во время начала исполнения исполняемого модуля, либо вызывать тесты в методах-обработчиков объектов классов-модулей, например, для MIRA.

При инициализации исполнения бенчмарк-тестов передаются аргументы загрузки исполняемого модуля. Для рассматриваемой задачи тестирования важны следующие пары \enquote{ключ-значение} запуска:
\begin{description}[noitemsep]
	\item [--benchmark\_out]  -- определяет URL файла с результатами тестирования. 
	\item [--benchmark\_out\_format] -- определяет формат файла с результатами тестирования. Может быть определен как \inline{console}, \inline{csv} и \inline{json}. Поскольку произвольный вывод консоли сложно поддается разбору и анализу, а csv файл содержит в себе пользовательские счетчики только первого бенчмарк-теста, то единственным подходящим вариантом будет \inline{json}.
	\item [--benchmark\_repetitions] -- указывает сколько раз требуется запустить каждый тест. В данном исследовании выборка будет состоять из 10 значений, хотя, анализатор, описываемый в разделе \ref{title:chapter2:testing_analyser} может обрабатывать выборку от 2 до 40 значений.
\end{description}

Таким образом, были определены необходимые интерфейсы и подходы для реализации бенчмарк-тестов для каждого отдельного \marm{}.
%	\subsection{\todo{Детали реализации тестовых случаев}}
%		\subsubsection{\todo{ROS}}
%		\label{title:chapter2:ros_test_details}
%		Узел инициализируется при помощи \inline{ros::init(int argc, char** argv)} и данные об узле передаются при инициализации объекта класса \inline{ros::NodeHandle}.

Для получения возможности публиковать сообщения по определенному топику или подписываться на топики, требуется инициализировать объекты \inline{ros::Publisher} для узла-издателя и \inline{ros::Subscriber} для узла-подписчика. Инициализация происходит при помощи вызова методов объекта узла класса \inline{ros::NodeHandle}:
\begin{itemize}[noitemsep]
	\item \inline{advertise()} для публикации топика и инициализации объекта издателя, где \inline{topic_name} - название топика, а 
\end{itemize}

Сообщения в ROS описываются текстовым форматом, отдельно для сообщений для механизма топиков и отдельно для сервисов, но в ходе компиляции  они преобразуются в заголовочные C++ файлы. Для сообщений сервисов создается несколько дополнительных заголовков: для запроса и ответа, а так же для случая, если ответ не требуется. Имеется возможность описать данные как массив, который в реализации C++ является объектом класса \include{std::vector}.

		
%		\subsubsection{\todo{YARP}
%		\subsubsection{\todo{MIRA}}
%		\subsubsection{\todo{OROCOS}}
	\subsection{Автоматизация обработки результатов}
	\label{title:chapter2:testing_analyser}
		\subsubsection{Описание решения}
		\label{title:chapter2:testing_analyser_description}
		Результатом выполнения бенчмарк-тестирования является множество json-файлов (не менее одного), в которых записаны результаты тестирования. Для автоматизации обработки множества результатов тестирования, а так же их визуализации была реализована программа, которая обрабатывает все указанные json-файлы c результатами и формирует отчет с таблицами и графиками результатов тестирования в формате rmarkdown. Исходный код программы представлен в приложении \ref{application:analyser}. 

Обработку результатов можно разбить на следующие итерации:
\begin{enumerate}
	\item Анализ файла конфигурации \inline{merge_config.xml} и формирование промежуточного файла \inline{full.json}, содержащего список всех бенчмарк-тестов со всеми полученными значениями измерений.
	
	Исходные json-файлы можно разделить на два типа: первичные и вторичные.
	
	\begin{description}[noitemsep]
		\item [Первичный] -- это json-файл, полученный исполнением непосредственно Google Benchmark исполняемого модуля. В данном json-файле располагается список \inline{benchmarks} c описанием данных каждого проведенного теста в формате \enquote{ключ-значение}. Наиболее важные поля:
		\begin{itemize}[noitemsep]
			\item \inline{name} -- название теста с возможными суффиксами \enquote{\_mean}, \enquote{\_stddev} и \enquote{\_median}. Суффиксы добавляются средствами Google Benchmark, если указан ключ для повторений тестов и означают соответственно среднеарифметическое результатов, среднеквадратическое отклонение и медианное значение. Поскольку анализ данных будет производиться отдельно, то записи, содержащие в имени данные суффиксы стоит исключить при обработке.
			
			\item \inline{real_time} -- задержка, основанная на системных часах. При указании замера реального времени, данное значение будет указывать на реальное время в наносекундах, которое было затрачено на выполнение теста.
			
			\item \inline{CPU_time} -- задержка, основанная на количестве циклов процессора. Наиболее точный возможный показатель, но т.к. большинство методов коммуникации в рассматриваемых \marm{} используют технологии многопоточного программирования, данный результат не может считаться адекватным во всех тестах: данный результат требуется рассматривать вместе с задержкой реального времени.
			
			\item \inline{bytes_per_second} -- широта пропускания данных, которое вычисляется на основании количества итераций, количества переданных байтов в отдельный счетчик \inline{SetBytesProcessed()} у объекта типа \inline{benchmark::State} и затраченного реального времени. Результат представляется в количестве байтов в секунду.
			
			\item \inline{bm_id} -- уникальный идентификатор теста. Данное поле задается разработчиком, Google Benchmark не предусматривает никакой автоматической идентификации тестов. Данное поле требуется для связи с результатами производительности полученными на иных узлах связи, отличных от того, на котором работает Google Benchmark.
		\end{itemize}
		
		\item[Вторичный] -- json файл с результатами тестирования полученными каким-то отличным от использования Google Benchmark образом. Например, путем вычисления реальной задержки передачи данных на узлах-получателях. Узел-получатель записывает значения измерений в формате \enquote{измерение-значение}, а для связи с узлом-отправителем, на котором исполняется Google Benchmark, используется уникальный идентификатор теста. В общем случае, не для каждой записи из вторичного json найдется соответствие в первичном, потому что Google Benchmark самостоятельно определяет какие итерации следует пропустить. Как следствие, различие в значениях, полученных во вторичных json-файлах, следует дополнительно объяснять большим количеством переданных сообщений, а так же большим разбросом в значениях. Последнее объясняется тем, что первые итерации измерения производительности скорее-всего не будут адекватными из-за отсутствия часто повторяющихся команд в кэше процессора. Google Benchmark автоматически регулирует количество итераций и набор итераций, которые идут результатом выполнения теста.
	\end{description}
	
	В общем случае, для связи первичного и вторичных json-файлов рекомендуется использовать поле \inline{bm_id}, но так же допускается конфигурирование связи между файлами по другому имени поля.
	
	Исходные json-файлы, их тип и ключ связи для вторичных json-файлах указываются в файле конфигурации \inline{merge_config.xml}. Пример файла конфигурации приведен в приложении \ref{application:config:merge}
	
	Все измерения в найденных по указанному ключу записях из вторичных файлов вставляются в записи первичного json-файла. После слияния первичных json-файлов с соответствующими им вторичными, итоговый файл \inline{full.json} получается путем объединения записей всех первичных json-файлов.
	
	\item Для каждого бенчмарк-теста из списка, полученного на предыдущей итерации, формируется список для бенчарк-тестов без повторений имен (на предыдущей итерации, поскольку тесты могли быть запущены несколько раз, имя записи теста могло повторяться пропорционально количеству запуска тестов) со сгруппированными значениями измерений. Таким образом, сформирована структура данных, которая для каждого набора факторов содержит результаты измерений пропорционально выборке эксперимента. Структура сохраняется в файл \inline{benchmark_processed.json}
	
	\item Далее все результаты статистически обрабатываются: находится среднее значение измерения и среднеквадратичная ошибка среднеарифметического. Результаты записываются в файл \inline{benchmarks_results.json}.
	
	\item Предпоследней итерацией является разбиение результатов, путем получения различных комбинаций тестов, основываясь на определяющих их факторах. Для определения информации о факторах тестирования используется описание факторов в конфигурационном файле первой итерации.
	
	На каждом шаге выбирается переменный фактор, являющийся абсциссой на графиках результатов тестирования. Для всех остальных факторов требуется найти все возможные фиксированные комбинации, относительно которых будет изменяться выбранный переменный фактор.
	
	Таким образом будет сформировано множество таблиц, представляющих зависимость результатов измерений от выбранного переменного фактора и фиксированных остальных.
	
	Результирующий список описания табличных значений записывается в файл \inline{tables.json}.
	
	\item Финальной итерацией является трансляция описаний табличных значений результатов тестирования производительности из файла \inline{tables.json} в rmarkdown файл \inline{result.rmd}. Для конфигурации именования измерений и факторов, а так же описания генерируемых графиков, используется конфигурационный файл \inline{rmd_config.xml}, пример которого представлен в приложении \ref{application:config:rmd}.
\end{enumerate}

Таким образом, было реализовано приложение, принимающее в качестве входных данных множество результатов измерений производительности в формате json, а так же двух конфигурационных файлов с описаниями исходных данных и описанием выходного результата. Результатом работы приложения является rmarkdown-файл, готовый к преобразованию в html страницу при помощи транслятора knit.
		\subsubsection{Примеры работы}
		\label{title:chapter2:analyser_examples}
		На рисунках \ref{img:chapter2:analyser_toc}, \ref{img:chapter2:analyser_table}, \ref{img:chapter2:analyser_latency} и \ref{img:chapter2:analyser_bandwidth} представлены скриншоты html-страницы, полученной из rmarkdown-файла -- результата работы анализатора на некоторых исходных данных.

\img{img/analyser_toc.png}{Содержание, состоящее из гиперссылок на соответствующие результаты тестов}{img:chapter2:analyser_toc}{\textwidth}

\img{img/analyser_table.png}{Таблица средних результатов измерений для переменного фактора \enquote{size}}{img:chapter2:analyser_table}{\textwidth}

\img{img/analyser_latency.png}{Пример графика зависимости времени задержки от размера сообщения}{img:chapter2:analyser_latency}{\textwidth}

\img{img/analyser_bandwidth.png}{Пример графика зависимости широты пропускания от размера буфера}{img:chapter2:analyser_bandwidth}{\textwidth}
\section{Выводы}
	Таким образом, были выявлены предмет тестирования и тестовые случаи, было создано тестовое окружение при помощи Docker-контейнеров, реализованы тестовые случаи при помощи описанного API Google Benchmark и было реализовано приложение, обрабатывающее входные результаты экспериментов в формате json и дающее на выходе rmarkdown файл с таблицами и графиками результатов экспериментов, который может быть преобразован в html страницу.

