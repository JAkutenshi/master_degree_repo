\begin{table}[]
	\centering
	\caption{Факторизация параметров для тестирования производительности MIRA}
	\label{table:chapter2:mira_test_cases}
%	\footnotesize
	\def\arraystretch{1.3}
%	\resizebox{\textwidth}{!}{%
		\begin{tabular}{|p{4cm}|p{3cm}|p{3cm}|p{3cm}|}
			\hline
			\textbf{Подход к коммуникации} & \textbf{Реализация модуля} & \textbf{Размер сообщения} & \textbf{Время хранения сообщений в очереди} \\ \hline
			Шаблон наблюдатель & В одном процессе (разделяемая память) & 1 КБ   & 1 мс              \\ \hline
			RPC                & В разных процессах (TCP)              & 4 КБ   & 10 мс             \\ \hline
			\multirow{8}{*}{}  & \multirow{8}{*}{}                     & 16 КБ  & 100 мс            \\ \cline{3-4} 
			                   &                                       & 64 КБ  & 1 с               \\ \cline{3-4} 
			                   &                                       & 256 КБ & 10 с              \\ \cline{3-4} 
			                   &                                       & 1 МБ   & 100 с             \\ \cline{3-4} 
			                   &                                       & 16 МБ  & \multirow{4}{*}{} \\ \cline{3-3}
			                   &                                       & 64 МБ  &                   \\ \cline{3-3}
			                   &                                       & 256 МБ &                   \\ \cline{3-3}
                               &                                       & \todo{1 ГБ} &  \\ \hline
		\end{tabular}%
%	}
	
\end{table}