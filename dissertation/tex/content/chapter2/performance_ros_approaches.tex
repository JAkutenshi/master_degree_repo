Система коммуникации в ROS состоит из двух типов: взаимодействие между узлами при помощи топиков и сервисов \cite{o2014gentle}.

\begin{itemize}[noitemsep]
	\item \textit{Взаимодйствие при помощи топиков} -- это широковещательный подход взаимодействия по шаблону \enquote{наблюдатель (издатель-подписчик)}. Узлы-издатели публикуют \enquote{тему} общения -- топик -- доступную для всех узлов. Все подписавшиеся на опубликованный топик узлы-читатели получают возможность читать отправленные на данный топик сообщения. Факторы, влияющие на производительность:

	\item \textit{Клиент-сервисное взаимодействие} -- это подход, согласно которому каждый узел может реализовывать сущность RPC сервиса, исполняющего какой-либо процесс по запросу узла-клиента, возможно, с аргументами, и, возвращающий, ответ на запрос, возможно, пустой. Факторы, влияющие на производительность, в сравнении с топиками, идентичные, но, с точки зрения метрик, отдельно требуется рассматривать задержку передачи как для запроса, так и для ответа с разным объемом сообщений для запроса и ответа.
\end{itemize}

Узлы взаимодействуют напрямую, Мастер-сервис только предоставляет информацию об именах существующих узлов, как DNS \todo{(Domain Name Service)}. Узлы-подписчики посылают запрос узлу, посылающему сообщения по определенному топику, и устанавливает с ним соединение по протоколу TCPROS \cite{ros-tcpros}, использующий в себе стандартные TCP/IP сокеты \cite{ros-concepts}.

Управление потоками данных и приоритетом сообщений между узлами в ROS отсутствует. Этот механизм планируется реализовать в ROS 2.0.

Таким образом, факторы, влияющие на производительность коммуникации в ROS:
\begin{itemize}[noitemsep]
	\item тип взаимодействия:
	\begin{itemize}[noitemsep]
		\item через топики;
		\item через сервисы.
	\end{itemize}
	\item количество взаимодействующих узлов
	\begin{itemize}[noitemsep]
		\item подписчиков для топиков;
		\item клиентов для сервисов.
	\end{itemize}
	\item объем сообщений;
	\item размер буферов каналов коммуникации.
\end{itemize}