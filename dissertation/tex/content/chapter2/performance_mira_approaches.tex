В MIRA предоставляются стандартные два вида коммуникации: отправка сообщений между узлами по шаблону \enquote{наблюдатель} и удаленный вызов методов. В терминологии MIRA, узлами являются модули (units), объединяющиеся в домены. Модули компилируются и используются как разделяемые библиотеки \cite{mira-units}, что позволяет загружать нужные модули во время исполнения робототехнической системы на прикладном уровне. Основанный на разделяемом коде, высоком уровне абстракции и сериализации объектов классов при помощи сохранении информации о классе -- рефлексии, как аналогичный механизм, использующийся в Java для получения метаинформации о классах объектов во время исполнения программ -- подход позволяет уменьшить затраты ресурсов памяти в прикладных приложениях, а так же предоставляет разделение ответственности для разработчиков, позволяя максимально абстрактно реализовывать модули, подобно плагинам. 

В отличии от ROS и YARP, MIRA не позволяет как-либо управлять очередью сообщений. Известно, что эта очередь есть и имеется возможность сохранять сообщения в памяти на определенный промежуток времени, который может указать разработчик. Это позволяет получать доступ к \enquote{прошлому} - к сообщениям, которые пришли какое-то время назад. 

Для передачи сообщений между процессами используется протокол TCP, внутри одного процесса -- разделение памяти.

Объекты сообщений передаются между модулями в бинарном виде путем сериализации объектов, созданных при помощи шаблона \enquote{фабрика объектов}. Вся метаинформация указывается в переопределяемом методе \inline{template <typename Reflector> void mira::reflect(Reflector& r)} для любого объекта MIRA, таким образом, позволяя передавать любые объекты в среде коммуникации MIRA.

Управление потоками данных, приоритетами сообщений между модулями отсутствует.

Таким образом, основными факторами, возможно, влияющими на производительность системы коммуникации MIRA являются:
\begin{itemize}[noitemsep]
	\item способ передачи данных:
	\begin{itemize}[noitemsep]
		\item по шаблону \enquote{наблюдатель};
		\item при помощи вызова удаленных процедур.
	\end{itemize}
	\item локализация модуля:
	\begin{itemize}[noitemsep]
		\item в одном процессе (используется разделяемая память);
		\item в разных (используется протокол TCP).
	\end{itemize}
	\item объем пересылаемых данных;
	\item время хранения сообщений в буфере соединения.
\end{itemize}