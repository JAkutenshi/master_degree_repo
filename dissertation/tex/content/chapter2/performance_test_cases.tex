Исходя из факторов, описанных в разделе \ref{title:chapter2:performance_approaches}, можно составить набор сценариев для проведения тестирования производительности.

\subsection{ROS}
В столбцах таблицы \ref{table:chapter2:ros_test_cases} приведены различные значения по выделенным ранее факторам для тестирования. Таким образом, множество всех сценариев тестов производительности -- это все возможные подмножества из возможных значений факторов таблицы. \ref{table:chapter2:ros_test_cases}.
\begin{table}[!h]
\centering
\small
\def\arraystretch{1.3}
\caption{Факторизация параметров для тестирования производительности ROS}
\label{table:chapter2:ros_test_cases}
\resizebox{\textwidth}{!}{%
\begin{tabular}{|p{4cm}|p{3cm}|p{3cm}|p{2cm}|}
\hline
\textbf{Тип взаимодействия} & \textbf{Размер сообщения} & \textbf{Количество подписчиков или клиентов} & \textbf{Размер буфера}     \\ \hline
Издатель/ подписчик & 1 КБ             & 1                                   & 10                \\ \hline
Сервисы            & 10 КБ            & 2                                   & 100               \\ \hline
\multirow{8}{*}{}  & 100 КБ           & 4                                   & 1000              \\ \cline{2-4} 
                   & 1 МБ             & 8                                   & 2000              \\ \cline{2-4} 
                   & 10 МБ            & 16                                  & 5000              \\ \cline{2-4} 
                   & 50 МБ            & 32                                  & 10000             \\ \cline{2-4} 
                   & 100 МБ           & \multirow{4}{*}{}                   & \multirow{4}{*}{} \\ \cline{2-2}
                   & 250 МБ           &                                     &                   \\ \cline{2-2}
                   & 500 МБ           &                                     &                   \\ \cline{2-2}
                   & \todo{1 ГБ}             &                                     &                   \\ \hline
\end{tabular}%
}
\end{table}

\subsection{YARP}
В столбцах таблицы \ref{table:chapter2:yarp_test_cases} приведены различные значения по выделенным ранее факторам для тестирования. Таким образом, множество всех тестов, это все возможные подмножества из возможных значений факторов из столбцов таблицы \ref{table:chapter2:yarp_test_cases}.
% Please add the following required packages to your document preamble:
% \usepackage{multirow}
% \usepackage{graphicx}
\begin{table}[]
	\centering
	\caption{Факторизация параметров для тестирования производительности YARP}
	\label{table:chapter2:yarp_test_cases}
%	\footnotesize
	\def\arraystretch{1.5}
	\newcommand{\sr}{\rule[-0.45cm]{0pt}{0.9cm}}
%	\resizebox{\textwidth}{!}{%
		\begin{tabular}{|p{2cm}|p{2.5cm}|p{2.4cm}|p{2.1cm}|p{2.7cm}|p{2.4cm}|}
		\hline
		\multirow{2}{2cm}{\textbf{Тип порта}} & \multirow{2}{2.5cm}{\textbf{Транспорт соединения}} & \multirow{2}{2.4cm}{\textbf{Размер сообщения}} & \multirow{2}{2.1cm}[-0.2cm]{\textbf{Число входящих портов}} & \multicolumn{2}{l|}{ \sr \textbf{QoS}} \\ \cline{5-6} 
		 & & & & \sr \textbf{Статус } & \sr \textbf{Приоритет} \\ \hline
		Обычный           & tcp   & 1 КБ    & 1  & Применяется       & LOW               \\ \hline
		С буфером         & udp   & 4 КБ    & 2  & Не применяется    & NORMAL            \\ \hline
		RPC               & mcast & 16 КБ   & 4  & \multirow{8}{*}{} & HIGH              \\ \cline{1-4} \cline{6-6} 
		\multirow{7}{*}{} & shmem & 64 КБ   & 8  &                   & CRITICAL          \\ \cline{2-4} \cline{6-6} 
		 & \todo{local}           & 256 КБ  & 16                &    & \multirow{6}{*}{} \\ \cline{2-4}
	     & fast\_tcp              & 1 МБ    & 32                &    &                   \\ \cline{2-4}
		 & \multirow{4}{*}{}      & 4 МБ    & \multirow{4}{*}{} &    &                   \\ \cline{3-3}
		 &                        & 16 МБ   &                   &    &                   \\ \cline{3-3}
		 &                        & 64 МБ   &                   &    &                   \\ \cline{3-3}
		 &                        & 256 МБ  &                   &    &                   \\ \cline{3-3}
		 &                        & \todo{1 ГБ}  &              &    &                   \\ \hline
		\end{tabular}%
%	}

\end{table}

\subsection{MIRA}
В столбцах таблицы \ref{table:chapter2:mira_test_cases} приведены различные значения по выделенным ранее факторам для тестирования. Таким образом, множество всех сценариев тестов производительности -- это все возможные подмножества из возможных значений факторов таблицы \ref{table:chapter2:mira_test_cases}.
\begin{table}[]
	\centering
	\caption{Факторизация параметров для тестирования производительности MIRA}
	\label{table:chapter2:mira_test_cases}
%	\footnotesize
	\def\arraystretch{1.3}
%	\resizebox{\textwidth}{!}{%
		\begin{tabular}{|p{4cm}|p{3cm}|p{3cm}|p{3cm}|}
			\hline
			\textbf{Подход к коммуникации} & \textbf{Реализация модуля} & \textbf{Размер сообщения} & \textbf{Время хранения сообщений в очереди} \\ \hline
			Шаблон наблюдатель & В одном процессе (разделяемая память) & 1 КБ   & 1 мс              \\ \hline
			RPC                & В разных процессах (TCP)              & 4 КБ   & 10 мс             \\ \hline
			\multirow{8}{*}{}  & \multirow{8}{*}{}                     & 16 КБ  & 100 мс            \\ \cline{3-4} 
			                   &                                       & 64 КБ  & 1 с               \\ \cline{3-4} 
			                   &                                       & 256 КБ & 10 с              \\ \cline{3-4} 
			                   &                                       & 1 МБ   & 100 с             \\ \cline{3-4} 
			                   &                                       & 4 МБ   & \multirow{4}{*}{} \\ \cline{3-3}
			                   &                                       & 16 МБ  &                   \\ \cline{3-3}
  			                   &                                       & 64 МБ  &                   \\ \cline{3-3}
			                   &                                       & 256 МБ &                   \\ \cline{3-3}
                               &                                       & \todo{1 ГБ} &  \\ \hline
		\end{tabular}%
%	}
	
\end{table}

%\subsection{\orocos{}}
%В данном случае имеется четкое разделение на взаимодействие компонентов внутри одного процесса и между процессами. В таком случае, образуется 4 возможных базовых сценариев тестирования:
%\begin{itemize}[noitemsep]
%	\item внутрипроцессное взаимодействие при помощи разделяемой памяти;
%	\item межпроцессное взаимодействие:
%	\begin{itemize}[noitemsep]
%		\item MQueue;
%		\item CORBA;
%		\item CORBA + MQueue;
%	\end{itemize}	
%\end{itemize}
%Таким образом, в таблице \ref{table:chapter2:orocos_test_cases} приведены различные значения по выделенным ранее факторам для тестирования, а множество всех сценариев тестов производительности -- это все возможные подмножества из возможных значений факторов таблицы, не считая первую половину столбца 1, который представлен для разделения расположения компонентов: в одном или в разных процессах.
%\begin{table}[]
	\centering
	\caption{Факторизация параметров для тестирования производительности \toolchain}
	\label{table:chapter2:orocos_test_cases}
%	\footnotesize
	\def\arraystretch{1.3}
%	\resizebox{\textwidth}{!}{%
		\begin{tabular}{|p{3.2cm}|p{2.6cm}|p{3.6cm}|p{2.5cm}|p{1.6cm}|}
			\hline
			\multicolumn{2}{|l|}{\textbf{Протокол коммуникации}} & \multirow{2}{3.6cm}{\textbf{Тип взаимодействия}} & \multirow{2}{2.5cm}{\textbf{Размер сообщения}} & \multirow{2}{1.6cm}{\textbf{Размер буфера}} \\ \cline{1-2}
			\textbf{Расположение} & \textbf{Протокол} &  &  &  \\ \hline
			Внутри процесса & Разделяемая память & Порты & 1 КБ & 10 \\ \hline
			Между процессами & MQueue & Удаленный вызов операции & 4 КБ & 100 \\ \hline
			\multirow{2}{*}{} & CORBA & Сервис & 16 КБ & 1000 \\ \cline{2-5} 
			& CORBA+ MQueue & \multirow{7}{*}{} & 64 КБ & 2000 \\ \cline{1-2} \cline{4-5} 
			\multicolumn{2}{|l|}{\multirow{6}{*}{}} &  & 256 КБ & 5000 \\ \cline{4-5} 
			\multicolumn{2}{|l|}{} &  & 1 МБ & 10000 \\ \cline{4-5} 
			\multicolumn{2}{|l|}{} &  & 16 МБ & \multirow{4}{*}{} \\ \cline{4-4}
			\multicolumn{2}{|l|}{} &  & 64 МБ &  \\ \cline{4-4}
			\multicolumn{2}{|l|}{} &  & 256 МБ &  \\ \cline{4-4}
			\multicolumn{2}{|l|}{} &  & \todo{1 ГБ} &  \\ \hline
		\end{tabular}%
%	}
	
\end{table}