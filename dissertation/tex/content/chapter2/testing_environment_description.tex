Для создания единого окружения для тестирования производительности всех \marm{} удобно использовать технологию контейнерной виртуализации на уровне ОС, например, Docker. По сравнению с программной виртуализацией при помощи гипервизора, виртуализация на уровне ОС требует гораздо меньше накладных расходов на абстрагирование за счет уменьшение и упрощение слоев между ОС предоставляющей сервис и гостевым приложением. В Docker для этого используется механизм ядра Linux - cgroups, изолирующий набор ресурсов компьютера для процессов. 

Таким образом, при помощи данного подхода можно организовать изолированные Docker-контейнеры с одним окружением для каждого из рассматриваемых \marm{}.

Для этого был реализован основной Docker-образ \inline{ubuntu-dev}, который основан на ОС Ubuntu 16.04, а так же содержит ряд пакетов:
\begin{itemize}[noitemsep]
	\item для удобства управления и конфигурирования системы (locales, lsb-release);
	\item для возможности работать с GUI приложениями (ssh, xorg, xauth);
	\item git для работы с удаленными репозиториями систем контроля версий;
	\item для удобного написания кода программ (tmux, ranger, vim);
	\item для сборки и компиляции программ на C++ (build-essential, cmake, pkg-config);
	\item htop для мониторинга состояния системы;
	\item пакеты с библиотеками libboost-all-dev и libxml2-dev, использующиеся для компиляции программ для исследуемых \marm{}.
\end{itemize}

Итоговый Docker-file образа \inline{ubuntu-dev} представлен в листинге \ref{application:dockerfiles:ubuntu-dev}.

Используя созданный образ как основу, были созданы 4 образа для каждого из рассматриваемых \marm{}, Docker-файлы которых представлены на листингах \ref{application:dockerfiles:ubuntu-ros}, \ref{application:dockerfiles:ubuntu-yarp}, \ref{application:dockerfiles:ubuntu-mira} и \ref{application:dockerfiles:ubuntu-orocos}. 

Сложности возникли только с \toolchain{}, поскольку c 2015 года был удален основной репозиторий проекта. После этого исходный код постепенно переносился на инфраструктуру GitHub, но, к сожалению, в сценариях сборки проекта оставалось много зависимостей на недоступные сетевые хранилища исходного кода или библиотек. Кроме того, из-за различия в версиях некоторых пакетов ОС Ubuntu 16.04 с более старыми версиями ОС, возникали ошибки сборки проекта. Данная проблема решалась сборкой в три итерации, после первых двух требовалось ручное исправление промежуточных файлов конфигурации. Результат работы сценариев сборки проекта двух итераций был заархивирован и передается при сборке Docker-образу. Из-за сложностей, возникших при разрешении задачи сборки \toolchain{} и составления Docker-файла, была составлена подробная инструкция на английском языке. На рисунке \ref{img:dockers} показана иерархия связей Docker-образов.
\img{img/dockers.png}{Иерархия использования Docker-образов.}{img:dockers}{\textwidth}

