В YARP основной способ передачи информации - система портов. Как и в ROS, это механизм, реализующий шаблон \enquote{наблюдатель}. Отличие от ROS в том, что реализован этот шаблон через подход портов - устройства потокового ввода и вывода. Кроме того, как и в ROS, имеется возможность использовать механизм RPC между узлами системы. Ниже разобраны оба механизма.

\begin{itemize}
	\item \textit{Порты} -- это именованные объекты, которые обеспечивают доставку сообщений множеству других портов, подписанных на данные порты по данному имени. Один узел может иметь множество портов внутри. 
	
	Порты в YARP реализованы как объекты потоков, в которые можно писать данные и с которых данные можно считывать. Этот подход позволяет разделить отправителя данных и принимающего данные, таким образом, соответствующим портам не требуется знать много информации друг о друге. Это с одной стороны повышает абстрактность, позволяя работать со множеством портом сразу и допускать, например, перезапуск или временную неработоспособность какого-либо узла, с другой влияет на производительность, так как полностью от зависимости передающего и получателя сообщения избавиться нельзя. В частности, порт знает, когда передача сообщений конкретному порту получателю закончена: только после этого посылающий может приступать к передаче следующему адресату. Объекты, которые пересылаются между портами, не копируются. Эту проблему можно решить разными способами \cite{yarp-ports}, например, хранением сообщений в очереди на отправку.
	
	Порты бывают двух типов: обычные и с буфером сообщений. Второй тип отличается от первого наличием очереди сообщений, в слотах которой под сообщения резервируются сообщение перед отправкой. В отличии от обычных портов, порты с буфером сообщений теперь отвечают за время жизни объектов, которые требуется пересылать между портами, предоставляя порту самому определять, например, очередность передачи сообщений. Например, поскольку, находящееся в очереди сообщение можно модифицировать, порт с буфером может определить сообщение до и после изменения данных и не станет пересылать старое сообщение, если не выставлен определенный флаг, указывающий на обязательность передачи всех сообщений. В отличии от ROS размер буфера порта не фиксированный и зависит от количества сообщений в очереди: длина очереди увеличивается если количество сообщений превышает некоторый порог.
	
	Обычные порты \inline{yarp::os::Port} и с буфером \inline{yarp::os::BufferedPort} сильно между собой отличаются и разумно сравнить их между собой для различных данных и различной нагрузке портов-получателей. Примером различия в производительности может служить возможность, если не выставлен специальный флаг, перейти к отправке других сообщений другим портам-адресатам пока все порты-получатели конкретного сообщения заняты.
	
	\item \textit{RPC} -- механизм синхронизированной передачи данных между портами. Формально, поскольку все порты имеют возможность двусторонней связи, не требуется никаких дополнительных абстракций для реализации подобного подхода в YARP. Тем не менее, разработчики выделили в отдельные классы \inline{yarp::os::RpcClient} и \inline{yarp::os::RpcServer} -- наследники базового класса всех портов \inline{yarp::os::Contactable} -- для наличия готового инструмента с обеспечением целостности и синхронности передачи данных. В отличии от ROS, в YARP вся коммуникация идет через порты, через единообразный интерфейс устройства потокового ввода и вывода. 
\end{itemize}

В YARP не требуется описывать данные в особых форматах. В самом простом случае, разработчик реализует обычный C++ класс и передает его шаблонным параметром при создании порта. Возможны трудности в случае, если передаются данные между различными машинными архитектурами, если данным требуется не очевидная сериализация. В целом, большинство проблем решаемы при помощи предоставляемых YARP макросами, интерфейсами и классами-обертками.

Для передачи данных может использоваться множество протоколов, а так же \enquote{транспортов} (carriers) - классов-обработчиков объектов-сообщений для передачи между портами. Есть множество транспортов, реализованных в YARP и доступных для работы со распространенными транспортными протоколами:
\begin{itemize}[noitemsep]
	\item TCP;
	\item UDP;
	\item multicast -- широковещательный протокол, наиболее эффективная реализация для передачи сообщений множеству YARP-портов;
	\item shared memory -- передача данных в пределах одной машины используя разделяемую область оперативной памяти.
\end{itemize}
Преимущество YARP состоит в том, что способ транспортировки сообщения в соединении изменяем в любой момент времени как реализацией внутри узлов системы, так и внешним конфигурированием узлов, что позволяет гибко управлять коммуникацией внутри робототехнической системы.

Отдельно стоит указать возможность управлять приоритетом соединений, т.е. наличие QoS (Quality of Service). Для определенного соединения можно назначить один из четырех возможных приоритетов передачи данных.

Узлы взаимодействуют напрямую, взаимодействия с сервисом имен можно избежать, задавая имена портам при инициализации самостоятельно. В таком случае контроль уникальности идентификаторов ложится на разработчика.

Таким образом, основным способом связи в YARP являются порты, на производительность передачи данных будут влиять следующие факторы:
\begin{itemize}[noitemsep]
	\item тип порта:
	\begin{itemize}[noitemsep]
		\item обычный;
		\item с буфером
		\item реализующий синхронизированную передачу данных;
	\end{itemize}
	\item тип транспорта (carriers):
	\begin{itemize}[noitemsep]
		\item tcp;
		\item udp;
		\item mcast -- широковещательный транспорт;
		\item shmem -- транспорт при помощи локально разделяемой памяти (используется ACE - Adaptive Communication Environment);
		\item local -- использование разделения памяти внутри одного процесса;
		\item fast\_tcp -- реализация tcp транспорта без подтверждения пакетов о доставке.
	\end{itemize}
	\item размер данных;
	\item количество портов-получателей;
	\item приоритет соединений, задаваемых при помощи QoS.
\end{itemize}