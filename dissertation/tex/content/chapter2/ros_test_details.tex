Узел инициализируется при помощи \inline{ros::init(int argc, char** argv)} и данные об узле передаются при инициализации объекта класса \inline{ros::NodeHandle}.

Для получения возможности публиковать сообщения по определенному топику или подписываться на топики, требуется инициализировать объекты \inline{ros::Publisher} для узла-издателя и \inline{ros::Subscriber} для узла-подписчика. Инициализация происходит при помощи вызова методов объекта узла класса \inline{ros::NodeHandle}:
\begin{itemize}[noitemsep]
	\item \inline{advertise()} для публикации топика и инициализации объекта издателя, где \inline{topic_name} - название топика, а 
\end{itemize}

Сообщения в ROS описываются текстовым форматом, отдельно для сообщений для механизма топиков и отдельно для сервисов, но в ходе компиляции  они преобразуются в заголовочные C++ файлы. Для сообщений сервисов создается несколько дополнительных заголовков: для запроса и ответа, а так же для случая, если ответ не требуется. Имеется возможность описать данные как массив, который в реализации C++ является объектом класса \include{std::vector}.
