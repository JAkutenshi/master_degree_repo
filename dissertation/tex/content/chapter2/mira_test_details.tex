Поскольку особенностью MIRA является компиляция модулей не в исполняемые модули, а в динамические библиотеки, то использование Google Benchmark не представляется возможным. Учитывая, что этот подход используется так же в \toolchain{}, то необходимо реализовать библиотеку, которая выполняла бы тестирование без требования определенной точки входа в программу.

Решением данной проблемы является реализация отдельного объекта-бенчмарка, который создается динамически, а так же контейнера, который содержит множество данных объектов, а так же поддерживает сериализацию результатов тестирования в совместимый с Google Benchmark формат JSON.

\img{img/uc.png}{Диаграмма прецедентов использования для библиотеки тестирования производительности}{img:uc}{.5\textwidth}
\img{img/class.png}{Диаграмма классов для библиотеки тестирования производительности}{img:class}{\textwidth}
\img{img/sm.png}{Диаграмма состояний для библиотеки тестирования производительности}{img:sm}{.5\textwidth}

Диаграмма прецедентов использования представлена на рисунке \ref{img:uc}. 

Диаграмма классов представлена на рисунке \ref{img:class}. 

Диаграмма состояний представлена на рисунке \ref{img:sm}. 

Использование библиотеки возможно как с использованием методов классов \inline{Benchmark} и \inline{BenchmarkKeeper}, так и при помощи макросов. Полная реализация представлена в приложении \ref{application:lib}.

Преимущества реализации:
\begin{itemize}[noitemsep]
	\item библиотека реализована только на заголовочных файлах, не требуется предварительная компиляция;
	\item интерфейс использования реализован при помощи макросов, использование которых приближено к методу сэмплирования в профилировщиках;
	\item тесты выполняются фиксированное количество итераций;
	\item первые итерации отбрасываются для повышения точности результата;
	\item сериализация результатов тестирования в JSON файл, совместимый с Google Benchmark.
\end{itemize}

Таким образом, весь необходимый функционал для тестирования реализован, а так же предоставляется возможность использовать реализованную библиотеку для тестирования таких фреймворков, как MIRA и ORoCoS, в которых узлы системы - скомпилированный код, загружающийся во время исполнения всей робототехнической системы.


