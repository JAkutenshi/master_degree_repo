Для разработки бенчмарков, согласно изложенному в разделе \ref{title:chapter1:performance_testing_benchmarks_review}, используется Google Benchmark. Для составления тестов использовались нижеописанные концепции.

\begin{description}[noitemsep]
	\item [Фикстуры] -- это класс-наследник от \inline{benchmark::Fixture} с переопределенными методами:
	\begin{itemize}[noitemsep]
		\item \inline{void SetUp(const benchmark::State &state)} -- функция, подготавливающая данные перед началом каждого теста. Примеры применения: 
		\begin{itemize}[noitemsep]
			\item создание сообщений определенного размера; 
			\item установка соединений.
		\end{itemize}
		\item \inline{void TearDown(const benchmark::State &state)} -- функция, вызывающаяся после окончания всех итераций тестирования. Может использоваться для корректного закрытия соединений и завершения потоков.
	\end{itemize}
	Фикстуры удобны для создания единого окружения для отдельных бенчмарк-тестов, позволяя комбинировать классы фикстур между различными тестами, а так же переиспользовать код.

	\item [Бенчмарк-тест] -- зарегистрированная для выполнения фреймворком функция. Пример описания приведен в листинге \ref{lst:chapter2:testing_google_benchmark_api:benchmark}. В данном описании сразу используется фикстура \inline{FIXTURE_NAME}, для подготовки данных для теста.
	
	\begin{lstlisting}[label=lst:chapter2:testing_google_benchmark_api:benchmark, caption=Описание бенчмарка с фикстурой]
BENCHMARK_DEFINE_F(FIXTURE_NAME, BM_TEST)(benchmark::State& state) {
	for(auto _ : state) {
		// Цикл выполнения бенчмарк-теста
	}
}
BENCHMARK_REGISTER_F(FIXTURE_NAME, BM_TEST)
	->Apply(CustomArguments);
	\end{lstlisting}
	
	Кроме того, в бенчмарк-тесты можно передавать аргументы. Передача аргументов может выполняться разными способами, но поскольку значения факторов в рассматриваемых случаях очень различны, стандартные методы Google Benchmark не подходят. Для передачи произвольного набора аргументов используется метод \inline{Apply}, в который передается идентификатор функции, которая формирует список кортежей аргументов. Пример задания произвольных значений приведен в листинге \ref{lst:chapter2:testing_google_benchmark_api:custom_args}.
	\begin{lstlisting}[label=lst:chapter2:testing_google_benchmark_api:custom_args, caption={Описание функции, формирующей список кортежей аргументов для бенчмарк-теста}]
static void CustomArguments(benchmark::internal::Benchmark* b) {
     for (int topics = 1; topics <= 32; topics *= 2)
         for (int size = (1 << 10); size <= 1 << 28 ; size *= 2)
             for (int buffer = 1; buffer <= 100000; buffer *= 10)
                    b->Args({topics, size, buffer});
}
	\end{lstlisting}
	По итогу выполнения данной функции в тесты будут передаваться кортежи параметров, например \inline{(1, 1024, 1), (1, 1024, 10)} \etc{} Для изъятия аргументов для текущего теста используется объект типа \inline{benchmark::State}, обращение к которому через метод \inline{range(index)}, вернет значение аргумента текущего кортежа аргументов по индексу \inline{index}. 
	
	\item [Пользовательские счетчики] -- это возможность по окончанию теста записать значение пользовательского счетчика в результат выполнения теста. Для этого в поле хэш-таблицы \inline{counters} объекта типа \inline{benchmark::State} передается имя (ключ) и значение пользовательского счетчика: \inline{state.counters["bm_id"] = bm_id}. Данные в хэш-таблице хранятся типа \inline{double}, таким образом важно следить за тем, чтобы не терялось значение счетчика при сохранении результата. Примером может быть сохранение идентификатора бенчмарка в сообщении, для обработки задержки передачи данных на другом узле. Идентификатором является UNIX-time в наносекундах. Изначально идентификатор был типа \inline{unsigned long long}, но после преобразования в \inline{unsigned long double} может упасть точность до последних 4 цифр. Это следует учитывать при обработке результатов теста.
	
	\item [Управление временем] -- это возможность приостанавливать замер времени выполнения теста. Выполняется соответствующими остановке замера и возобновлению методами \inline{PauseTiming()} и \inline{ResumeTiming()} объекта типа \inline{benchmark::State}.
	
	\item [Указание способа вычисления времени] -- это возможность указать фреймворку считать время выполнения теста при помощи часов реального времени. Это важно в тех методах, в которых есть многопоточное исполнение, или ожидание ответа. Указать данное требование можно вызвав у бенчмарка метод \inline{UseRealTime()}. 
\end{description}

Пример запуска всех зарегистрированных бенчмарк-тестов показан на листинге \ref{lst:chapter2:testing_google_benchmark_api:main}.
\begin{lstlisting}[label=lst:chapter2:testing_google_benchmark_api:main, caption={Пример запуска всех зарегистрированных тестов при помощи Google Benchmark}]
int main(int argc, char** argv) {
	benchmark::Initialize(&argc, argv);
	if (benchmark::ReportUnrecognizedArguments(argc, argv)) return 1;
	benchmark::RunSpecifiedBenchmarks();
	
	return 0;
}
\end{lstlisting}

Google Benchmark имеет макрос, который генерирует представленную функцию точки входа в листинге \ref{lst:chapter2:testing_google_benchmark_api:main}, но в рассматриваемых фреймворках требуется инициализировать узлы во время начала исполнения исполняемого модуля, либо вызывать тесты в методах-обработчиков объектов классов-модулей, например, для MIRA.

При инициализации исполнения бенчмарк-тестов передаются аргументы загрузки исполняемого модуля. Для рассматриваемой задачи тестирования важны следующие пары \enquote{ключ-значение} запуска:
\begin{description}[noitemsep]
	\item [--benchmark\_out]  -- определяет URL файла с результатами тестирования. 
	\item [--benchmark\_out\_format] -- определяет формат файла с результатами тестирования. Может быть определен как \inline{console}, \inline{csv} и \inline{json}. Поскольку произвольный вывод консоли сложно поддается разбору и анализу, а csv файл содержит в себе пользовательские счетчики только первого бенчмарк-теста, то единственным подходящим вариантом будет \inline{json}.
	\item [--benchmark\_repetitions] -- указывает сколько раз требуется запустить каждый тест. В данном исследовании выборка будет состоять из 10 значений, хотя, анализатор, описываемый в разделе \ref{title:chapter2:testing_analyser} может обрабатывать выборку от 2 до 40 значений.
\end{description}

Таким образом, были определены необходимые интерфейсы и подходы для реализации бенчмарк-тестов для каждого отдельного \marm{}.