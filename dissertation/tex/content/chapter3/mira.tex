\subsection{Различие производительности различных типов модулей}
MIRA выделяется подходом к модулям - узлам распределенной системы. В MIRA модулями являются не исполняемые модули, а объекты классов, полученные при помощи реализации шаблона \enquote{Фабрика объектов} внутри самого фреймворка. В общем случае, каждый модуль - это объект класса с определенным интерфейсом. Все модули компилируются как разделяемые библиотеки. Это позволяет реализовать наиболее быстрое межпроцессное взаимодействие: разделяемая память внутри процесса. 

Кроме того, имеется возможность запускать модули в разных процессах, разработчики реализовывают взаимодействие в данном случае при помощи протокола TCP. Тем не менее, гарантия доставки сообщений отсутствует, т.к. сообщение может быть на момент прибытия в нужный узел \enquote{неактуальным} и будет пропущено. В случае, если модули находятся в разных процессах этот факт очень заметен: в 7 из 90 случаев сообщения в ходе тестирования не были зафиксированы системой тестирования. Этим можно объяснить колебания задержки, измеряемой на получателе. Кроме того, получатель не может формально отличить некорректные сообщения первых итераций тестирования, таким образом результаты, измеряемые на узле-получателе так же имеют дополнительную ошибку в виде первых нескольких измерений для каждого теста.

\subsubsection{Задержка передачи данных}
\img{img/mira/mira_res_i_o_ml.png}{График зависимостей задержек передачи данных внутри одного процесса и между двумя процессами в зависимости от объема передаваемых данных на стороне принимающего.}{img:mira_res_i_o_ml}{\textwidth}

На рисунке \ref{img:mira_res_i_o_ml} показаны графики задержки передачи сообщений. На них четко видно, что реальная задержка передачи сообщений между модулями внутри одного процесса на несколько порядков (порядок 1000 наносекунд внутри одного процесса и 10 миллисекунд в разных) ниже, чем в различных процессах. 

\img{img/mira/mira_res_i_o_sl.png}{График зависимостей задержек передачи данных внутри одного процесса и между двумя процессами в зависимости от объема передаваемых данных на стороне посылающего.}{img:mira_res_i_o_sl}{\textwidth}

При этом, на рисунке  \ref{img:mira_res_i_o_sl} отображен график, из которого следует, что время непосредственно публикации данных на модуле-передатчике внутри одного процесса примерно в $1.5$ раза дольше, чем в разных процессах.

\subsubsection{Пропускная способность}

Пропускную способность стоит рассматривать с учетом времени, которое занимает передача сообщения от передатчика к получателю. Если рассматривать время только на передатчике, то результат будет не соответствовать реальности. 

\img{img/mira/mira_res_i_o_bps.png}{График зависимостей пропускной способности передачи данных внутри одного процесса и между двумя процессами в зависимости от объема передаваемых данных.}{img:mira_res_i_o_bps}{\textwidth}

Как видно из графика на рисунке \ref{img:mira_res_i_o_bps}, пропускная способность передачи данных внутри процесса составляет гигабайты в секунду. Фактически, в случае внутрипроцессного взаимодействия, пропускная способность ограничена лишь доступом к оперативной памяти. В случае же межпроцессной передачи данных, пропускная способность достаточно низкая, около десятков мегабайт в секунду.

\subsection{Производительность различных подходов коммуникации}

\img{img/mira/mira_res_rpc_pubsub.png}{График зависимостей задержки передачи данных для различных подходов коммуникации в зависимости от объема передаваемых данных.}{img:mira_res_rpc_pubsub}{\textwidth}
\img{img/mira/mira_res_rpc_bps.png}{График зависимости пропускной способности передачи данных при удаленном вызове процедуры от объема передаваемых данных.}{img:mira_res_rpc_bps}{\textwidth}

В MIRA, кроме шаблона \enquote{издатель-подписчик}, доступно взаимодействие по шаблону \enquote{сервис-клиент}. Сравнение будет вестись для модулей в разных процессах, поскольку чаще всего стоит задача удаленного вызова процедур у другого процесса.

График на рисунке \ref{img:mira_res_rpc_pubsub} показывает, что при удаленном вызове процедур с увеличением объема данных резко начинает возрастать время на передачу данных. Стоит отметить, что в тестах объем данных передавался в обе стороны: от клиента к сервису и обратно. Тем не менее, это не объясняет стремительный рост времени передачи данных при больших объемах данных.

График на рисунке \ref{img:mira_res_rpc_bps} показывает, что при удаленном вызове процедур виден порог пропускной способности: примерно 22 мегабайта в секунду.