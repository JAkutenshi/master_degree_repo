\subsection{Зависимость задержки от определенных факторов}
Разбирая вопрос факторов, влияющих на производительность прикладного ПО для системы ROS, предполагалась важность следующих факторов:
\begin{itemize}[noitemsep]
	\item размер буфера;
	\item количество подписчиков.
\end{itemize}
В ходе тестирования были получены данные, которые позволяют подтвердить или опровергнуть гипотезы о значимости данных факторов для производительности ROS приложений.

\subsubsection{Зависимость от буфера}
Для проверки гипотезы о том, что задержка передачи данных как-либо зависит от размера буфера топика, рассмотрим графики на рисунке \ref{img:ros_buf_l_k}, в которых отражены результаты тестирования задержки передачи данных в системе \enquote{издатель-подписчик} при единственном подписчике, но при разных объемах данных в зависимости  от размера буфера топика.
\img{img/ros/ros_buf_l_k.png}{График зависимостей задержек передачи данных при разных объемах данных в пределах мегабайта от размера буфера топика.}{img:ros_buf_l_k}{\textwidth}
Зависимость примерно линейная и задержка практически никак не зависит от размера буфера топика.

\subsubsection{Зависимость от количества подписчиков}
Для проверки гипотезы о том, что задержка передачи данных как-либо зависит от количества узлов-подписчиков на топик, рассмотрим графики на рисунках \ref{img:ros_subs_l_m} и \ref{img:ros_subs_l_k}, в которых отражены результаты тестирования задержки передачи данных в системе \enquote{издатель-подписчик} при буфере равном 1000, но при разных объемах данных в зависимости  от количества подписчиков.

\img{img/ros/ros_subs_l_k.png}{График зависимостей задержек передачи данных при разных объемах данных в пределах мегабайта от количества подписчиков.}{img:ros_subs_l_k}{\textwidth}
\img{img/ros/ros_subs_l_m.png}{График зависимостей задержек передачи данных при разных объемах данных больше мегабайта от количества подписчиков.}{img:ros_subs_l_m}{\textwidth}

Из графиков видно, что большого влияния количество подписчиков не оказывает на быстродействие системы. Однако, следует заметить высокое, пропорциональное количеству подписчиков, использование памяти. Первоначально тестирование планировалось для 32 подписчиков, но 64 Гб доступной оперативной памяти не хватало даже на 16 подписчиков при объеме сообщений 64 Мб. 


\subsection{Показатели производительности при различных объемах данных}
\subsubsection{\enquote{Издатель-подписчик}}
Разберем графики с зависимостью задержки (секунды) и пропускной способности (мегабайты в секунду) от различных объемов данных, передаваемых в системе. Для рассмотрения берется система из одного издателя, одного подписчика и буфера размера 1000.

\img{img/ros/ros_pubsub_l.png}{График зависимостей задержек передачи данных при разных объемах данных больше мегабайта от количества подписчиков.}{img:ros_pubsub_l}{\textwidth}
\img{img/ros/ros_pubsub_bw.png}{График зависимостей пропускной способности относительно публикующего и относительно принимающего от разного объемах данных.}{img:ros_pubsub_bw}{\textwidth}

На рисунке \ref{img:ros_pubsub_l} видно, что при сообщениях от 16 Мб задержка составляет больше секунды. Это для робототехнической системы является неприемлимым с точки зрения предметной области. Следовательно, ROS не рекомендуется использовать для передачи больших объемов данных, либо учитывать большую задержку при обмене информацией.

На рисунке \ref{img:ros_pubsub_bw} видно, что при сообщениях больше 16 Кб реальная пропускная способность падает из-за роста задержки передачи сообщений. Кроме того, пропускная способность относительно узла-издателя так же падает при передаче сообщений больше мегабайта.

\subsubsection{\enquote{Клиент-сервис}}
\img{img/ros/ros_rpc_l_k.png}{График зависимости времени запроса и ответа при объеме данных до 64 Кб.}{img:ros_rpc_l_k}{\textwidth}
\img{img/ros/ros_rpc_l_m.png}{График зависимости времени запроса и ответа при объеме данных от 256 Кб.}{img:ros_rpc_l_m}{\textwidth}
\img{img/ros/ros_rpc_bps.png}{График зависимости пропускной способности от объема передаваемых даных}{img:ros_rpc_bps}{\textwidth}
\img{img/ros/ros_rpc_pubsub.png}{График зависимостей задержки передачи данных от объема передаваемых даных для разных типов коммуникации в ROS}{img:ros_rpc_pubsub}{\textwidth}
Рассмотрим графики с зависимостью задержки в миллисекундах и пропускной способности в мегабайтах в секунду от различных объемов данных, передаваемых в системе. При рассмотрении клиент-сервисного подхода уникальны следующие детали:
\begin{itemize}[noitemsep]
	\item данные передаются в обе стороны: на запрос и ответ - следовательно объем передаваемых учитывается два раза в полосе пропускания;
	\item имеется возможность измерить время передачи данных в обе стороны.
\end{itemize}

На графиках \ref{img:ros_rpc_l_k} и \ref{img:ros_rpc_l_m} видно, что до 256 Кб задержка отправки ответа незначительна, но при передаче данных от 1 Мб роль задержки ответа резко возрастает.

На графике \ref{img:ros_rpc_bps} видно, что при передаче данных посредством вызова удаленных процедур есть предел полосы пропускания около 650 Мб/с.

Кроме того, интересен график на рисунке \ref{img:ros_rpc_pubsub}, из которого следует, что передача данных при помощи вызова удаленных процедур, при том, что данных фактически при \enquote{сервис-клиент} подходе передавалось в 2 раза больше, задержка у подхода \enquote{сервис-клиент} примерно в 4 раза меньше и приемлима для систем автономных роботов.





