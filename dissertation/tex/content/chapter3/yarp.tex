\subsection{Выявленные ограничения}
Несмотря на то, что изначально предполагалось использование объема данных вплоть до 64 Мб, в ходе тестирования было невозможно отправить данные от 1 Мб. Тестирование либо прерывалось с сигналом \textit{Segment Fault}, либо длилось слишком долго. Тестирование прерывалось примерно спустя 15 минут, если управление не переходило к следующему тесту. Если предположить, что тест из 10 итераций занимает 900 секунд и более, то передача данных в автономной робототехнической системе с полученной задержкой неприемлима.

Кроме того, для гарантии целостности передачи информации при использовании RPC-порта не поддерживается протокол UDP.

\subsection{Сравнение типов портов}
Для сравнения производительности различных типов портов будет использоваться протокол TCP. График на рисунке \ref{img:yarp_ports_l} показывает требуемые зависимости от объема передаваемых данных.
\img{img/yarp/yarp_ports_l.png}{График зависимостей задержек передачи данных для разных типов портов от объема передаваемых данных}{img:yarp_ports_l}{\textwidth}
\img{img/yarp/yarp_ports_bw.png}{График зависимостей пропускной способности для разных типов портов от объема передаваемых данных}{img:yarp_ports_bw}{\textwidth}

Как видно на графике, производительность порта с буфером и без практически не различается с минимальным отставанием порта без буфера сообщений.

Если рассматривать пропускную способность всех типов портов, то график на рисунке \ref{img:yarp_ports_bw} показывает, что все три типа портов имеют низкую пропускную способность, колеблющуюся между 115 и 120 килобайтами в секунду.

\subsection{Сравнение протоколов коммуникации}

Сравним производительность буферизованного порта и RPC-порта с различными доступными для них протоколами.

\img{img/yarp/yarp_protocol_buf_l.png}{График зависимостей задержек передачи данных для буферизованного порта с разными протоколами передачи данных от объема передаваемых данных}{img:yarp_protocol_buf_l}{\textwidth}
\img{img/yarp/yarp_protocol_buf_bw.png}{График зависимостей пропускной способности для буферизованного порта с разными протоколами передачи данных от объема передаваемых данных}{img:yarp_protocol_buf_bw}{\textwidth}
\img{img/yarp/yarp_protocol_rpc_l.png}{График зависимостей задержек передачи данных для RPC-порта с разными протоколами передачи данных от объема передаваемых данных}{img:yarp_protocol_rpc_l}{\textwidth}
\img{img/yarp/yarp_protocol_rpc_bw.png}{График зависимостей пропускной способности для RPC-порта с разными протоколами передачи данных от объема передаваемых данных}{img:yarp_protocol_rpc_bw}{\textwidth}

На графиках \ref{img:yarp_protocol_buf_l} и \ref{img:yarp_protocol_buf_bw} изображены показатели производительности различных протоколов для буферизованного порта. Учитывая, что протокол UDP не гарантирует достижение сообщения получателем и реальная пропускная способность у данного протокола ниже из-за потерь дэйтаграмм, то наиболее стабильным с точки зрения пропускной способности и эффективным по времени передачи данных является использование разделяемой памяти, за что в YARP отвечает фреймворк ACE.

На графиках \ref{img:yarp_protocol_rpc_l} и \ref{img:yarp_protocol_rpc_bw} изображены показатели производительности различных протоколов для RPC-порта. Протоколы TCP и FastTCP показывают одинаково низкие результаты производительности и, как и в случае с буферизованным портом, использование разделяемой памяти предоставляет наибольшую производительность коммуникации между узлами робототехнической системы.





