Сравниваться между собой будут следующие конфигурации фреймворков:
\begin{itemize}
	\item ROS \enquote{издатель-подписчик} с одним подписчиком и буфером на 1000 сообщений;
	\item MIRA каналы в разных процессах;
	\item буферизованные порты YARP с протоколом TCP.
\end{itemize}

Данные конфигурации выбраны поскольку для различных процессов используется так или иначе протокол TCP. Кроме того, общей является возможность вызова удаленных процедур.

\img{img/comp/comp_tcp_l.png}{График зависимостей задержек передачи данных от одного узла другому для различных фреймворков от объема передаваемых данных}{img:comp_tcp_l}{\textwidth}
\img{img/comp/comp_rpc_l.png}{График зависимостей задержек передачи данных при удаленном вызове процедур для различных фреймворков от объема передаваемых данных}{img:comp_rpc_l}{\textwidth}

Согласно графикам \ref{img:comp_tcp_l} и \ref{img:comp_rpc_l} MIRA даже при использовании различных процессов имеет наименьшую задержку при передачи сообщений между узлами системы. YARP даже с учетом неполных данных из-за невозможности передать данные больше 256 Кб, на обозреваемых результатах показывает худшие результаты. ROS показывает средние результаты, задержка передачи сообщений между двумя узлами при больших объемах данных резко возрастает. Кроме того стоит заметить, что ROS затрачивает гораздо большие по-сравнению с MIRA ресурсы оперативной памяти.
