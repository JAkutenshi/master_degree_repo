Для обсуждения конкретных решений требуется из всего множества существующих фреймворков выбрать наиболее подходящие для данного исследования. Ниже приведены использовавшиеся критерии.

\begin{description}[noitemsep]
	\item [Наличие открытого исходного кода.] Для лучшего понимания работы фреймворка, а так же лучшего понимания результатов тестирования и возможного улучшения тестовых задач желательно иметь возможность прочитать исходный код реализации функционала, влияющего на результат тестирования. Это может быть, например, реализация коммуникации между приложениями, которые были написаны при помощи исследуемого фреймворка.
	
	\item [Наличие документации.] Написание обоснованных и корректных тестовых задач под конкретный фреймворк очень затруднительно, если отсутствуют инструкций по его использованию, если отсутствует подробное описание API. Без наличия документации разработка приложений становится слишком сложной и корректность их выполнения не гарантируется.
	
	\item [Текущий статус разработки.] Проекты, которые больше не поддерживаются разработчиками вполне могут рассматриваться для исследования, но в них скорее-всего используются устаревшие подходы, что приведет к низким показателям производительности.
	
	\item[Архитектура фреймворка.] Согласно требованиям к робототехническим системам, описанным в разделе \ref{title:chapter1:robo_middleware_definition}, требуется как минимум распределенная архитектура. Чем ближе архитектура будет к P2P (peer-to-peer, одноранговая сеть), тем надежнее будет все робототехническое ПО. Для данного исследования будут рассматриваться распределенное гибридное и близкое к чистым P2P архитектурам \marm{}.
	
	\item [Наличие инструментов для мониторинга и конфигурации системы.] Для анализа ППО и обслуживаемых им приложений требуются такие инструменты, как мониторы используемых ресурсов и системы журналирования. Кроме того, развертывание большого распределенного ПО для проведения тестов является трудоемкой задачей и крайне желательны инструменты конфигурации и развертывания системы на целевой ОС и аппаратном обеспечении.
	
	\item [Поддержка различных языков программирования.] Несмотря на то, что в данной работе наибольший интерес представляет именно промежуточный уровень системы, наличие альтернатив между различными языками программирования прикладного слоя робототехнической системы является преимуществом, поскольку позволяет в зависимости от задачи выбрать между, например, низкоуровневым программированием с возможным преимуществом в производительности и высокоуровневыми языками с наличием удобных для разработки интерфейсов и прикладных библиотек.
	
\end{description}

В данной работе не важны критерии:
\begin{itemize}[noitemsep]
	\item поддержки ограничений системы реального времени, поскольку это относится не к скорости работы, а к предсказуемости системы. Наличие данного пункта является преимуществом в целом, но относительно производительности оказывается не существенным;
	\item поддержки конкретных операционных систем, поскольку рассматривается вопрос производительности слоя абстракции между, собственно, ОС и прикладным ПО;
	\item наличия инструментов симуляции, графических пользовательских интерфейсов и набора прикладных библиотек с реализациями наиболее распространенных алгоритмов, поскольку это относится к функциональному уровню системы, ближе к прикладному. Вопросы производительности отдельных инструментов, которые могут требоваться при разработке, отладке и администрированию робототехнических систем не относятся к проблематике потребления ресурсов на поддержание каркаса, основы системы - фреймворка.
\end{itemize}
