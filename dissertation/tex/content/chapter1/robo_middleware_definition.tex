Под ППО понимается набор ПО, находящегося по уровню абстракции между ОС (операционными системами) и прикладными приложениями, предназначенного для управления неоднородностью аппаратного обеспечения с целью упрощения и снижения стоимости разработки ПО. В состав ППО входят:
\begin{itemize}
	\item ПО для осуществления коммуникации между прикладными приложениями, использующими данное ППО, между предоставляемыми компонентами промежуточного слоя данного ППО;
	\item наборы программных библиотек и их API (application programming interface) для разработки прикладный приложений;
	\item инструменты для разработки прикладных программ, среди которых возможны собственные компиляторы, менеджеры проектов, системы построения проектов;
	\item утилиты управления и мониторинга системы, состоящей из ППО и прикладных программ, использующий инфраструктуру ППО.
\end{itemize}

Особенностями робототехнических систем являются:
\begin{itemize}[noitemsep]
	\item c точки зрения инфраструктуры:
	\begin{itemize}[noitemsep]
		\item модульность;
		\item упор на параллельность выполнения процессов системы и распределенность компонентов по различным процессам;
		\item надежность и отказоустойчивость;
		\item стремление к предсказуемости и соответствию системам реального времени.
	\end{itemize}
	\item c точки зрения прикладного программирования:
	\begin{itemize}[noitemsep]
		\item большой объем программных библиотек, реализующих алгоритмы, важные в робототехнике, таких как кинематика, компьютерное зрение и распознавание объектов, локализация построение карты окружения \etc;
		\item большой объем предоставляемых драйверов и программных библиотек, обслуживающих внешние устройства, различную аппаратуру;
		\item предоставление ПО для симуляции робототехнических моделей.
	\end{itemize}
\end{itemize}

Особенности и требования к инфраструктуре робототехнических систем приводят к выводу, что наиболее выгодным является использование распределенной архитектуры и многоагентного подхода. В статье \cite{blasco2012multiagent} приводятся доводы и рассуждения о преимуществах использования многоагентных фреймворков для разработки робототехнического ППО, а так же примеры многоагентных фреймворков, которые могут использоваться. 

Таким образом, многоагентный подход и \marm{}  являются наиболее предпочтительными в разработке автономных робототехнических систем и именно их рассмотрение будет в данной работе.