\begin{description}[noitemsep]
	\item [ROS] одно из наиболее распространенных \marm{}, имеется и обширная документация, и открытый исходный код, разработка продолжается, широко используется. Имеет гибридную архитектуру, средства мониторинга и автоматизации развертывания системы. Для разработки по-умолчанию есть возможность использовать C++ и Python 2.7 \cite{ros-main-site}.

	\item[MIRA] этот проект активно разрабатывается, имеется открытый исходный код и обширная документация. Имеет децентрализованную архитектуру \cite{einhorn2012mira}, реализован на C++. Имеются возможности для ведения журналирования приложений, мониторы состояния коммуникаций и ресурсов системы. Для разработки предлагается использовать C++, Python и JavaScript \cite{mira-main-site}.

	\item [MOOS] развивающийся проект, имеется документация и исходный код. На данный момент разрабатывается бета-версия MOOS 10. Общая проблема: клиент-серверная архитектура, которая является спорным решением для разработки робототехнических систем из-за проблем устойчивости ПО к ошибкам. По этой причине в данной работе этот фреймворк рассматриваться не будет \cite{moos-main-site}.
		
	\item [ORoCoS Toolchain и Rock] являются очень распространенными фреймворками на основе ORoCoS RTT (Real-Time Toolkit)и OCL (Orocos Component Library) -- одного из немногих поддерживающих ограничения реального времени ППО вместе с широко использующимися библиотеками кинематики и динамики, Байесовских фильтров \etc. Документация обширная, но из-за удаления сервиса хранения удаленных репозиториев gitorious.org \cite{gitorious-valhala} как исходный код, так и документация разбросаны по разным ресурсам и не всегда актуальны. Стабильная разработка в основном ведется над набором инструментов Rock. Используется гибридная децентрализованная архитектура. Разработка ведется на C++, для разработки прикладных программ предоставляются такие языки, как C++, Python, Simulink \cite{blasco2012multiagent}. Имеются инструменты для развертывания и мониторинга системы \cite{orocos-toolchain-main-site,rock-main-site}. Из-за технических проблем исходного проекта непосредственное тестирование ORoCoS Toolchain в данной работе производиться не будет, но будут разобраны методы коммуникации.
	
	\item [ASEBA] является проектом, в котором для программирования прикладного слоя используется концепция языков программирования пятого поколения: GUI, блоки, коннекторы. Разработка является скорее обучающим продуктом с коммерческой составляющей в виде конкретной модели робота thymio. По этой причине рассматриваться в данной работе не будет \cite{aseba-main-site}.
	
	\item [SmartSoft] разрабатывающийся проект с обширной документацией. Для реализации компонентов используется C++. Практически не используется децентрализация, основной шаблон взаимодействия клиент-сервер. Кроме того, используется многопоточный подход, а не многопроцессный \cite{smartsoft-main-site}, что ставит под вопрос общую устойчивость всей системы. В данной работе рассматриваться не будет.
	
	\item [YARP] активно разрабатывающееся \marm{} с открытым исходным кодом. Имеется обширная и подробная документация. Является одним из немногих практически полностью децентрализованных технологий робототехнического ППО, кроме того, поддерживает ограничения систем реального времени. Разрабатывается в основном на C++ и поддерживает такие языки как C++, Python, Java, Octave. Инструменты мониторинга и развертывания приложений имеются \cite{yarp-main-site}.
	
	\item [OpenRTM-aist] распространенное \marm{} с открытым исходным кодом, является реализацией стандарта RT-middleware \cite{openrtmaist-old-site}. Основная проблема: часть документации, при скромном содержании, на японском языке, а на момент написания работы (апрель -- май 2018 года) документация отсутствует ввиду переноса проекта на инфраструктуру GitHub, в связи с этим доступ к старому сайту был убран, по тому же URL (Uniform Resource Locator) находится временная страница проекта. На данный момент разработка не имеет доступной документации \cite{openrtmaist-new-site}, авторы на запрос документации не ответили. Архитектура гибридная децентрализованная, имеются инструменты мониторинга, журналирования и развертывания, языки разработки: C++, Java, Python. Рассматриваться не будет по причине отсутствия доступа к документации.
	
	\item [URBI] данный проект имеет много недостатков: приостановленная разработка, о чем свидетельствует последнее изменение от 2014 года \cite{urbi-repo} и не работающий сайт самого проекта \cite{urbi-main-site}, централизованная архитектура. Рассматриваться в данной работе не будет. 
\end{description}

В таблицах \ref{table:chapter1:mars_solutions_1} и \ref{table:chapter1:mars_solutions_2} отображено соответствие найденного \marm{} предложенным выше критериям.

Таким образом, в исследовании будут учавствовать следующее \marm{}:
\begin{itemize}[noitemsep]
	\item ROS;
	\item MIRA;
%	\item OROCOS/Rock;
	\item YARP.
\end{itemize}

\begin{table*}[h!]
%	\scriptsize
	\centering
	\def\arraystretch{1.3}
%	\resizebox{\textwidth}{!}{
		\begin{tabular}{|l|p{3cm}|p{4.6cm}|p{3.6cm}|}
			\hline
			\textbf{НАЗВАНИЕ}     & \textbf{ОТКРЫТЫЙ КОД} & \textbf{НАЛИЧИЕ ПОНЯТНОЙ ДОКУМЕНТАЦИИ} & \textbf{ПОСЛЕДНИЕ ИЗМЕНЕНИЯ} \\ \hline
			ROS          & Да           & Да                            & Недавно \\ \hline
			MIRA         & Да           & Да                            & Недавно \\ \hline
			MOOS         & Да           & Да                            & Недавно \\ \hline
			OROCOS/Rock  & Да           & Да                            & 2016    \\ \hline
			ASEBA        & Да           & Нет                           & Недавно \\ \hline
			YARP         & Да           & Да                            & Недавно \\ \hline
			OpenRTM-aist & Да           & Нет                           & 2016    \\ \hline
			URBI         & Да           & Нет                           & 2016    \\ \hline
		\end{tabular}
%	}
	\caption{Соответствие найденного робототехнического ППО выделенным в разделе \ref{title:chapter1:mars_criterias} критериям (часть 1)}
	\label{table:chapter1:mars_solutions_1}
\end{table*}

\begin{table*}[h!]
	%\small
	\centering
	\def\arraystretch{1.3}
	\resizebox{\textwidth}{!}{%
		\begin{tabular}{|l|l|p{4.2cm}|p{3.8cm}|}
			\hline
			\textbf{НАЗВАНИЕ}     & \textbf{АРХИТЕКТУРА}   & \textbf{ИНСТРУМЕНТЫ МОНИТОРИНГА} & \textbf{ПОДДЕРЖКА ЯП} \\ \hline
			ROS                   & Гибридная              & Да                      & C++, Python               \\ \hline
			MIRA                  & Децентрализованная     & Да                      & C++, Python, JavaScript   \\ \hline
			MOOS                  & Централизованная       & Да                      & C++, Java                 \\ \hline
			OROCOS/Rock           & Гибридная              & Да                      & C++, Python, Simulink     \\ \hline
			ASEBA                 & Распределенная         & Да                      & Собственный язык          \\ \hline
			SmartSoft             & Распределенная         & Да                      & C++                       \\ \hline
			YARP                  & Децентрализованная     & Да                      & C++, Python, Java, Octave \\ \hline
			OpenRTM-aist          & Гибридная              & Да                      & C++, Java, Python         \\ \hline
			URBI                  & Централизованная       & Да                      & C++, Java, urbiscript     \\ \hline
		\end{tabular}
	}
	\caption{Соответствие найденного робототехнического ППО выделенным в разделе \ref{title:chapter1:mars_criterias} критериям (часть 2)}
	\label{table:chapter1:mars_solutions_2}
\end{table*}
