Для тестирования производительности \marm{} будет требоваться доступ на уровне вызова функций из кода программ. В таком случае наиболее подходящим решением будет фреймворк для написания бенчмарков -- это программы в изначальном значении (синтетические бенчмарки \cite{curnow1976synthetic}) для измерения скорости работы процессора. В более общем и подходящем в данном контексте значении -- программы для измерения каких-либо количественных характеристик быстродействия системы.

Все \marm{} имеют возможность писать прикладные программы на C++ - эффективном в контексте быстродействия написанных на нем программ. Следовательно, требующийся бенчмарк-фреймворк должен иметь программные интерфейсы для измерения как минимум времени работы блока операторов, написанных на языке C++. 

Вышеописанные критерии являются базовыми для выбора инструмента проведения тестирования производительности \marm{}. Для того, чтобы из возможных вариантов выбрать какое-либо подмножество наиболее подходящих, возможны дополнительные критерии для сравнения:

\begin{itemize}[noitemsep]
	\item наличие возможности форматированного вывода результата;
	\item наличие возможности создавать пользовательские метрики;
	\item наличие возможности автоматически выполнять блок кода множество раз;
	\item наличие понятной документации;
	\item наличие концепции фикстур;
	\item способ вычисления времени;
	\item количество зависимостей;
\end{itemize}

Таким образом, были выделены необходимые критерии для отбора фреймворков для написания бенчмарков на языке C++, а так же приведен список дополнительных критериев для сравнения бенчмарк-фреймворков между собой.

