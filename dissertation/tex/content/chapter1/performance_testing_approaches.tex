Целью тестирования производительности является демонстрация соответствия системы заявленным требованиям производительности относительно приемлимых показателей времени отклика, требующегося на обработку определенного объема данных \cite{dustin1999automated}.

Ниже приведены четыре вида, на которые условно можно разделить тестирование производительности.
\begin{description}[noitemsep]
	\item [Нагрузочное тестирование] проводится с целью узнать быстродействие системы при планируемой для системы нагрузке.
	\item [Стресс-тестирование] проводится с целью узнать нагрузку, при которых система перестает удовлетворять заявленным требованиям.
	\item [Конфигурационное тестирование] проводится с целью определить быстродействие при различных конфигурациях системы.
	\item [Тестирование стабильности] проводится с целью определить быстродействие и корректность работы системы при длительной постоянной нагрузке.
\end{description}

Стоит отметить, что приведенное выше разделение является условным: стресс-тестирование является скорее подмножеством нагрузочного, четкой граница между ними нет. Кроме того, при тестировании производительности может использоваться комбинация различных видов тестирования производительности, так как, например, тестирование стабильности можно проводить вместе с нагрузочным.

Ниже приведены метрики, которые могут быть получены в ходе тестирования производительности.
\begin{description}[noitemsep]
	\item [Задержка] -- время выполнения какой-либо операции. Задержка бывает нескольких видов в зависимости от способа измерения:
	\begin{itemize}[noitemsep]
		\item задержка, получаемая как разность тактов ядра процессора до начала и после завершения измеряемого кода пропорционально установленной тактовой частоте процессора;
		\item задержка реального времени, получаемая как разность значения системных часов до начала и после завершения измеряемого кода.
	\end{itemize}
	Разница данных подходов видна в случае обращения к внешним устройствам (устройствам ввода-вывода, вычислительным устройствам, таким как GPU (graphics processing unit)), в случае потери контроля ядра процессора потоком, в котором выполняется измерение (например, в случае перевода потока в состояние ожидания). В случае, если не гарантировано выполнение кода на одном CPU (Central Processing Unit), измерение задержки выполнения данного кода при помощи тактов процессора может быть некорректным.
	\item [IOPS (Input/Output Per Second)] -- скорость обращения к устройству или системам хранения данных, измеряемая в количестве блоков, которое можно записать или прочитать с устройства в единицу времени. 
	\item [Широта пропускания] - объем информации, которое может обрабатываться системой в единицу времени.
\end{description}

В зависимости от предмета тестирования будут различны методы тестирования. В самом простом случае тестирование производительности представляет из себя запуск исследуемой системы и ручное измерение, сбор и анализ характеристик системы. Данный подход имеет следующие недостатки: низкая точность измерения, низкая скорость выполнения тестирования, высокая сложность изменения тестовых сценариев, сложность с вычислением метрик отдельных блоков кода.

Более эффективным способом тестирования производительности является написание и исполнение приложений, которые имеют доступ к интересующему компоненту системы по требуемому интерфейсу:
\begin{itemize}[noitemsep]
	\item вызовов функций на уровне исходного кода рассматриваемого приложения;
	\item запуска программных модулей на уровне ОС;
	\item потоков информации на уровне протоколов коммуникации.
\end{itemize}

Существует множество решений, позволяющих конфигурировать автоматизированные сценарии тестирования, нагрузку на систему, структурировать и визуализировать результаты тестирования.

Таким образом, возникает проблема выбора инструмента для тестирования производительности \marm{};