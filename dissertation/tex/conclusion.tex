\chapter*{Заключение}
\addcontentsline{toc}{chapter}{Заключение}

В ходе работы над ВКР было проведено исследование такой предметной области, как промежуточное ПО для разработки прикладных систем автономных роботов. В ходе обзора предметной области были отобраны для подробного рассмотрения четыре из них: ROS, YARP, MIRA и ORoCoS Toolchain. Из-за технических проблем, связанных с сопровождением проекта ORoCoS Toolchain, для последнего не реализовывались тестовые случаи.

В ходе составления плана тестирования были подробно рассмотрены различные аспекты многоагентных фреймворков и промежуточного ПО для автономных роботов, как следствие был составлен список факторов, которые гипотетически могли влиять на производительность всей робототехнической системы. Главным общим фактором является система межпроцессного взаимодействия, коммуникации.

В ходе рассмотрения технических решений и особенностей каждого из рассматриваемых \marm{} были выявлены особенности реализации, а так же возможные факторы влияния на производительность робототехнической системы. Выявив факторы, были построены тестовые случаи для тестирования производительности \marm{}.

В ходе реализации тестовых случаев было реализовано изолированное единообразное окружение при помощи системы виртуализации Docker. Реализация тестовых случаев для ROS и YARP использовала Google Benchmark фреймворк, но особенность из-за особенностей MIRA пришлось реализовать собственную библиотеку для реализации тестовых случаев. Поскольку в ходе тестирования получается множество сложных для восприятия файлов с результатами, потребовалось написать приложение для автоматизации анализа результатов тестирования. В качестве результата работы анализатора был выбран отчет в формате rmarkdown с отображением исходного кода на языке R. Это позволило сделать ряд первоначальных выводов о результатах тестирования, а так же построить информативные графики, иллюстрирующие различные тезисы о производительности рассматриваемых \marm{}.

В ходе анализа были определены наиболее сильные и слабые стороны каждого фреймворка в частности, а так же проведено сравнение между всеми рассматриваемыми \marm{}.

В ходе выполнения работы над дополнительным разделом ВКР был предложен бизнес-план по коммерциализации анализатора результатов тестирования.

У данной работы есть несколько вариантов развития:
\begin{itemize}
	\item рассмотрение большего числа робототехнических фреймворков;
	\item улучшение инструментария тестирования производительности:
	\begin{itemize}
		\item добавление функционала для библиотеки бенчмарк-макросов:
		\begin{itemize}
			\item возможность добавлять и статистически обрабатывать пользовательские показатели производительности;
			\item автоматическое подстраивание количества итераций под заданную точность измерений;
			\item возможность выделять тестовые случаи в отдельные методы.
		\end{itemize}
		\item улучшение инструмента анализа результатов:
		\begin{itemize}
			\item добавить GUI;
			\item разработать архитектуру приложения с учетом большой вариативности способов отображения результатов тестирования;
			\item реализовать интеграцию с интерпретатором языка R.
		\end{itemize}
	\end{itemize}
\end{itemize}
