\chapter*{Определения, обозначения и сокращения}
В настоящей пояснительной записке применяют следующие термины с соответствующими определениями:

\termDef{ОС}{операционная система.}
\termDef{ПО}{программное обеспечение.}
\termDef{ППО}{промежуточное программное обеспечение.}
\termDef{\marm{}}{многоагентное робототехническое программное обеспечение.}
\termDef{API}{Application Programming Interface, интерфейс прикладного программирования.}
\termDef{P2P}{Peer-To-Peer, одноранговая сеть.}
\termDef{URL}{Uniform Resource Locator, единый указатель ресурса.}
\termDef{RTT}{Real-Time Toolkit.}
\termDef{OCL}{ORoCoS Component Library.}
\termDef{GPU}{Graphics Processing Unit, графический процессор.}
\termDef{CPU}{Central Processing Unit, центральный процессор.}
\termDef{ORB}{Object Request Brokers, брокеры объектных запросов.}
\termDef{QoS}{Quality of Service, предоставление различному трафику различный приоритет в обслуживании.}
\termDef{XML}{eXtensible Markup Language, язык разметки.}
\termDef{JSON}{JavaScript Object Notation, текстовый формат обмена данными.}
\termDef{DNS}{Domain Name Service, система для получения информации о доменах в сети.}
\termDef{GUI}{Graphical User Interface, графический пользовательский интерфейс.}