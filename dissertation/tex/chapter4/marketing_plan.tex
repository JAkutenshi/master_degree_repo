\subsection{План продаж}
	Как показывают тренды последних лет, то ПО все чаще не продается \enquote{как есть}, а предоставляется по платной подписке за гораздо меньшую по сравнению с себестоимостью всего продукта цену. 
	
	Поскольку сам проект небольшой, то идея распространения по очень дешевой платной подписке с большим объемом продаж, взамен на которую пользователи получают постоянные обновления и поддержку по продукту - потенциально прибыльный план продажи нашего продукта.
	
	Вычислим безубыточную цену подписки в месяц: за месяц на разработку тратится 58328 руб (из раздела \ref{subseq:production_plan_results}), а минимальный объем ожидаемых клиентов по истечению третьего месяца 2000 пользователей (вывод раздела \ref{subseq:demand_for_product}). Следовательно безубыточная цена, представленная покупателям в начале второго месяца выхода продукта на рынок:
	\[
		\dfrac{58328}{2000} = 29.17 руб
	\]
	
	С учетом, что равных по функционалу конкурентов на рынке нет, то цена в \$1 на иностранном рынке и 60 руб. на отечественном в 2 раза превышает себестоимость при этом являясь очень низкой в сравнении с затратами на ручной анализ результатов тестирования.
	
	В таблице \ref{table:trading_plan} представлен предполагаемый объем продаж на 2018 год.
		
	\begin{table}[!h]
		\def\arraystretch{1.3}
		\centering	
		\caption{План продаж на 2018 год}
		\label{table:trading_plan}
		\begin{tabular}{|p{3cm}|r|r|r|r|r|}
			\hline
			\multicolumn{1}{|c|}{\multirow{2}{*}{Показатели}} & \multicolumn{4}{c|}{Квартал} & \multicolumn{1}{c|}{\multirow{2}{*}{Всего}} \\ \cline{2-5}
			\multicolumn{1}{|c|}{} & \multicolumn{1}{c|}{II} & \multicolumn{1}{c|}{III} & \multicolumn{1}{c|}{IV} & \multicolumn{1}{c|}{IV} & \multicolumn{1}{c|}{} \\ \hline
			Ожидаемый объем продаж & 2000 & 2500 & 3000 & 3500 & 11000 \\ \hline
			Цена с НДС & 180 & 180 & 180 & 180 & 720 \\ \hline
			Нетто выручка (без НДС) & 360000 & 450000 & 540000 & 630000 & 1980000 \\ \hline
			Выручка с НДС (18\%) & 424800 & 531000 & 637200 & 743400 & 2336400 \\ \hline
			Сумма НДС & 64800 & 81000 & 97200 & 113400 & 356400 \\ \hline
		\end{tabular}
	\end{table}
	
\subsection{Стратегия маркетинга}
	Таким образом, ставка делается на качественный и узконаправленный продукт с длительным жизненным циклом. Ставка в ценообразовании делается на большой объем сбыта с низкой ценой, а так же экономии на затратах.
	
	В частности, распространение будет производиться через бесплатный хостинг OpenShift от RedHat. Для нашего не такого уж и большого проекта вычислительных мощностей более чем достаточно: исходя из цифр раздела \ref{subseq:demand_for_product} база данных размером в 400,000 успешно размещается в этом сервисе, не требуя за это плату. Проблема с доменным именем решается путем переадресации со страницы OpenSourse версии разработки на GitHub. В итоге получатель будет иметь доступ к понятному доменному имени cppbenchmark.github.io, в которой уже удобный процесс получения платной подписки будет перенаправлен на веб-приложение, размещенное на openshift.com на бесплатной основе. 
	
	Говоря о рекламе, поскольку у нас имеется версия с открытым исходным кодом, то она будет рекламироваться на целевых для программистов площадках: linux.org.ru, habrahabr.ru, opennet.ru, reddit.com/unix и т.д. Написание рекламных тем не требует никакой оплаты, лишь затрат времени, учтенного в таблице \ref{table:lifecicle}, как последний день подготовки к релизу. Остальное сообщество сделает самостоятельно. Т.о., предоставив открытый код мы расширили круг заинтересованных лиц и получили сильную рекламу своего продукта.