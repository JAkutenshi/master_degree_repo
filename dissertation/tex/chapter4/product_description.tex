\subsection{Наименование продукта}
	Продуктом является программное средство для реализации и анализа результатов тестов производительности. Определение \enquote{программа} и \enquote{программный модуль} регулируется ГОСТ 19781-90 и формулируется следующим образом:
	\begin{description}[noitemsep]
		\item [Программа] - данные, предназначенные для управления конкретными компонентами системы обработки информации в целях реализации определенного алгоритма.
		\item [Программный модуль] - программа или функционально завершенный фрагмент программы, предназначенный для хранения, трансляции, объединения с другими программными модулями и загрузки в оперативную память.
	\end{description}
%
\subsection{Назначение продукта} \label{subseq:products_purpose}
	Продукт разработан для разработчиков ПО, которым требуется:
	\begin{itemize}[noitemsep]
		\item выполнить сравнение производительности разработанного и конкурирующих решений;
		\item определить время исполнения критически важных модулей разработанного ПО,
		\item установить предельную нагрузку на разработанную систему, для доказательства удовлетворения нефункциональных потребностей заказчика.
	\end{itemize}
	Разработанный программный продукт предоставляет библиотеку Google Benchmark с API для реализации тестов производительности, а так же программу-анализатор, обрабатывающую результаты тестирования. Анализатор имеет следующие особенности:
	\begin{itemize}[noitemsep]
		\item простота реализации: любой разработчик может изменить обработку данных под требования конкретной задачи;
		\item анализ формата json, как наиболее простого для понимания и обработки формата данных;
		\item выдача результата в виде текстового файла rmarkdown, что позволяет сохранить отчет в удобный для систем контроля версий текстовый файл и, при надобности, преобразовать отчет в любой из форматов: html, pdf, \LaTeX;
		\item генерация графиков по всем факторам анализа производительности;
		\item конфигурируемость входных данных, возможность выбирать выборки входных данных и факторы для анализа результатов;
		\item конфигурируемость отчета, возможность выбирать информацию, отображаемую на графиках;
		\item конфигурация выполняется путем редактирования простых для понимания xml файлов.
	\end{itemize}

\subsubsection{Основные характеристики продукции}
	Оценка и характеристика качества ПО регулируется ГОСТ Р ИСО/МЭК 9126-93.
	\begin{description}[noitemsep]
		\item[Функциональные возможности.] 
			В данный момент в анализаторе реализованы все заявленные выше, в подразделе \ref{subseq:products_purpose}, функции.
		\item[Надежность.] 
			На данный момент приложение тестировалось на результатах измерения производительности \marm{}. При соблюдении требований к формату исходных данных, результат работы разработанной программы остается стабильным. тем не менее, анализатор не устойчив к ошибкам пользователя: ошибкам в формате результатов тестов или конфигурационных файлах.
		\item[Практичность.]
			Разработанное решение является практичным засчет высокой степени автоматизации процессов анализа и получения наглядного результата. Для выполнения анализа результатов тестирования производительности требуется заполнить два xml файла для передачи информации об измеряемых факторах, об анализируемых тестах, о требующихся для построения графиках.
		\item[Эффективность.]
			Анализатор не является самым эффективным (под эффективностью понимается скорость работы и ресурсоемкость ПО) решением, но при этом занимает удобную нишу на рынке: прямых конкурентов у данной программы нет, так как в большинстве случаев для тестов столь низкого уровня анализируют результаты вручную, либо при помощи разработанных внутри предприятия инструментов. Для анализатора результатов важна точность результатов и удобные интерфейсы для разработки тестов. По сравнению с выполнением самих тестов производительности, время работы анализатора не значительно.
		\item[Сопровождаемость.]
			Исходный код подробно документирован. Так же, реализация на языке Python позволяет проще писать самодокументиуемый код, т.е. код, который понятен в большинстве случаев без излишних комментариев со стороны разработчика. Так же была написана инструкция по использованию данного программного решения.
		\item[Мобильность.]
			Анализатор разработан на языке Python для интерпретатора Python3.7 и данную работу можно запустить на любой ОС с наличием соответствующего интерпретатора.
	\end{description}

\subsection{Потребительские свойства}
	В ходе опроса разработчиков распределенных систем было установлено, что в ходе тестирования производительности для анализа результатов тестирования производительности в большинстве случаев используется занесение данных в Excel таблицы и построение результатов в соответствующем офисном пакете. У данного подхода имеется ряд недостатков, так как каждый раз для анализа либо придется переделывать старые таблицы, либо создавать новые. В данной работе предоставляется подход к тестированию производительности от составления плана тестирования до получения конечного результата в виде таблиц и графиков.
	
	Анализатор прост в использовании, для выполнения анализа результатов требуется подготовка двух конфигурационных файлов. Далее всю работу выполняет реализованная программа.
	
	Кроме того, автоматизация работы и параллельность выполнения анализа позволяет свободным от этой рутины разработчикам выполнять свои должностные обязанности, например, работать над тестированием нескольких разработок одновременно.

\subsection{Основные конкурентные преимущества продукции}
	\begin{description}[noitemsep]
		\item[Способ распространения.]
			Распространение работы происходит через интернет, имея возможность загрузить исходный код, документацию, обоснование работы, описание, а так же примеры реализации тестирования производительности для \marm{}. 
			
		\item[Дальнейшее сопросвождение ПО и поддержка клиентов.]
			Инструменты анализа производительности будут улучшаться в дальнейшем: был получен ряд отзывов и предложений о  желаемом функционале, который планируется добавить в последующей разработке. Как пример: увеличение гибкости и настраиваемости графиков, возможность сравнения различных тестов по отдельным факторам.
			
			Кроме того, предоставляется поддержка клиентов при возникновении трудностей при работе с разработкой или при обнаружении ошибок в реализации.
			
		\item[Простота использования.]
			Для использования анализатора на практике в общем случае не требуется знать специальных знаний из области программирования. Тем не менее, если потребуется внести коррективы в алгоритм обработки данных, то разработчик это может сделать не затратив много времени из-за простоты языка программирования и открытого исходного кода.
		
		\item[Предоставление методики проведения тестирования производительности.]
			К исходному коду программы предоставляется описание и примеры реализации тестов производительности \marm{}. Разработчики, выбравшие предоставленное в данной работе решение могут использовать испробованную на практике методику тестирования различных распределенных систем, учитывая решения возникавших проблем.
			
		\item[Использование промежуточных файлов вычислений в формате json]
			Анализатор сохраняет промежуточные результаты различных этапов анализа в отдельных файлах в формате json, простом для обработки как с программной точки зрения, так и с точки зрения удобства чтения для человека. Промежуточные вычисления можно использовать для расширения возможностей анализатора и уточнения результатов тестов.
	\end{description}

\subsection{Основные потребители и направления использования продукции}
	Анализатор будет востребован для разработчиков и тестировщиков ПО, которым требуется проводить анализ разрабатываемых программных решений с точки зрения улучшения качества ПО относительно собственной разработки, так и относительно возможных конкурентов. На основе сравнительных данных разработчики будут иметь возможность предоставлять конструктивные аргументы технического характера для обоснования принятия решений при согласовании ПО с заказчиком, при разработке и анализе качества ПО, а так же может использоваться в качетсве одного из инструментов маркетинга, как демонстрация преимущества перед конкурентами.
	
	Основными потребителями являются разработчики ПО на языке C++, специализирующиеся на тестировании и анализе качества ПО.

\subsection{Структура выпуска продукции}
	Плагин будет выпускаться в двух версиях:
	\begin{enumerate}[noitemsep]
		\item 
			Бесплатная версия с базовым функционалом с открытым исходным кодом под лицензией BSD. Эта версия нужна по причинам:
			\begin{itemize}[noitemsep]
				\item Открытый исходный код позволит снизить нагрузку на поддержку, т.к. заинтересованные в новом функционале разработчики смогут дописать его сами. Кроме того, эти наработки можно использовать в дальнейшем для улучшения функционала платной версии продукта.
				\item Бесплатная версия часто воспринимается как демо-версия продукта и является частью стратегии маркетинга.
			\end{itemize}
		\item Платная версия, распространяемая по подписке с полным функционалом, c документацией и примерами полного цикла выполнения тестирования производительности, поддержкой пользователей и дальнейшее сопровождение анализатора будет продолжаться только для платной версии.
	\end{enumerate}