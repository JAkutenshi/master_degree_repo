Проведем анализ рисков \textit{сценарным} способом:
\begin{description}
	\item[Пессимистичный прогноз,] когда на количество подписчиков в 2000 пользователей придется выходить не 1 квартал, а весь год, начиная с 500 в первом квартале. В таком случае:
		\begin{itemize}
			\item ROI = 1.88
			\item T = 0.5319
			\item DPP = 2.422
			\item NPV = 108817.2257
			\item DPI = 3.1091
			\item IRR = 57\%
		\end{itemize}
	Как видно, даже в ситуации, когда рост пользователей будет в 4 раза медленнее, инвестиции окупятся уже в 3 квартале.
	\item[Реалистичный прогноз] был рассмотрен в ходе составления настоящего Бизнес-плана и в разделе \ref{subseq:performance} были получены следующие значения:
	\begin{itemize}
		\item ROI = 8.0514
		\item T = 0.1242
		\item DPP = 0.2198
		\item NPV = 713224.2435
		\item DPI = 20.3778
		\item IRR = 504\%
	\end{itemize}
	\item[Оптимистичный прогноз,] когда количество пользователей не только начинается с 2000, но и растет в 2 раза быстрее. В таком случае:
	\begin{itemize}
		\item ROI = 11,1371
		\item T = 0.0898
		\item DPP = 0.2198
		\item NPV = 978281.5546
		\item DPI = 27.9809
		\item IRR = 539\%
	\end{itemize}
\end{description}

Таким образом, даже при пессимистичном прогнозе инвестиции вернутся в течении года. Оптимистичный прогноз не сильно улучшает ситуацию, но на такой быстрый рост, с другой стороны, надеяться не приходится.