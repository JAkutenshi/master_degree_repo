\subsection{План прибылей и убытков } 
	Таблица \ref{table:profit_and_loss} представляет собой финансовый документ, в котором отражаются доходы, расходы и финансовые результаты за период реализации проекта.
	\begin{table}[!h]		
		\centering
		\def\arraystretch{1.3}
		\caption{План прибылей и убытков}
		\label{table:profit_and_loss}
		\resizebox{\textwidth}{!}{%
			\begin{tabular}{l|l|l|r|r|r|r|r|}
				\hline
				\multicolumn{3}{|c|}{Показатели} & \multicolumn{4}{c|}{Квартал} & \multicolumn{1}{c|}{\multirow{2}{*}{Всего}} \\ \cline{1-7}
				\multicolumn{1}{|l|}{\#} & \multicolumn{2}{l|}{} & \multicolumn{1}{c|}{I} & \multicolumn{1}{c|}{II} & \multicolumn{1}{c|}{III} & \multicolumn{1}{c|}{IV} & \multicolumn{1}{c|}{} \\ \hline
				\multicolumn{1}{|l|}{1} & \multicolumn{2}{l|}{Выручка-нетто (без учета НДС) от реализации} & 360000 & 450000 & 540000 & 630000 & 1980000 \\ \hline
				\multicolumn{1}{|l|}{2} & \multicolumn{2}{l|}{Переменные производственные затраты} & 127680 & 127680 & 127680 & 127680 & 510720 \\ \hline
				& 1 & Переменные материальные затраты & 0 & 0 & 0 & 0 & 0 \\ \cline{2-8} 
				& 2 & Переменные затраты на оплату труда & 127680 & 127680 & 127680 & 127680 & 510720 \\ \cline{2-8} 
				& 3 & Переменные общепроизводственные затраты & 0 & 0 & 0 & 0 & 0 \\ \hline
				\multicolumn{1}{|l|}{3} & \multicolumn{2}{l|}{Валовая прибыль} & 223320 & 313320 & 403320 & 493320 & 1433280 \\ \hline
				\multicolumn{1}{|l|}{4} & \multicolumn{2}{l|}{Переменные управленческие и коммерческие затраты} & 0 & 0 & 0 & 0 & 0 \\ \hline
				\multicolumn{1}{|l|}{5} & \multicolumn{2}{l|}{Маржинальная прибыль} & 232320 & 322320 & 412320 & 502320 & 1469280 \\ \hline
				\multicolumn{1}{|l|}{6} & \multicolumn{2}{l|}{Постоянные затраты} & 9000 & 9000 & 9000 & 9000 & 36000 \\ \hline
				& 2 & Постоянные общепроизводственные затраты & 9000 & 9000 & 9000 & 9000 & 36000 \\ \cline{2-8} 
				& 3 & Постоянные управленческие и коммерческие затраты & 0 & 0 & 0 & 0 & 0 \\ \hline
				\multicolumn{1}{|l|}{7} & \multicolumn{2}{l|}{Прибыль от продаж} & 223320 & 313320 & 403320 & 493320 & 1433280 \\ \hline
				\multicolumn{1}{|l|}{8} & \multicolumn{2}{l|}{Прочие доходы и расходы} & 24280 & 0 & 0 & 0 & 24280 \\ \hline
				& 1 & Прочие доходы & 0 & 0 & 0 & 0 & 0 \\ \cline{2-8} 
				& 2 & Прочие расходы & 24280 & 0 & 0 & 0 & 24280 \\ \hline
				\multicolumn{1}{|l|}{9} & \multicolumn{2}{l|}{Прибыль до налогообложения} & 199040 & 313320 & 403320 & 493320 & 1409000  \\ \hline
				\multicolumn{1}{|l|}{10} & \multicolumn{2}{l|}{Налог на прибыль (20\%)} & 39808 & 62664 & 80664 & 98664 & 281800 \\ \hline
				\multicolumn{1}{|l|}{11} & \multicolumn{2}{l|}{Чистая (нераспределенная) прибыль} & 159232 & 250656 & 322656 & 394656 & 1127200 \\ \hline
				\multicolumn{1}{|l|}{12} & \multicolumn{2}{l|}{Капитализируемая прибыль} & \multicolumn{5}{l|}{} \\ \hline
				& 1 & Резервы & 35000 & 194232 & 444888 & 767544 & 1162200 \\ \cline{2-8} 
				& 2 & Реинвестиции & 0 & 0 & 0 & 0 & 0 \\ \hline
				\multicolumn{1}{|l|}{13} & \multicolumn{2}{l|}{Потребляемая прибыль} & \multicolumn{5}{l|}{} \\ \hline
				& 1 & Дивиденды & 0 & 0 & 0 & 0 & 0 \\ \cline{2-8} 
				& 2 & Прочие цели & 0 & 0 & 0 & 0 & 0 \\ \cline{2-8} 
			\end{tabular}%
		}
	\end{table}
	Как видно из таблицы \ref{table:profit_and_loss}, продажа анализатора является безубыточной даже при очень низкой цене за подписку и небольшом приросте потребителей. Основная статья затрат -- зарплата единственному разработчику.

\subsection{План движения денежных средств} 
	Таблица \ref{table:flow_of_funds} представляет собой финансовый документ, в котором отражаются денежные потоки от операционной (текущей), инвестиционной и финансовой деятельности на планируемый период.

	\begin{table}[!h]
		\centering
		\def\arraystretch{1.3}
		\caption{План движения денежных средств}
		\label{table:flow_of_funds}
		\resizebox{\textwidth}{!}{%
			\begin{tabular}{l|l|l|r|r|r|r|r|}
				\hline
				\multicolumn{3}{|c|}{Показатели} & \multicolumn{4}{c|}{Квартал} & \multicolumn{1}{c|}{\multirow{2}{*}{Всего}} \\ \cline{1-7}
				\multicolumn{1}{|l|}{\#} & \multicolumn{2}{l|}{} & \multicolumn{1}{c|}{I} & \multicolumn{1}{c|}{II} & \multicolumn{1}{c|}{III} & \multicolumn{1}{c|}{IV} & \multicolumn{1}{c|}{} \\ \hline		
				\multicolumn{1}{|l|}{1} & \multicolumn{2}{l|}{Остаток денежных средств на начало периода}   & 35000 & 145208 & 357560 & 641912 & 1179680 \\ \hline
				\multicolumn{1}{|l|}{2} & \multicolumn{2}{l|}{Поступление денежных средств} & 424800 & 531000 & 637200 & 743400 & 2336400 \\ \hline
				& 1 & Поступления от продажи продажи продукции  & 424800 & 531000 & 637200 & 743400 & 2336400 \\ \cline{2-8} 
				& 2 & Прочие поступления & 0 & 0 & 0 & 0 & 0 \\ \hline
				\multicolumn{1}{|l|}{3} & \multicolumn{2}{l|}{Всего наличие денежных средств} & 459800 & 676208 & 994760 & 1385312 & 3516080 \\ \hline
				\multicolumn{1}{|l|}{4} & \multicolumn{2}{l|}{Выбытие денежных средств} & 136680 & 136680 & 136680 & 136680 & 546720 \\ \hline
				& 1 & Оплата поставщикам & 0 & 0 & 0 & 0 & 0 \\ \cline{2-8} 
				& 2 & Оплата труда & 127680 & 127680 & 127680 & 127680 & 510720 \\ \cline{2-8} 
				& 3 & Оплата общепроизводственных расходов & 9000 & 9000 & 9000 & 9000 & 36000 \\ \cline{2-8} 
				& 4 & Оплата управленческих и коммерческих расходов & 0 & 0 & 0 & 0 & 0 \\ \hline
				\multicolumn{1}{|l|}{5} & \multicolumn{2}{l|}{Инвестиции} & 35000 & 0 & 0 & 0 & 35000 \\ \hline
				\multicolumn{1}{|l|}{6} & \multicolumn{2}{l|}{Уплата налогов} & 142912 & 181968 & 216168 & 250368 & 791416 \\ \hline
				& 1 & НДС & 64800 & 81000 & 97200 & 113400 & 356400 \\ \cline{2-8} 
				& 2 & Отчисления на соц. нужды & 38304 & 38304 & 38304 & 38304 & 153216 \\ \cline{2-8} 
				& 3 & Налог на прибыль  & 39808 & 62664 & 80664 & 98664 & 281800 \\ \cline{2-8} 
				& 4 & Прочие налоги & 0 & 0 & 0 & 0 & 0 \\ \hline
				\multicolumn{1}{|l|}{7} & \multicolumn{2}{l|}{Всего оттоков} & 314592 & 318648 & 352848 & 387048 & 1373136 \\ \hline
				\multicolumn{1}{|l|}{8} & \multicolumn{2}{l|}{Чистый денежный поток} & 145208 & 357560 & 641912 & 998264 & 2142944 \\ \hline
				\multicolumn{1}{|l|}{9} & \multicolumn{2}{l|}{Дополнительное финансирование} & 0 & 0 & 0 & 0 & 0 \\ \hline
				& 1 & Привлечение кредитов & 0 & 0 & 0 & 0 & 0 \\ \cline{2-8} 
				& 2 & Погашение кредитов & 0 & 0 & 0 & 0 & 0 \\ \cline{2-8} 
				& 3 & Погашение \%\% по кредиту & 0 & 0 & 0 & 0 & 0 \\ \hline
				\multicolumn{1}{|l|}{10} & \multicolumn{2}{l|}{Остаток денежных средств на конец периода} & 145208 & 357560 & 641912 & 998264 & 2142944 \\ \hline
			\end{tabular}
		}
	\end{table}
	
\subsection{Оценка эффективности реализации проекта} \label{subseq:performance}
	По имеющимся данным, будем брать за срок отчетности - один квартал. Тогда ср. чистая прибыль из таблицы  \ref{table:profit_and_loss}:
	\begin{equation}
	\overline{CF}_T = \dfrac{1}{T} \sum_{t = 1}^{T} CF_t = 281800
	\end{equation}
	
	Ставку дисконта примем за $R = 16\%$ в квартал, постаравшись вложить туда и высокие риски, и инфляцию. Если даже с таким R эффективнось будет высока, то сомневаться в рентабельности и выгодности инвестиций будет сложно.
	
	Единственные инвестиции I, которые мы получаем, это 35000 руб изначально требующегося капитала. Больше во внешних притоках мы не нуждаемся.
	
	Имея данные по таблицам \ref{table:profit_and_loss} и \ref{table:flow_of_funds}, считать будем за 1 год по каждому из 4-х кварталов.
	\subsubsection{Статические критерии}
	Простая норма рентабельности:
	\begin{equation}
	ROI = \dfrac{\overline{CF}}{I} = 8.0514
	\end{equation}
	
	Простой период окупаемости
	\begin{equation}
	T = \dfrac{I}{\overline{CF}} = 0.1242
	\end{equation}
	
	\subsubsection{Динамические критерии}
	Дисконтный период окупаемости:
	\begin{equation}
	DPP = \sum_{t = 0}^{T} \dfrac{CF_t}{1 + R^t} = -I + \sum_{t = 1}^{T} \dfrac{CF_t}{1 + R^t}
	\end{equation}
	Посчитав до T = 1 уже видно, что в I квартале инвестиции окупятся:
	\begin{equation}
	\begin{aligned}
	DPP_{T = 1} &= -35000 + \dfrac{159232}{1 + 0.16} = 102268.9655
	\end{aligned}				
	\end{equation} 
	Уточняя результат: $\dfrac{Balance}{CF_{disc\_I}} = \dfrac{35000}{159232} = 0.2198$ I квартала.
	
	Чистая текущая стоимость проекта:
	\begin{equation}
	\begin{aligned}
	NPV &= -I + \sum_{t = 1}^{T} \dfrac{CF_t}{(1 + R)^t} =\\
	&= -35000 + 137268.9655 + 186278.2402 + 206712.0423 + 217964.9955 =\\ 
	&= 713224.2435
	\end{aligned}
	\end{equation}	
	Критерий NPV > 0 является обязательным и в нашем случае он выполняется.
	
	Дисконтированный индекс рентабельности:
	В данном конкретном случае DPI будет равен обычному индексу рентабельности PI, поскольку команда нужнается только в первоначальных инвестициях.
	\begin{equation}
	DPI = \dfrac{NPV}{\sum_{t = 0}^{T} \dfrac{I_t}{(1 + R)^t}} = \dfrac{NPV}{I} = 20.3778
	\end{equation}
	
	Внутренняя норма рентабельности:
	IRR - это такое R, что NPV = 0. В итоге требуется решить уравнение:
	\begin{equation}
	-I + \sum_{t = 1}^{T} \dfrac{CF_t}{(1 + IRR)^t} = 0
	\end{equation}	
	Рассчитанное IRR = 504\%
