Производство ПО имеет достаточно большие и выигрышные особенности, по сравнению с производством большинства материальных благ и услуг. В частности:
\begin{itemize}
	\item 
		Почти все затраты в производстве ПО - на зарплату разработчиков. В данном случае не требуется даже оплачивать аренду, т.к. разработчик один и способен вести разработку по месту проживания. В итоге косвенных затрат почти нет, а постоянные состоят из выплат зарплат разработчикам.
	\item
		Процесс разработки ПО достаточно четко разбивается на этапы, т.к. уже есть множество наработок в этом направлении: как правильно устроить рабочий процесс.
	\item
		В нынешнее время цифрового распространения ПО так же избегаются затраты на транспортировку продукта до конечного пользователя.
\end{itemize}

\subsection{Описание производственного процесса}
Был выбран жизненный цикл ПО RAD (Rapid Application Development). В настоящий момент была написана основа решения, а так же одностраничный сайт, для продажи продукта. Все это было сделано за 14 дней, т.е. 112 рабочих часов. На данный момент разработка находится на этапе тестирования, после чего будет повторяться цикл из итераций с этапами, описанными в таблице \ref{table:lifecicle}.
 
\begin{table}[!h]
	\centering
	\small
	\def\arraystretch{1.3}
	\resizebox{\textwidth}{!}{%
		\begin{tabular}{|p{1ex}|p{4cm}|p{4cm}|p{2cm}|p{3cm}|}
			\hline
			\textbf{\#} & \textbf{Этап итерации} & \textbf{Результат} & \textbf{Сроки (дни)} & \textbf{Затраты (руб)}  \\ 
			\hline
			1 & Анализ полученных за предыдущую итерацию отзывов & Cписок функционала, который требуется реализовать & 2 & 3040 \\
			\hline
			2 & Анализ и проектирование решения задачи по реализации этого функционала & Набор алгоритмов, дизайн интерфейса, проектные диаграммы & 5 & 7600 \\
			\hline
			3 & Разработка прототипа & Прототип с реализованным, но не протестированным функционалом & 5-10 & 7600-15200 \\
			\hline
			4 & Внутреннее тестирование & Закрытие как можно большего количества критических ошибок за ограниченный период времени & 2 & 3040 \\
			\hline
			5 & Внешнее тестирование ограниченным числом пользователей с получением от них отзывов и автоматически генерируемых программой отчетов о выполнении работы программы & Внесение исправлений в элементы дизайна, проекта функционала, исправление выявленных в ходе внешнего тестирования ошибок & 10 & 15200 \\
			\hline
			6 & Подготовка к релизу и релиз с обновленным функционалом & Список изменений на сайте продукта и новостных ресурсах, уведомление клиентов об обновлении, готовая новая версия продукта, готовая для загрузки на клиентские машины & 1 & 1520\\
			\hline
			\hline
			$\sum$ & & & 25-35 & 38000-53200 \\
			\hline
		\end{tabular}
	}
	\caption{Этапы выполнения итерации жизненного цикла}
	\label{table:lifecicle}
\end{table}

\subsection{Оплата труда}
	Разработчиком является один человек, час работы каждого составляет 190 р. Итого, учитывая уже затарченные 112 часов, стоимость продукта равна 21,280 руб.
	
	За каждую итерацию в ЖЦ ПО по таблице \ref{table:lifecicle} на оплату труда будет уходить не меньше 53200 руб.
	
	В месяц на оплату труда будет уходить 42560 руб.
	
	Отчисления на соц.нужды (в месяц):
	\[
		42560 \cdot 0.3 = 12768 \text{\ руб.}
	\]
\subsection{Амортизационные отчисления}
	Каждый из разработчиков имеет ноутбук Dell Inspirion n7110, покупавшийся за 36,000 руб. с гарантией 2 года.  Ежемесячные амортизационные отчисления будут рассчитаны линейным образом и составят:
	\begin{align}
		H_{a} = \frac{1}{T} &= \frac{1}{24} = 0.42 \\
		A_{\text{месяц}} = C_{\text{ноутбуков}} \cdot H_a &= \frac {36,000 \cdot 2}{24} = 3,000\ \text{руб.}
	\end{align}
	
\subsection{Итоги} \label{subseq:production_plan_results}
	В итоге, месячная стоимость продукта составляет 58328 руб., а объем первоначальных инвестиций составит 
	\[
		24280 + 58328 = 109608\ \text{руб.}
	\]
