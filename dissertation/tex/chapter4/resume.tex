В данном документе рассмотрен бизнес-план по разработке и поддержке анализатора результатов тестирования производительности на основе Google Benchmark для проектов на C/C++.

Продукт ориентирован на C/C++-программистов, в большей степени на тех из них, кто занимается анализом и тестированием ПО.

Маркетинговая стратегия основывается на следующих идеях:
\begin{itemize}
	\item Оплата идет по месячной подписке.
	\item Рекламную функцию выполняет ограниченная в функционале бесплатная версия с открытым кодом, а так же новости о продукте на ресурсах общения целевой аудитории.
	\item Вытеснение конкурентов будет происходить узкой направленностью программы и низкой ценой, основанной на большом объеме продаж.
\end{itemize}

Команда разработчиков состоит всего из одного программиста. Были минимизированы косвенные затраты, используя бесплатные площадки для размещения и распространения продукта, практически не жертвуя при этом качеством. Это обуславливается небольшим размером и низкой сложностью самого продукта и способов его продвижения в современное время.

Для начала кампании требуются небольшие инвестиции, которые разработчик способен покрыть самостоятельно. Эти инвестиции окупаются спустя уже в первом квартале от начала кампании по наиболее реалистичному сценарию и приносят стабильный доход. Чистая текущая стоимость проекта в конце года гораздо больше 0 и составляет 713224.2435 руб. Внутренняя норма доходности по итогам года составляет 504\%. Дисконтный индекс рентабельности равен 20.3778.

Основные риски приходятся на колебание объема продаж.