\subsection{Потребители продукции}
	Разработка программного обеспечения -- одна из ведущих отраслей рынка развивающихся и развитых стран. Существует множество методологий, стандартов, принципов, которые позволяют систематизировать и поставить на конвейер производство программных продуктов. Один из важных этапов разработки ПО, без которого практически не обойтись при разработке крупной автоматизированной системы -- тестирование. На этом этапе важно правильно выделить критические для производительности всей системы элементы, иметь возможность в кратчайшие сроки реализовать тестовые случаи и получить понятный и удобный для анализа результат.
	
	Таким образом, круг заинтересованных в данном решении: программисты, использующие язык С++ как основной в своих проектах. Кроме того, т.к. анализатор в первую очередь используется для автоматизации рутины -- составление отчетов с таблицами и визуализацией данных, удобство работы с данными, анализ и сравнение производительности ПО -- задачи разработчиков, ответственных за качество ПО, то именно для этого, более узкого круга специалистов предназначенно разработанное приложение.
	
	Стоит отметить, что потребители данной продукции -- разработчики в секторе крупных разработчиков ПО, которые уделяют большое внимание контролю качества выпускаемого ПО. Именно на таких потребителей будет ориентирован платный вариант продукта. Сектор открытого программного обеспечения так же является заинтересованным лицом, т.к. может использовать и разрабатывать в своих интересах бесплатную версию. Оценить бесплатный сектор во-первых сложно (он черезвычайно хаотичен и по нему нету статистики как таковой), во-вторых ожидаемая польза от этого сектора в виде сторонних открытых разработок, которые мы сможем применить - это скорее возможный приятный бонус, нежели сколько-либо гарантируемая прибыль. Наиболее важная цель бесплатной версии с открытым кодом: маркетинг. Таким образом, сегмент открытого ПО не приносит нам прямой прибыли, но уменьшает расходы на рекламу.

\subsection{Основной сегмент рынка}
	Основным сегментом рынка, как следует из предыдущего раздела, является сектор крупных разработчиков ПО, основными модулями которого являются разработки на языке C/C++, проекты данной отрасли будут состоять из высоконагруженных и больших проектов, требующих внимательного подходу к качеству, производительности и устойчивости.

\subsection{Товарные особенности сегмента и позиционирование продукта}	
	Сегмент, работающий с подобными большими проектами хочет получить быстрый результат, минуя как можно больше рутинных задач.
	Именно так и будет позиционироваться плагин: решение конкретной задачи, вместо c одной стороны написания небольших временных внутрикорпоративных решений, и дорогих программных средств с перегруженным функционалом с другой. Предлагаемое решение за меньшую стоимость, дающий стандартизированный и понятный для подавляющего большинства программистов результат, удобный в использовании, а так же решение, запуск которого возможен на всех распространенных платформах.
	
\subsection{Каков спрос на продукт; каков потенциал рынка в целом и по секторам} \label{subseq:demand_for_product}
	Со стабильным трендом на C/С++ как язык разработки по данным рейтинга tiobe и растущим спросом на программистов, знающих эту технологию, за 4 года точно был большой рост потенциальной аудитории данного анализатора. Более того, абсолютно все аналитические сайты указывают на рост потребности в C/C++ разработчиках и дальше. По информации с того же рейтинга Tiobe, интерес к C/C++ по отдельности с 2000 года конкурировал лишь с языком Java. На данный момент С/С++ суммарно занимают 21.668\% от всех остальных технологий.
	
	По данным Google Trends запрос \enquote{С++ benchmark} имеет стабильно высокий спрос с 2014 года. В целом, способ маркетинга ПО, основанном на сравнительной статистике с производительностью конкурентов является выигрышным и распространенным. Формально, для этой цели может подойти любой продукт, если у разработчиков будет серьезное желание коммерциализировать проект.
	
	Для оценки потребительского спроса, можно использовать статистику по проектам на крупнейшем сайте разработчиков ПО с открытым исходным кодом: github.com. Согласно статистике, количество проектов, написанных преимущественно на C/C++ составляет 1.42 миллиона. Каждый из этих проектов - потенциальный клиент.
	
	Из-за сконцентрированности на поставленной задаче, особенностях маркетинга, наше решение будет дешевле и выгоднее для обоих сегментов: и открытого ПО, которые для своих проектов будут более чем способны заплатить за инструмент анализа, так и для больших компаний, которые будут ценить время и затраты человеческих ресурсов в разы дороже, чем стоимость анализатора. Для сравнения, средняя з/п в месяц системного архитектора по Москве за апрель - от 100 тыс. руб., т.е. 625 руб в час. В среднем, чтобы подробно разобраться с большим проектом, уходит 10 рабочих дней, что эквивалентно 80 человеко-часам. Следовательно, за только разбор большого проекта, не говоря о создании подробного отчета, работодатель потратит в среднем около 50 тыс. руб. В аутсорсинговой компании, с большим количеством \enquote{чужих} проектов и быстро текущих проектов, эта затрата будет отнюдь не единовременная. 
	
	Т.е. образом реализованный анализатор сэкономит затраты анализ результатов тестирований и подготовку отчетной информации. Поддержка проекта должна помочь потеснить с рынка менее качественных бесплатных конкурентов, а так же дорогих решений, предоставляющих слишком большой и, зачастую, ненужный функционал за большую плату. 