\chapter*{Введение}
\addcontentsline{toc}{chapter}{Введение}

Для программирования роботов доступно множество различных версий фреймворков с различными принципами работы, написанные на разных языках программирования и под разные платформы. В связи с тем, что появляются новые разработки, возникают новые задачи для автономных роботов, а так же технологии разработки ПО для них - возникает желание рассмотреть доступные и развивающиеся в данный момент решения и проанализировать с целью установки характеристик производительности и сравнения по полученным параметрам, чтобы разработчики могли обосновывать свой выбор при разработки приложений для автономных роботов. При этом необходимо учитывать как соответствие фреймворков возможным общим критериям (лицензия, статус разработки), так и важным для конкретной области: разработки ПО (программного обеспечения) для роботов. Для выполнения тестирования, следует определиться с тем, какие задачи, выполняемые фреймворком, являются значимыми для производительности системы в целом, какие компоненты системы требуется протестировать.

Для выполнения тестирования требуется использовать корректные инструменты. Тестирование производительности можно выполнять множеством различных методов, технические детали, лежащие в основе драйверов тестирования производительности различны, например, для разных архитектур процессоров. В данной работе рассматривается сравнение ряда фреймворков для автоматизации проведения тестирования производительности.

Составив план тестирования, определившись с тестовыми случаями и реализацией тестовых модулей, требовалось проанализировать полученные результаты и составить сравнительные выводы по итогам тестирования производительности.

\textit{Объектом исследования} в данной работе является множество ППО (промежуточного ПО) для разработки прикладного ПО автономных роботов.

\textit{Предметом исследования} является производительность подмножества наиболее доступного и используемого МАРППО (многоагентного робототехнического ППО).

\textit{Целью исследования} является получение результатов тестирования производительности для наиболее доступного и используемого \marm{}.