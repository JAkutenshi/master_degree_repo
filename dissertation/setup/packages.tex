%=== Настройка кодировок, шрифта и языка
\usepackage[utf8]{inputenc}
\usepackage{extsizes}
\usepackage[main=russian, english]{babel}
\usepackage[T2A, T1]{fontenc}
%\DisemulatePackage{setspace}
\usepackage{setspace}           % Интерлиньяж
\usepackage{geometry}           % Разметка документа
\usepackage{indentfirst}        % Красная строка с первого предложения
\usepackage{titlesec}	        % Форматирование заголовков
\usepackage{titletoc}           % Форматирование содержания
\usepackage{fontspec}           % Установка шрифта для XeLaTeX

%=== Таблицы
\usepackage{tabularx}	        % основной тип таблиц, выравнивание по ширине
\usepackage{longtable}	        % для таблиц, не вмещающихся на одну страницу
\usepackage{multirow}	        % для разбиения ячеек на несколько строк
\usepackage{multicol}	        % на несколько колонок

%=== Работа с формулами
\usepackage{amsmath}            % Набор пакетов, сильно расширяющих возможности по набору формул
\usepackage{amssymb}            % добавляет специфические для русских статей мат. символы вроде \leqslant
\usepackage{amsthm}	            % добавляет окружения для теорем и лемм	
\usepackage{mathtools}          % номера только для тех формул, на которые есть ссылки в тексте

%=== Разное
\usepackage{graphicx}           % Работа с изображениями
\usepackage[unicode]{hyperref}  % Работа с гиперссылками
\usepackage{pdflscape}          % Панорамное расположение страниц
\usepackage{ragged2e}           % Для установки --
\usepackage{microtype}          % -- осустствия переносов
\usepackage{color}				% Использование цветов (нужно для \todo)
\usepackage{listings}			% Листинги кода