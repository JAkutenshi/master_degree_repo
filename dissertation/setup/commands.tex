%=== Заголовок оглавления - содержания
\addto\captionsrussian{ 
	\renewcommand*\contentsname{\uppercase{Содержание}}
}

%=== Команда для удобной записи определения в соответствующем разделе ВКР
\newcommand{\termDef}[2]{#1 --- #2 \par}

%=== Выделение раздела красным если он не закончен
\newcommand{\todo}[1]{{\color{red} #1}}

%=== Вставка кода в строку
\newcommand{\inline}[1]{\colorbox{white}{\lstinline[basicstyle=\ttfamily\color{black}]|#1|}}

%=== Вставка изображения
\newcommand{\img}[4]{               %
\begin{figure}                      %
	\centering                      %
	\includegraphics[width=#4]{#1} %
	\caption{#2}                    %
	\label{#3}                      %
\end{figure}                        %
}

%=== Отдельные имена
\newcommand{\marm}{МАРППО}
\newcommand{\etc}{и т.д.}
\newcommand{\orocos}{ORoCoS}
\newcommand{\toolchain}{\orocos{} Toolchain}
\newcommand{\rtt}{\orocos{} RTT}


